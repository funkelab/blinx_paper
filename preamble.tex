\usepackage{amsmath}
\usepackage{amssymb}
\usepackage{graphicx}
\usepackage{ccaption}
\usepackage{caption}
\captionsetup{font=footnotesize}
\usepackage{gensymb}
\usepackage{tikz}
\usepackage{pgfplots}
\usepackage[textwidth=1.7cm,textsize=tiny,linecolor=orange!60!white,bordercolor=orange!60!white,color=orange!20!white]{todonotes}
\setlength{\marginparwidth}{1.7cm}
\usepackage{xspace}
\usepackage{soul}
\usepackage{siunitx}
\usepackage{natbib}
\usepackage{bm}
\usepackage[title]{appendix}
\usepackage[hidelinks]{hyperref}
\usetikzlibrary{external}
\tikzexternalize[prefix=figures/tikzexternal/]
\makeatletter
\renewcommand{\todo}[2][]{\tikzexternaldisable\@todo[#1]{#2}\tikzexternalenable}
\makeatother
% monochrome
\definecolor{funkey_bright}{HTML}{FFFFFF}
\definecolor{funkey_lightgrey}{HTML}{cccccc}
\definecolor{funkey_grey}{HTML}{666666}

% principal colors
\definecolor{funkey_color_1}{HTML}{834D9D}  % purple
\definecolor{funkey_color_2}{HTML}{F2A431}  % orange
\definecolor{funkey_color_3}{HTML}{55B849}  % green
\definecolor{funkey_color_4}{HTML}{DB8457}  % peach
\definecolor{funkey_color_5}{HTML}{8174B1}  % lavender
\definecolor{funkey_color_6}{HTML}{ADD8E6}  % aquamarine
\definecolor{funkey_color_7}{HTML}{000080}  % dark blue
\definecolor{funkey_color_8}{HTML}{800020}  % dark red
\definecolor{funkey_color_9}{HTML}{228B22}  % dark green
\definecolor{funkey_color_10}{HTML}{F5C108} % lemon
\definecolor{funkey_color_11}{HTML}{DA3074} % dirty pink
\definecolor{funkey_color_12}{HTML}{CC00FF} % electric purple
\definecolor{funkey_color_13}{HTML}{FF9900} % deep orange
\definecolor{funkey_color_14}{HTML}{006666} % teal blue
\definecolor{funkey_color_15}{HTML}{660066} % dark magenta

% background color
\colorlet{funkey_dark}{purple!10!black}

% how to use colors in funkey theme
\colorlet{funkey_highlight}{funkey_color_2}
\colorlet{funkey_textcolor}{funkey_bright}
\colorlet{funkey_bg}{funkey_dark}
\colorlet{funkey_alt_bg}{funkey_lightgrey}

\usetikzlibrary{arrows.meta}
\usetikzlibrary{calc}
\pgfplotsset{compat=1.18}

\tikzstyle{var}=[circle,draw,fill=orange!20!white]
\tikzstyle{arrow}=[->,-{Latex}]

\def\tikzmath#1#2{\tikz[remember picture,baseline=(#1.base),inner sep=0pt] \node (#1) {$\displaystyle #2$};}

\newenvironment{panel}[2]{%
  \begin{minipage}[t][][t]{#2}%
    \strut#1%

    \vspace{-\baselineskip}%
}{%
  \end{minipage}%
}%
\newenvironment{panelcolumn}[1]{%
  \begin{minipage}[t][][t]{#1}%
}{%
  \end{minipage}%
}%

% useful for setting/computing the width of plots
\newlength{\plotwidth}
\newlength{\plotheight}

% draw 5 example fluorophores, args are "on" or "off"
\def\blinky#1#2#3#4#5{
  \begin{tikzpicture}[node distance=3mm]
    \node[site#1] (s1) {};
    \node[site#2,right of=s1] (s2) {};
    \coordinate (s12) at ($(s1)!.5!(s2)$);
    \node[site#3,right of=s2] (s3) {};
    \node[site#4,below of=s12] (s4) {};
    \node[site#5,right of=s4] (s5) {};
    \foreach \angle in {0, 45, 90, 135, 180, 225, 270, 315}{
      \begin{scope}[rotate=\angle]
        \draw[sitelight#1] ($(s1)+(0,0.13)$) -- ($(s1)+(0,0.16)$);
        \draw[sitelight#2] ($(s2)+(0,0.13)$) -- ($(s2)+(0,0.16)$);
        \draw[sitelight#3] ($(s3)+(0,0.13)$) -- ($(s3)+(0,0.16)$);
        \draw[sitelight#4] ($(s4)+(0,0.13)$) -- ($(s4)+(0,0.16)$);
        \draw[sitelight#5] ($(s5)+(0,0.13)$) -- ($(s5)+(0,0.16)$);
      \end{scope}
    }
  \end{tikzpicture}
}

% continued caption texts
%
% Usage:
%\begin{figure}
%  \input{figure sources}
%  \caption{
%    ...
%    \captbc
%  }
%\end{figure}
%\begin{figure}
%  \contcaption{
%    \capcont
%    ...
%  }
%  \label{goes here}
%\end{figure}
\def\captbc{\emph{(caption continued on next page)}\xspace}
\def\capcont{\emph{(continued)\ }}


% references
\def\figref#1{Fig.~\ref{#1}}
\def\panelref#1{{\bf(#1)}}
\def\secref#1{Section~\ref{#1}}
\def\tabref#1{Table~\ref{#1}}
\def\eqref#1{(Eq.~\ref{#1})}

% common abbreviations
\newcommand{\eg}{\emph{e.g.}\xspace}
\newcommand{\ie}{\emph{i.e.}\xspace}


% introduce a new word
\def\introduce#1{{\color{funkey_color_1}\ul{#1}}}
%\def\introduce#1{#1}  % uncomment this for final version

\def\mathlet#1#2{\pgfmathparse{#2}\let#1\pgfmathresult}
