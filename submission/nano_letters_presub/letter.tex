\documentclass{article}
\usepackage[centering,tmargin=30mm,bmargin=20mm,lmargin=1.0in,rmargin=1.0in]{geometry}
\usepackage[hidelinks]{hyperref}
\usepackage{wallpaper}

\renewcommand{\familydefault}{\sfdefault}

\date{\today}
\makeatletter
\let\thedate\@date
\makeatother

\setlength\parindent{0mm}
\setlength\parskip{2mm}

\begin{document}

\ThisULCornerWallPaper{1}{janelia-letterhead.pdf}
\thispagestyle{empty}
Managing Editor\\
American Chemical Society\\
United States\\

\vspace{5mm}
Dear Managing Editor at Nano Letters,


We are writing to inquire about the suitability of our manuscript
``A Bayesian Solution to Count the Number of Molecules within a Diffraction Limited Spot''
for publication in Nano Letters. 
%
%We believe this manuscript is of great interest to you and your readers as it
%proposes a new method for quantifying biological phenomena at the nano-scale.
%
Our work contributes to the fields of super-resolution and single molecule
fluorescent imaging, and as such we believe it would be a good fit for Nano
Letters, given your journal's history of publishing on related work.
%
% We have noticed that your journal has a significant history of publishing on
% super-resolution and single-molecule fluorescence microscopy, and this manuscript 
% proposes a significant new method building on this body of work. 

Specifically, we propose a computational method to quantify the number of
blinking fluorescent emitters within a diffraction limited spot, a common
problem when imaging biology at the nano-scale.
%
We solve this problem by directly modeling the photophysics and kinetics over
time to produce a probabilistic estimate of the number of emitters.
%
Our method demonstrates a 2-fold increase in accuracy compared to the current
state of the art method (lbFCS, see Stein et al. 2019, ``Toward Absolute
Molecular Numbers in DNA-PAINT'', published in Nano Letters). Despite the
increase in accuracy, we also show that our method is applicable to a wider
range of experimental systems, which widens the applicability of this method
beyond the constraints of other methods like qPAINT.

We believe that being able to count the number of fluorescent emitters with
high accuracy will be beneficial for future super-resolution techniques: with
our method, samples can be imaged in a regime that is not restricted to
spatio-temporally separated blinking events of single emitters. Furthermore, an
accurate probabilistic count of the number of fluorescent emitters attached to
a single molecule will be instrumental in single molecule identification, e.g.,
for in-situ peptides sequencing.

We are aware that our manuscript would have to be reformatted to fit your
journal's word limits. We would therefore appreciate if you could let us know
if this work falls within the scope of your journal and whether we should
consider preparing a submission.

Please see our pre-print on bioRxiv for more information, and let us know if
you have any questions regarding the manuscript:
\url{https://www.biorxiv.org/content/10.1101/2024.04.18.590066v2.abstract}

Thank you very much, and we look forward to hearing from you.

\vspace{5mm}
Alex Hillsley,\\
Jan Funke\\
HHMI Janelia Research Campus

\end{document}
