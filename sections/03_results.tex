\section{Results}

\begin{figure*}
  \includegraphics[width=\linewidth]{figures/placeholders/figure_2_simulated_counting.png}
  \caption{A) Trace simulated from 10 emitters, and the posterior distribution estimated by blinx B) simulated from 20 emitters
  C) The blinx posterior is able to accurately estimate significantly higher molecular counts than the current state of the art, lbFCS
  D) As trace length increases, the variance of the blinx posterior decreases E) As the signal to noise ratio increases the variance of 
  the blinx posterior decreases. }
  \label{fig:method:overview}
\end{figure*}

\begin{figure*}

  \begin{panel}{(a)}{0.58\textwidth}
    \vspace{-2mm}%
    \small%
    \setlength\plotwidth{42mm}%
    \setlength\plotheight{60mm}%
    \def\kineticsheatmapcsv{figures/data/kinetics_grid/heatmap.csv}%
\tikzsetnextfilename{kinetics_sim_sweep}%
\begin{tikzpicture}
  \begin{axis}[
    width=\plotwidth,
    height=\plotheight,
    scale only axis=true,
    name=sweep,
    xlabel=\pon,
    ylabel=\poff,
    enlarge x limits=false,
    enlarge y limits=false,
    scaled ticks=false,
    ticklabel style={/pgf/number format/.cd,fixed,precision=3},
    xtick distance=0.04,
    xtick align=outside,
    xtick pos=lower,
    ytick align=outside,
    ytick pos=lower,
    colorbar,
  ]

    \pgfplotsset{
      colormap={posteriorcolormap}{
        color(0.0)=(funkey_color_2!50!white)
        color(0.1)=(funkey_color_2)
        color(0.9)=(funkey_color_1)
        color(1.0)=(funkey_color_1!50!black)
      }
    }

    \addplot[
      matrix plot*,
      mesh/cols=10,
      point meta=explicit,
    ] table [
      col sep=comma,
      x=p_on,
      y=p_off,
      meta=error,
    ] {\kineticsheatmapcsv};

    % highlight qpaint and lbfcs points
    \node[circle,draw=funkey_color_9,thick] (qpaint) at (axis cs:0.01, 0.2) {};
    \node[circle,draw=funkey_color_9,thick] (lbfcs) at (axis cs:0.02, 0.02) {};

  \end{axis}

  \node[anchor=north] (qpaint_zs) at ($(sweep.north east)+(2.8,0)$) {
    \def\zhistogramcsv{figures/data/kinetics_grid/state_histogram.csv}%
    \def\zhistogramcol{qpaint}%
    \def\noylabels{}%
    \setlength\plotwidth{14mm}%
    \setlength\plotheight{16mm}%
    \tikz{\@ifundefined{noylabels}{}{%
  \pgfplotsset{yticklabel=\empty}%
}%
\@ifundefined{noxlabel}{%
  \pgfplotsset{xlabel=\z{}}%
}{%
  \pgfplotsset{xlabel=\empty}%
}%
\begin{axis}[
  width=\plotwidth,
  height=\plotheight,
  scale only axis=true,
  grid=major,
  grid style={dashed, very thin},
  enlarge x limits=0.1,
  enlarge y limits={value=0.2,upper},
  ymin=0,
  scaled ticks=false,
]

  \addplot+[
    ybar,
    bar width=1,
    mark=none,
    fill=ztracecolor,
    fill opacity=0.6,
    draw=ztracecolor,
  ] table [
    col sep=comma,
    y=\zhistogramcol,
    x=bin,
  ] {\zhistogramcsv};

\end{axis}
}%
  };
  \node[fit=(qpaint_zs),inner sep=0,rounded corners,draw=funkey_color_9,thick] (qpaint_box) {};

  \node[anchor=south] (lbfcs_zs) at (sweep.south-|qpaint_zs.south) {
    \def\zhistogramcsv{figures/data/kinetics_grid/state_histogram.csv}%
    \def\zhistogramcol{lbfcs}%
    \def\noylabels{}%
    \setlength\plotwidth{14mm}%
    \setlength\plotheight{16mm}%
    \tikz{\@ifundefined{noylabels}{}{%
  \pgfplotsset{yticklabel=\empty}%
}%
\@ifundefined{noxlabel}{%
  \pgfplotsset{xlabel=\z{}}%
}{%
  \pgfplotsset{xlabel=\empty}%
}%
\begin{axis}[
  width=\plotwidth,
  height=\plotheight,
  scale only axis=true,
  grid=major,
  grid style={dashed, very thin},
  enlarge x limits=0.1,
  enlarge y limits={value=0.2,upper},
  ymin=0,
  scaled ticks=false,
]

  \addplot+[
    ybar,
    bar width=1,
    mark=none,
    fill=ztracecolor,
    fill opacity=0.6,
    draw=ztracecolor,
  ] table [
    col sep=comma,
    y=\zhistogramcol,
    x=bin,
  ] {\zhistogramcsv};

\end{axis}
}%
  };
  \node[fit=(lbfcs_zs),inner sep=0,rounded corners,draw=funkey_color_9,thick] (lbfcs_box) {};

  \draw[funkey_color_3,thick] (qpaint) -- (qpaint-|qpaint_box.west);
  \draw[funkey_color_3,thick] (lbfcs) -- (lbfcs-|lbfcs_box.west);

  \node[anchor=south] at (sweep.north) {\strut Expected Squared Error};

\end{tikzpicture}

  \end{panel}
  \begin{panel}{(b)}{0.42\textwidth}
    \vspace{-2mm}%
    \setlength\plotwidth{56mm}%
    \setlength\plotheight{60mm}%
    \input{figures/panels/kinetics_experiments.tikz}
  \end{panel}

  \begin{panel}{(c)}{\textwidth}
    \small%
    \setlength\plotwidth{\textwidth}%
    \setlength\plotheight{36mm}%
    \def\tracecsv{figures/data/kinetics_grid/traces.csv}%
\def\traceintensitycol{qpaint_trace}%
\def\zcol{qpaint_fit}%
\def\histogramcsv{figures/data/kinetics_grid/intensity_histogram.csv}%
\def\histbincol{qpaint_bins}%
\def\histcountcol{qpaint_measured}%
\def\modelfitcol{qpaint_model}%
\def\posteriorcsv{figures/data/kinetics_grid/posteriors.csv}%
\def\posteriorcol{qpaint}%
\tikzsetnextfilename{sim_trace_qpaint}%
\begin{tikzpicture}
  \def\plotlabel{$\n = 10$}
  \@ifundefined{nolabels}{
  \pgfplotsset{xlabel=frames}
  \pgfplotsset{ylabel=intensity}
  \pgfplotsset{grid=major}
  \pgfplotsset{grid style={dashed}}
}{
  \pgfplotsset{xticklabel=\empty}
  \pgfplotsset{yticklabel=\empty}
  \pgfplotsset{ticks=none}
  \pgfplotsset{scaled ticks=false}
}
\@ifundefined{histogramcsv}{}{
  \setlength{\plotwidth}{0.73\plotwidth}
}
\begin{axis}[
  name=trace,
  width=\plotwidth,
  height=\plotheight,
  scale only axis=true,
  enlarge x limits=false,
  xtick distance=500,
  ticklabel style={font=\small},
  legend style={nodes={scale=0.6, transform shape}},
]

  \addplot [
    color=tracecolor,
    mark=*,
    mark size=0.7pt,
    mark options={line width=0},
    fill opacity=0.8,
    draw opacity=0.0,
  ] table [
    col sep=comma,
    x=frames,
    y=\traceintensitycol
  ] {\tracecsv};
  \@ifundefined{nolabels}{\addlegendentry{intensity trace}}{}

  \@ifundefined{zcol}{}{
    \addplot [
      color=ztracecolor,
      thick
    ] table [
      col sep=comma,
      x=frames,
      y=\zcol
    ] {\tracecsv};
    \@ifundefined{nolabels}{\addlegendentry{inferred state}}{}
  }

  % remember min/max y-axis values for next plot
  \pgfplotsextra{
    \pgfmathparse{\pgfkeysvalueof{/pgfplots/ymin}}
    \global\let\ymin\pgfmathresult
    \pgfmathparse{\pgfkeysvalueof{/pgfplots/ymax}}
    \global\let\ymax\pgfmathresult
  }

\end{axis}
\@ifundefined{plotlabel}{}{
  \node[anchor=north west,rectangle,draw,fill=white] at (trace.north west) {\plotlabel};
}

\@ifundefined{histogramcsv}{}{
  \begin{axis}[
    at={($(trace.east) + (4mm,0)$)},
    name=histogram,
    anchor=west,
    width=0.25\plotwidth,
    height=\plotheight,
    scale only axis=true,
    ylabel=\empty,
    yticklabel=\empty,
    xtick distance=0.01,
    xlabel=probability,
    grid=major,
    grid style={dashed, very thin},
    enlarge x limits={value=0.4,upper},
    scaled ticks=false,
    ymin=\ymin,
    ymax=\ymax,
    ticklabel style={font=\small},
    legend style={nodes={scale=0.6, transform shape}},
  ]

    \addplot+[
      xbar interval,
      mark=none,
      color=tracecolor,
      fill=tracecolor,
      fill opacity=0.6,
      draw=none,
    ] table [
      col sep=comma,
      y=\histbincol,
      x=\histcountcol,
    ] {\histogramcsv};
    \addlegendentry{intensity histogram}

    \addplot[
      color=intensitymodelcolor!80!black,
      thick
    ] table [
      col sep=comma,
      y=\histbincol,
      x=\modelfitcol
    ] {\histogramcsv};
    \addlegendentry{inferred model}

  \end{axis}
}

  \def\noylabels{}
  \def\noxlabel{}
  \setlength\plotwidth{0.1\plotwidth}
  \setlength\plotheight{\plotwidth}
  \pgfplotsset{every axis/.style={
    anchor=below south east,
    at={($(histogram.south east)-(2mm,-4mm)$)},
    xtick distance=10
  }}
  \def\eps{0.001}%  skip bars for values below eps
\@ifundefined{noylabels}{}{
  \pgfplotsset{yticklabel=\empty}
  \pgfplotsset{ylabel=\empty}
}
\@ifundefined{noxlabel}{
  \pgfplotsset{xlabel=$n$}
}{
  \pgfplotsset{xlabel=\empty}
}
\begin{axis}[
  width=\plotwidth,
  height=\plotheight,
  scale only axis=true,
  enlarge x limits={abs=1.5},
  enlarge y limits=0,
  ymin=0,
  ymax=1,
  scaled ticks=false,
  ticklabel style={font=\tiny},
  xtick distance=1,
  axis background/.style={fill=white},
]

  \addplot+[
    ybar,
    bar width=1,
    mark=none,
    fill=posteriorcolor,
    fill opacity=0.6,
    draw=posteriorcolor,
    y filter/.expression={
      y < \eps ? nan : y
    },
  ] table [
    col sep=comma,
    y=\posteriorcol,
    x=n,
  ] {\posteriorcsv};

  \ifdefined\posteriorcolextra
    \addplot+[
      ybar,
      bar width=1,
      mark=none,
      fill=posteriorcolor!60!black,
      fill opacity=0.6,
      draw=posteriorcolor,
      y filter/.expression={
        y < \eps ? nan : y
      },
    ] table [
      col sep=comma,
      y=\posteriorcolextra,
      x=n,
    ] {\posteriorcsv};
  \fi

\end{axis}

\end{tikzpicture}
%
    \vspace{-4mm}%
  \end{panel}

  \begin{panel}{(d)}{\textwidth}
    \small%
    \setlength\plotwidth{\textwidth}%
    \setlength\plotheight{36mm}%
    \def\tracecsv{figures/data/kinetics_grid/traces.csv}%
\def\traceintensitycol{lbfcs_trace}%
\def\zcol{lbfcs_fit}%
\def\histogramcsv{figures/data/kinetics_grid/intensity_histogram.csv}%
\def\histbincol{lbfcs_bins}%
\def\histcountcol{lbfcs_measured}%
\def\modelfitcol{lbfcs_model}%
\def\posteriorcsv{figures/data/kinetics_grid/posteriors.csv}%
\def\posteriorcol{lbfcs}%
\tikzsetnextfilename{sim_trace_lbfcs}%
\begin{tikzpicture}
  \@ifundefined{nolabels}{
  \pgfplotsset{xlabel=frames}
  \pgfplotsset{ylabel=intensity}
  \pgfplotsset{grid=major}
  \pgfplotsset{grid style={dashed}}
}{
  \pgfplotsset{xticklabel=\empty}
  \pgfplotsset{yticklabel=\empty}
  \pgfplotsset{ticks=none}
  \pgfplotsset{scaled ticks=false}
}
\@ifundefined{histogramcsv}{}{
  \setlength{\plotwidth}{0.73\plotwidth}
}
\begin{axis}[
  name=trace,
  width=\plotwidth,
  height=\plotheight,
  scale only axis=true,
  enlarge x limits=false,
  xtick distance=500,
  ticklabel style={font=\small},
  legend style={nodes={scale=0.6, transform shape}},
]

  \addplot [
    color=tracecolor,
    mark=*,
    mark size=0.7pt,
    mark options={line width=0},
    fill opacity=0.8,
    draw opacity=0.0,
  ] table [
    col sep=comma,
    x=frames,
    y=\traceintensitycol
  ] {\tracecsv};
  \@ifundefined{nolabels}{\addlegendentry{intensity trace}}{}

  \@ifundefined{zcol}{}{
    \addplot [
      color=ztracecolor,
      thick
    ] table [
      col sep=comma,
      x=frames,
      y=\zcol
    ] {\tracecsv};
    \@ifundefined{nolabels}{\addlegendentry{inferred state}}{}
  }

  % remember min/max y-axis values for next plot
  \pgfplotsextra{
    \pgfmathparse{\pgfkeysvalueof{/pgfplots/ymin}}
    \global\let\ymin\pgfmathresult
    \pgfmathparse{\pgfkeysvalueof{/pgfplots/ymax}}
    \global\let\ymax\pgfmathresult
  }

\end{axis}
\@ifundefined{plotlabel}{}{
  \node[anchor=north west,rectangle,draw,fill=white] at (trace.north west) {\plotlabel};
}

\@ifundefined{histogramcsv}{}{
  \begin{axis}[
    at={($(trace.east) + (4mm,0)$)},
    name=histogram,
    anchor=west,
    width=0.25\plotwidth,
    height=\plotheight,
    scale only axis=true,
    ylabel=\empty,
    yticklabel=\empty,
    xtick distance=0.01,
    xlabel=probability,
    grid=major,
    grid style={dashed, very thin},
    enlarge x limits={value=0.4,upper},
    scaled ticks=false,
    ymin=\ymin,
    ymax=\ymax,
    ticklabel style={font=\small},
    legend style={nodes={scale=0.6, transform shape}},
  ]

    \addplot+[
      xbar interval,
      mark=none,
      color=tracecolor,
      fill=tracecolor,
      fill opacity=0.6,
      draw=none,
    ] table [
      col sep=comma,
      y=\histbincol,
      x=\histcountcol,
    ] {\histogramcsv};
    \addlegendentry{intensity histogram}

    \addplot[
      color=intensitymodelcolor!80!black,
      thick
    ] table [
      col sep=comma,
      y=\histbincol,
      x=\modelfitcol
    ] {\histogramcsv};
    \addlegendentry{inferred model}

  \end{axis}
}

  \def\noylabels{}
  \def\noxlabel{}
  \setlength\plotwidth{0.1\plotwidth}
  \setlength\plotheight{\plotwidth}
  \pgfplotsset{every axis/.style={anchor=below south east,at={($(histogram.south east)-(2mm,0)$)}}}
  \def\eps{0.001}%  skip bars for values below eps
\@ifundefined{noylabels}{}{
  \pgfplotsset{yticklabel=\empty}
  \pgfplotsset{ylabel=\empty}
}
\@ifundefined{noxlabel}{
  \pgfplotsset{xlabel=$n$}
}{
  \pgfplotsset{xlabel=\empty}
}
\begin{axis}[
  width=\plotwidth,
  height=\plotheight,
  scale only axis=true,
  enlarge x limits={abs=1.5},
  enlarge y limits=0,
  ymin=0,
  ymax=1,
  scaled ticks=false,
  ticklabel style={font=\tiny},
  xtick distance=1,
  axis background/.style={fill=white},
]

  \addplot+[
    ybar,
    bar width=1,
    mark=none,
    fill=posteriorcolor,
    fill opacity=0.6,
    draw=posteriorcolor,
    y filter/.expression={
      y < \eps ? nan : y
    },
  ] table [
    col sep=comma,
    y=\posteriorcol,
    x=n,
  ] {\posteriorcsv};

  \ifdefined\posteriorcolextra
    \addplot+[
      ybar,
      bar width=1,
      mark=none,
      fill=posteriorcolor!60!black,
      fill opacity=0.6,
      draw=posteriorcolor,
      y filter/.expression={
        y < \eps ? nan : y
      },
    ] table [
      col sep=comma,
      y=\posteriorcolextra,
      x=n,
    ] {\posteriorcsv};
  \fi

\end{axis}

\end{tikzpicture}

  \end{panel}

  \caption{%
    \panelref{a} The counting ability of \ours is dependant on the blinking kinetics, 
      showing increased model accuracy when $\pon > \poff$. 
      Top inset: when $\pon < \poff$, the distribution of
      observed \z{} states is shifted towards 0. Bottom inset: for
      $\pon = \poff$, the distribution of \z{} states is centered at $1/2$ $n$.
    %
    \panelref{b} Experimental blinking kinetics as a function of imager
      concentration (marker sizes) and temperature (color).
    \captbc
  }
\end{figure*}
\begin{figure*}
  \contcaption{
    \capcont
    %
    \panelref{c} Trace and \ours fit, \pon=0.01, \poff=0.2, and true count of $n$=10, representative of (a) top inset.
    %
    \panelref{d} Trace and \ours fit, \pon=0.02, \poff=0.02, and true count of $n$=10, representative of (a) bottom inset.
  }
  \label{fig:results:campare_kinetics}
\end{figure*}

\begin{figure*}

  \begin{panel}{(a)}{\textwidth}
    \hspace{-2mm}%
    \setlength\plotwidth{\textwidth}%
    \setlength\plotheight{36mm}%
    \input{figures/panels/qpaint_trace_n10.tikz}
    \vspace{-4mm}
  \end{panel}

  \begin{panel}{(b)}{\textwidth}
    \hspace{-2mm}%
    \setlength\plotwidth{\textwidth}%
    \setlength\plotheight{36mm}%
    \def\histogramcsv{figures/data/qpaint_kinetics/intensity_histogram.csv}%
\def\tracecsv{figures/data/qpaint_kinetics/trace_and_fit_N20.csv}%
\def\traceintensitycol{trace}%
\def\zcol{fit}%
\def\histbincol{N20_bins}%
\def\histcountcol{N20_measured}%
\def\modelfitcol{N20_model}%
\def\posteriorcsv{figures/data/qpaint_kinetics/posteriors.csv}%
\def\posteriorcol{posterior_20}%
\tikzsetnextfilename{qpaint_trace_n20}%
\begin{tikzpicture}
  \@ifundefined{nolabels}{
  \pgfplotsset{xlabel=frames}
  \pgfplotsset{ylabel=intensity}
  \pgfplotsset{grid=major}
  \pgfplotsset{grid style={dashed}}
}{
  \pgfplotsset{xticklabel=\empty}
  \pgfplotsset{yticklabel=\empty}
  \pgfplotsset{ticks=none}
  \pgfplotsset{scaled ticks=false}
}
\@ifundefined{histogramcsv}{}{
  \setlength{\plotwidth}{0.73\plotwidth}
}
\begin{axis}[
  name=trace,
  width=\plotwidth,
  height=\plotheight,
  scale only axis=true,
  enlarge x limits=false,
  xtick distance=500,
  ticklabel style={font=\small},
  legend style={nodes={scale=0.6, transform shape}},
]

  \addplot [
    color=tracecolor,
    mark=*,
    mark size=0.7pt,
    mark options={line width=0},
    fill opacity=0.8,
    draw opacity=0.0,
  ] table [
    col sep=comma,
    x=frames,
    y=\traceintensitycol
  ] {\tracecsv};
  \@ifundefined{nolabels}{\addlegendentry{intensity trace}}{}

  \@ifundefined{zcol}{}{
    \addplot [
      color=ztracecolor,
      thick
    ] table [
      col sep=comma,
      x=frames,
      y=\zcol
    ] {\tracecsv};
    \@ifundefined{nolabels}{\addlegendentry{inferred state}}{}
  }

  % remember min/max y-axis values for next plot
  \pgfplotsextra{
    \pgfmathparse{\pgfkeysvalueof{/pgfplots/ymin}}
    \global\let\ymin\pgfmathresult
    \pgfmathparse{\pgfkeysvalueof{/pgfplots/ymax}}
    \global\let\ymax\pgfmathresult
  }

\end{axis}
\@ifundefined{plotlabel}{}{
  \node[anchor=north west,rectangle,draw,fill=white] at (trace.north west) {\plotlabel};
}

\@ifundefined{histogramcsv}{}{
  \begin{axis}[
    at={($(trace.east) + (4mm,0)$)},
    name=histogram,
    anchor=west,
    width=0.25\plotwidth,
    height=\plotheight,
    scale only axis=true,
    ylabel=\empty,
    yticklabel=\empty,
    xtick distance=0.01,
    xlabel=probability,
    grid=major,
    grid style={dashed, very thin},
    enlarge x limits={value=0.4,upper},
    scaled ticks=false,
    ymin=\ymin,
    ymax=\ymax,
    ticklabel style={font=\small},
    legend style={nodes={scale=0.6, transform shape}},
  ]

    \addplot+[
      xbar interval,
      mark=none,
      color=tracecolor,
      fill=tracecolor,
      fill opacity=0.6,
      draw=none,
    ] table [
      col sep=comma,
      y=\histbincol,
      x=\histcountcol,
    ] {\histogramcsv};
    \addlegendentry{intensity histogram}

    \addplot[
      color=intensitymodelcolor!80!black,
      thick
    ] table [
      col sep=comma,
      y=\histbincol,
      x=\modelfitcol
    ] {\histogramcsv};
    \addlegendentry{inferred model}

  \end{axis}
}

  \def\noylabels{}
  \def\noxlabel{}
  \setlength\plotwidth{0.1\plotwidth}
  \setlength\plotheight{\plotwidth}
  \tikzset{every axis/.style={anchor=below south east,at={($(histogram.south east)-(2mm,0)$)}}}
  \def\eps{0.001}%  skip bars for values below eps
\@ifundefined{noylabels}{}{
  \pgfplotsset{yticklabel=\empty}
  \pgfplotsset{ylabel=\empty}
}
\@ifundefined{noxlabel}{
  \pgfplotsset{xlabel=$n$}
}{
  \pgfplotsset{xlabel=\empty}
}
\begin{axis}[
  width=\plotwidth,
  height=\plotheight,
  scale only axis=true,
  enlarge x limits={abs=1.5},
  enlarge y limits=0,
  ymin=0,
  ymax=1,
  scaled ticks=false,
  ticklabel style={font=\tiny},
  xtick distance=1,
  axis background/.style={fill=white},
]

  \addplot+[
    ybar,
    bar width=1,
    mark=none,
    fill=posteriorcolor,
    fill opacity=0.6,
    draw=posteriorcolor,
    y filter/.expression={
      y < \eps ? nan : y
    },
  ] table [
    col sep=comma,
    y=\posteriorcol,
    x=n,
  ] {\posteriorcsv};

  \ifdefined\posteriorcolextra
    \addplot+[
      ybar,
      bar width=1,
      mark=none,
      fill=posteriorcolor!60!black,
      fill opacity=0.6,
      draw=posteriorcolor,
      y filter/.expression={
        y < \eps ? nan : y
      },
    ] table [
      col sep=comma,
      y=\posteriorcolextra,
      x=n,
    ] {\posteriorcsv};
  \fi

\end{axis}

\end{tikzpicture}

    \vspace{-4mm}
  \end{panel}

  \begin{panel}{(c)}{\textwidth}
    \setlength\plotwidth{72mm}%
    \setlength\plotheight{72mm}%
    \def\posteriormatrixcsv{figures/data/qpaint_kinetics/heatmap.csv}%
\def\qpaintcsv{figures/data/qpaint_kinetics/qpaint_counts.csv}%
\tikzsetnextfilename{qpaint_counting_estimates}%
\begin{tikzpicture}
  \begin{axis}[
    name=trace,
    width=\plotwidth,
    height=\plotheight,
    xlabel=true \n,
    ylabel=estimated \n,
    enlarge x limits=false,
    enlarge y limits=false,
    grid=major,
    grid style={dashed},
    scaled ticks=false,
    ticklabel style={font=\small},
    xtick align=outside,
    xtick pos=lower,
    ytick align=outside,
    ytick pos=lower,
    colorbar,
  ]

    \pgfplotsset{
      colormap={posteriorcolormap}{
        color(0.0)=(white)
        color(0.2)=(funkey_color_2)
        color(1.0)=(funkey_color_2!50!black)
      }
    }

    \addplot[
      matrix plot*,
      mesh/cols=30,
      point meta=explicit,
    ] table [
      col sep=comma,
      x=n_true,
      y=n_fit,
      meta=posterior,
    ] {\posteriormatrixcsv};

  \end{axis}

  \begin{axis}[
    name=trace,
    width=\plotwidth,
    height=\plotheight,
    xmin=0.5,
    xmax=30.5,
    ymin=0.5,
    ymax=35.5,
    grid=major,
    grid style={dashed},
    xticklabel=\empty,
    yticklabel=\empty,
    scaled ticks=false,
    domain=0:30,
    ticklabel style={font=\small},
    legend pos=north west,
  ]

    \addlegendimage{mark=square*,only marks,funkey_color_2}
    \addlegendentry{\ours}
    \addlegendimage{mark=*,only marks,funkey_color_1}
    \addlegendentry{\qpaint}

    \addplot[
      mark=*,
      only marks,
      mark size=1.4pt,
      mark options={draw=white,fill=funkey_color_1,draw opacity=0.6,fill opacity=0.7},
    ] table [
      col sep=comma,
      x=n,
      y=qpaint_count,
    ] {\qpaintcsv};

    \addplot[no marks,thick,gray] {x};

  \end{axis}

\end{tikzpicture}

  \end{panel}

  \caption{
    \panelref{a, b} trace simulated with 10/20 emitters, \pon=0.006, \poff=0.2, 
    and the posterior distribution estimated by \ours
    %
    \panelref{c} At higher molecular counts ($n > 15$) \qpaint begins to underestimate the count, while 
    the posterior of \ours remains accurate.
  }
  \label{fig:results:qpaint_counting}
\end{figure*}


\begin{figure*}[t]

  \begin{panel}{(a)}{\textwidth}
    \hspace{-2mm}%
    \setlength\plotwidth{\textwidth}%
    \setlength\plotheight{36mm}%
    \input{figures/panels/exp_trace_n4.tikz}
    \vspace{-4mm}
  \end{panel}

  \begin{panel}{(b)}{\textwidth}
    \hspace{-2mm}%
    \setlength\plotwidth{\textwidth}%
    \setlength\plotheight{36mm}%
    \def\histogramcsv{figures/data/exp_counting/intensity_histograms.csv}%
\def\tracecsv{figures/data/exp_counting/traces_and_fits.csv}%
\def\posteriorcsv{figures/data/exp_counting/posteriors.csv}%
\def\traceintensitycol{N1_trace_good}%
\def\zcol{N1_fit_good}%
\def\histbincol{N1_good_bins}%
\def\histcountcol{N1_good_measured}%
\def\modelfitcol{N1_good_model}%
\def\posteriorcol{posterior_1}%
\tikzsetnextfilename{exp_trace_n1}%
\begin{tikzpicture}
  \@ifundefined{nolabels}{
  \pgfplotsset{xlabel=frames}
  \pgfplotsset{ylabel=intensity}
  \pgfplotsset{grid=major}
  \pgfplotsset{grid style={dashed}}
}{
  \pgfplotsset{xticklabel=\empty}
  \pgfplotsset{yticklabel=\empty}
  \pgfplotsset{ticks=none}
  \pgfplotsset{scaled ticks=false}
}
\@ifundefined{histogramcsv}{}{
  \setlength{\plotwidth}{0.73\plotwidth}
}
\begin{axis}[
  name=trace,
  width=\plotwidth,
  height=\plotheight,
  scale only axis=true,
  enlarge x limits=false,
  xtick distance=500,
  ticklabel style={font=\small},
  legend style={nodes={scale=0.6, transform shape}},
]

  \addplot [
    color=tracecolor,
    mark=*,
    mark size=0.7pt,
    mark options={line width=0},
    fill opacity=0.8,
    draw opacity=0.0,
  ] table [
    col sep=comma,
    x=frames,
    y=\traceintensitycol
  ] {\tracecsv};
  \@ifundefined{nolabels}{\addlegendentry{intensity trace}}{}

  \@ifundefined{zcol}{}{
    \addplot [
      color=ztracecolor,
      thick
    ] table [
      col sep=comma,
      x=frames,
      y=\zcol
    ] {\tracecsv};
    \@ifundefined{nolabels}{\addlegendentry{inferred state}}{}
  }

  % remember min/max y-axis values for next plot
  \pgfplotsextra{
    \pgfmathparse{\pgfkeysvalueof{/pgfplots/ymin}}
    \global\let\ymin\pgfmathresult
    \pgfmathparse{\pgfkeysvalueof{/pgfplots/ymax}}
    \global\let\ymax\pgfmathresult
  }

\end{axis}
\@ifundefined{plotlabel}{}{
  \node[anchor=north west,rectangle,draw,fill=white] at (trace.north west) {\plotlabel};
}

\@ifundefined{histogramcsv}{}{
  \begin{axis}[
    at={($(trace.east) + (4mm,0)$)},
    name=histogram,
    anchor=west,
    width=0.25\plotwidth,
    height=\plotheight,
    scale only axis=true,
    ylabel=\empty,
    yticklabel=\empty,
    xtick distance=0.01,
    xlabel=probability,
    grid=major,
    grid style={dashed, very thin},
    enlarge x limits={value=0.4,upper},
    scaled ticks=false,
    ymin=\ymin,
    ymax=\ymax,
    ticklabel style={font=\small},
    legend style={nodes={scale=0.6, transform shape}},
  ]

    \addplot+[
      xbar interval,
      mark=none,
      color=tracecolor,
      fill=tracecolor,
      fill opacity=0.6,
      draw=none,
    ] table [
      col sep=comma,
      y=\histbincol,
      x=\histcountcol,
    ] {\histogramcsv};
    \addlegendentry{intensity histogram}

    \addplot[
      color=intensitymodelcolor!80!black,
      thick
    ] table [
      col sep=comma,
      y=\histbincol,
      x=\modelfitcol
    ] {\histogramcsv};
    \addlegendentry{inferred model}

  \end{axis}
}

  \def\noylabels{}
  \def\noxlabel{}
  \setlength\plotwidth{0.1\plotwidth}
  \setlength\plotheight{\plotwidth}
  \pgfplotsset{every axis/.style={anchor=below south east,at={($(histogram.south east)-(2mm,2mm)$)}}}
  \def\eps{0.001}%  skip bars for values below eps
\@ifundefined{noylabels}{}{
  \pgfplotsset{yticklabel=\empty}
  \pgfplotsset{ylabel=\empty}
}
\@ifundefined{noxlabel}{
  \pgfplotsset{xlabel=$n$}
}{
  \pgfplotsset{xlabel=\empty}
}
\begin{axis}[
  width=\plotwidth,
  height=\plotheight,
  scale only axis=true,
  enlarge x limits={abs=1.5},
  enlarge y limits=0,
  ymin=0,
  ymax=1,
  scaled ticks=false,
  ticklabel style={font=\tiny},
  xtick distance=1,
  axis background/.style={fill=white},
]

  \addplot+[
    ybar,
    bar width=1,
    mark=none,
    fill=posteriorcolor,
    fill opacity=0.6,
    draw=posteriorcolor,
    y filter/.expression={
      y < \eps ? nan : y
    },
  ] table [
    col sep=comma,
    y=\posteriorcol,
    x=n,
  ] {\posteriorcsv};

  \ifdefined\posteriorcolextra
    \addplot+[
      ybar,
      bar width=1,
      mark=none,
      fill=posteriorcolor!60!black,
      fill opacity=0.6,
      draw=posteriorcolor,
      y filter/.expression={
        y < \eps ? nan : y
      },
    ] table [
      col sep=comma,
      y=\posteriorcolextra,
      x=n,
    ] {\posteriorcsv};
  \fi

\end{axis}

\end{tikzpicture}

    \vspace{-4mm}
  \end{panel}

  \begin{panel}{(c)}{0.59\textwidth}
    \vspace{6mm}%
    \hspace{4mm}%
    \setlength\plotwidth{22mm}%
    \tikzsetnextfilename{exp_origami_samples_n4}%
\begin{tikzpicture}

  \matrix[
      matrix of nodes,
      every node/.style={inner sep=0},
      column sep=1mm,
      row sep=1mm,
      ampersand replacement=\&] (samples) {
    \includegraphics[width=\plotwidth]{figures/data/exp_counting/4bs_origami_1.png}
    \&
    \includegraphics[width=\plotwidth]{figures/data/exp_counting/4bs_origami_2.png}
    \&
    \includegraphics[width=\plotwidth]{figures/data/exp_counting/4bs_origami_3.png}
    \&
    \includegraphics[width=\plotwidth]{figures/data/exp_counting/4bs_origami_4.png}
    \\
  };

  % scale bar
  \node[
    rectangle,
    minimum width=0.25\plotwidth, % images are 400px, 1/4 is 100px and that is 20nm
    fill=white,
    anchor=south east,
    inner sep=0pt,
    outer sep=2mm] (scalebar) at (samples-1-4.south east) {};
  \node[anchor=south,white] at (scalebar.center) {\tiny $20 \micrometer$};
\end{tikzpicture}
%
  \end{panel}
  \begin{panel}{(d)}{0.2\textwidth}
    \hspace{2mm}%
    \def\plotwidth{0.8\textwidth}%
    \def\plotheight{20mm}%
    \tikzsetnextfilename{exp_mean_probabilities_n1}%
\begin{tikzpicture}

  \pgfplotsset{every axis/.style={name=n1}}
  \def\posteriorcsv{figures/data/exp_counting/mean_probabilities.csv}
  \def\posteriorcol{n1_prob}
  \def\eps{0.001}%  skip bars for values below eps
\@ifundefined{noylabels}{}{
  \pgfplotsset{yticklabel=\empty}
  \pgfplotsset{ylabel=\empty}
}
\@ifundefined{noxlabel}{
  \pgfplotsset{xlabel=$n$}
}{
  \pgfplotsset{xlabel=\empty}
}
\begin{axis}[
  width=\plotwidth,
  height=\plotheight,
  scale only axis=true,
  enlarge x limits={abs=1.5},
  enlarge y limits=0,
  ymin=0,
  ymax=1,
  scaled ticks=false,
  ticklabel style={font=\tiny},
  xtick distance=1,
  axis background/.style={fill=white},
]

  \addplot+[
    ybar,
    bar width=1,
    mark=none,
    fill=posteriorcolor,
    fill opacity=0.6,
    draw=posteriorcolor,
    y filter/.expression={
      y < \eps ? nan : y
    },
  ] table [
    col sep=comma,
    y=\posteriorcol,
    x=n,
  ] {\posteriorcsv};

  \ifdefined\posteriorcolextra
    \addplot+[
      ybar,
      bar width=1,
      mark=none,
      fill=posteriorcolor!60!black,
      fill opacity=0.6,
      draw=posteriorcolor,
      y filter/.expression={
        y < \eps ? nan : y
      },
    ] table [
      col sep=comma,
      y=\posteriorcolextra,
      x=n,
    ] {\posteriorcsv};
  \fi

\end{axis}

  \node[anchor=south] at (n1.outer north) {origami with $\n=1$};

\end{tikzpicture}
%
  \end{panel}
  \begin{panel}{(e)}{0.2\textwidth}
    \hspace{2mm}%
    \def\plotwidth{0.8\textwidth}%
    \def\plotheight{20mm}%
    \tikzsetnextfilename{exp_mean_probabilities_n4}%
\begin{tikzpicture}

  \pgfplotsset{every axis/.style={name=n4}}
  \def\posteriorcsv{figures/data/exp_counting/mean_probabilities.csv}
  \def\posteriorcol{n4_prob}
  \def\eps{0.001}%  skip bars for values below eps
\@ifundefined{noylabels}{}{
  \pgfplotsset{yticklabel=\empty}
  \pgfplotsset{ylabel=\empty}
}
\@ifundefined{noxlabel}{
  \pgfplotsset{xlabel=$n$}
}{
  \pgfplotsset{xlabel=\empty}
}
\begin{axis}[
  width=\plotwidth,
  height=\plotheight,
  scale only axis=true,
  enlarge x limits={abs=1.5},
  enlarge y limits=0,
  ymin=0,
  ymax=1,
  scaled ticks=false,
  ticklabel style={font=\tiny},
  xtick distance=1,
  axis background/.style={fill=white},
]

  \addplot+[
    ybar,
    bar width=1,
    mark=none,
    fill=posteriorcolor,
    fill opacity=0.6,
    draw=posteriorcolor,
    y filter/.expression={
      y < \eps ? nan : y
    },
  ] table [
    col sep=comma,
    y=\posteriorcol,
    x=n,
  ] {\posteriorcsv};

  \ifdefined\posteriorcolextra
    \addplot+[
      ybar,
      bar width=1,
      mark=none,
      fill=posteriorcolor!60!black,
      fill opacity=0.6,
      draw=posteriorcolor,
      y filter/.expression={
        y < \eps ? nan : y
      },
    ] table [
      col sep=comma,
      y=\posteriorcolextra,
      x=n,
    ] {\posteriorcsv};
  \fi

\end{axis}

  \node[anchor=south] at (n4.outer north) {origami with $\n=4$};

\end{tikzpicture}
%
  \end{panel}

  \caption{TODO}
  \label{fig:results:experimental}
\end{figure*}


%-------------
\subsection{Simulated Experiments}
%-------------
% Simulated traces based on experimentally relevant parameters
Running \ours as a forward model to simulate experiment, 
we generated traces for counts $\n=1$ to 30 (\figref{fig:results:sim_counting}a,b).
	%
	We determined experimentally relevant camera parameters from a calibration of the
	sCMOS camera (\parametersc: $\camgain=2.17$, $\camoffset=4791$, $\camvar=774$). 
	%
	We determined kinetic (\parameterst: $\pon=0.036$, $\poff=0.028$), and emission parameters 
	(\parameterse: $\re=2.79$, $\rb=6.77$) from an initial experiment, 
	which closely match those reported in literature \cite{stein_2021}.
	%
	We generated traces for 4000 frames, corresponding to an imaging time of 14 minutes.

% priors were placed on camera parameters, but all others were uniform
For inference, we placed tight empirically determined, Gaussian priors on 
the camera parameters.
	%
	We also placed flat uniform priors on the kinetic parameters, allowing the model 
	to fit blinking rates without external bias,
	%
	and loose priors on the emission parameters (\rb, \re) to prevent 
	over-estimation of count (see \secref{discussion}). 
	% how we got these priors
	We experimentally determined the prior on \rb by averaging the 
	intensity of the pixels immediately outside the region of interest.
	%
	We estimated the prior on \re  by first fitting a count of $\n=1$ to the trace.
	%
	The distribution of all observed $z$-states is unimodal, 
	therefore, whichever \z{}-state is the most common, the second most common 
	will be $\z{} \pm 1$. 
	%
	We observed that fitting $n=1$ captures this specific transition, and provides a good
	estimate of photon emission rate of a single emitter \re.
	%
	
% blinx can count!
\ours accurately fit simulated traces with counts up to $\n=30$ (\figref{fig:results:sim_counting}c).
	%
	For $\n=10$ and below, \ours estimated the true count with a substantially 
	higher likelihood than all other possibilities. 
	%
	Above $\n=10$ the posterior broadened as other possibilities become more likely. 
	%
	The most likely estimate was the true count in 177/300 (59\%) of traces, and 
	was within 1 of the true count in 267/300 (89\%) of all traces.
	%
	Importantly, in many cases \ours is able to fit the true count despite the fact that 
	not every state is occupied \ie the model is not just counting the 
	number of peaks in the intensity histogram (\figref{fig:results:sim_counting}a,b).
	% blinx outperforms lbFCS, counting up to 30 
	This is a tripling in performance compared to \lbfcs, which counted 
	accurately up to $\n=6$ and was within 1 of the true count for 64/300 (21\%) of traces, 
	but failed to estimate higher counts.
	%
	Upon further analysis, the main limitation of lbFCS is the estimation of the 
	intensity of a single emitter, corresponding to \re in \ours. 
	%
	\lbfcs relies on a histogram based method to determine this value, while 
	\ours is able to jointly optimize this parameter with the others.
	%
	Providing this value to \lbfcs led to a substantial recovery of performance.
	% 
	With \re provided, \lbfcs identified 200/300 (67\%) of traces within 1 of the true count.
	%

% Longer traces lead to better fits
A longer trace provides more
information to the model and increases the counting accuracy.
	%
	We simulated and fit traces of lengths ranging from 100 to 10,000 frames, with the same parameters 
	as the previous section (\figref{fig:results:sim_counting}d).
	%
	As expected, accuracy increased and variance of the posterior distribution decreased, 
	as trace length increased.
	% Also notice a underestimation of count with low trace length
	Interestingly for short trace lengths, \ours consistently underestimated the true count. 
	% This is most likely due to the system only occupying a subset of all possible zs in this short time
	This is likely because only a subset of possible states were observed in this short time. % maybe more explanation?

% there is a lower bound to SNR, below a specific SNR, model maximally overcounts
We determined the robustness of our model and the minimum signal-to-noise ratio 
(SNR) needed to achieve an accurate count.
	%
	We artificially increased the variance of the readout noise \camvar, 
	to simulate traces with SNRs ranging from 9 to 1 (where 9 
	corresponds to the parameters previously used)
	%
	Because noise in our model is a function of intensity, quantifying the SNR of a trace 
	is not trivial. 
	%
	For simplicity, here we define SNR as the difference in intensity between 
	the first two states (\z{0} and \z{1}) divided by the standard deviation of the difference. 
	%
	In effect this is the higher bound of SNR for a given trace. 
	%
	\ours shows accurate counting for SNRs $\geq 4$ (\figref{fig:results:sim_counting}e). 
	%
	Interestingly, for  SNR $\leq 4$, \ours estimates the count as 25, 
	the highest \n tested, no matter the true count. 
	%
	This is due to the model adding states to compensate for the wide distribution of intensities  
	(see \secref{discussion}).

%-------------
\subsection{Effect of kinetic parameters}
%-------------
The performance of our model depends on the true kinetic parameters of the system (\pon and \poff).
%
	To find a regime that maximizes our models counting ability, we simulated and fit traces 
	from $\n=1$ to 20, with a range of kinetic parameters, 
	while holding all other parameters constant.
	%
	We ensured that this range covered experimentally relevant kinetics from literature including:
	qPAINT: (\pon, \poff) = (0.006, 0.2) \cite{jungmann_2016} and lbFCS (0.02, 0.02) \cite{stein_2021}. 
	%
	We summarized the resulting posteriors by calculating the expected squared error over all true counts: 
	$\sum w(\truen - \n)^2$, where $w$ is the probability of the estimated count \n 
	(\figref{fig:results:campare_kinetics}a).

The result can be divided into two clear regimes: $\pon < \poff$ and $\pon \geq \poff$.
	%
	In the first regime, where $\pon < \poff$ we see significant limitations to our models counting ability.
	%
	Furthermore, when \pon is substantially lower than \poff, our model loses almost 
	all ability to count and estimates $\truen=1$ or 2, regardless of the true $\truen$.
	%
	In contrast, in the second regime, where $\pon \geq \poff$ we see a heightened ability of our model, 
	and accurate counting up to $n=20$.
	%
	A clear difference between these two regimes can be seen in the distribution 
	of their hidden states \z{} (\figref{fig:results:campare_kinetics}a, insets).
	%
	When $\pon < \poff$, the distribution is shifted towards 0, and a majority 
	of the time is spent in the lowest two \z{}-states (\figref{fig:results:campare_kinetics}c). 
	In this regime, the true count becomes indistinguishable without prior information. 
	%
	When $\pon \geq \poff$, this distribution is centered, or even shifted towards $n$ 
	and a larger portion of the states are visited (\figref{fig:results:campare_kinetics}d). 
	%
	This provides ample information for the model to infer the correct $n$.
	
% Experimental
% Do we need to explain how DNA-PAINT works
In DNA-PAINT, the blinking rate is determined by the kinetics of single-stranded DNA binding,
which in turn are dependent on experimental conditions such as temperature and concentration, 
and the specific DNA sequence~\citep{jungmann_single-molecule_2010}.
	%
	As a result, the kinetic parameters \pon and \poff of our model can be 
	tuned by adjusting the temperature and imager concentration (\figref{fig:results:campare_kinetics}b).
	%
	We imaged DNA-origami with a known count of $\n=1$ at 25\textdegree C and an 
	imager concentration of 10 nM, we measured $\pon=0.028$ and $\poff=0.072$ 
	(\figref{fig:results:campare_kinetics}b),
	%
	which places these conditions firmly in the poor counting 
	accuracy regime of $\pon < \poff$ (\figref{fig:results:campare_kinetics}a). % fix
	%
	In order to move to a more favorable kinetic regime, 
	we increased the imager concentration and decreased the temperature.
	%
	Increasing imager concentration to 30 nM raised \pon to 0.071 and 
	decreasing temperature to 13\textdegree C decreased \poff to 0.037 
	(\figref{fig:results:campare_kinetics}b).
	%
	The effects of temperature and concentration were largely independent 
	of one another, providing precise control over the kinetic parameters. % cite jungmann paper
	%
	We also observed a substantial decrease in SNR with decreasing temperature, 
	and with increasing concentration. 
	%
	This was an expected side effect of increasing concentration, as more imager would increase \rb.
	%
	But the effect of temperature on SNR was surprising. 
	We hypothesize that this is due to the stabilization partial 
	binding between imager and docker strands at low temperatures.
	%
	Accounting for both the increase in counting accuracy and the decrease 
	in SNR, we identified imaging conditions of 20 nM and 13\textdegree C as optimal.

%-------------
\subsection{qPAINT Kinetic Regime}
%-------------
% qPAINT operates in a different kinetic regime that lbFCS
qPAINT relies on the accurate measure of the average dark time between blinking events~\citep{jungmann_2016}. 
	%
	As a result, this method is optimized for an entirely different kinetic regime, where blinking 
	events are short and infrequent (\figref{fig:results:qpaint_counting}a,b).
	% 
	This regime presents a challenge to \ours (\figref{fig:results:campare_kinetics}a top inset). 
	%
	If the only states ever observed are $\z{}=0$ or 1, there is insufficient 
	information to estimate count without prior knowledge.
	% 
	qPAINT faces the same limitation and relies on a calibration 
	of the blinking kinetics of a single binding site.
	% 
	\ours, being a Bayesian model can incorporate a similar calibration 
	as priors of the kinetic parameters.
	%
	With this addition, and the tightening of the priors on \re and \rb, the counting 
	ability of our model is restored (\figref{fig:results:qpaint_counting}c). 
	%
	\ours estimated the true count as the most likely in 133/300 (44\%) of traces, and 
	estimated within 1 of the true count in 269/300 (90\%) of all traces,
	comparable to the favorable kinetic regime (\figref{fig:results:sim_counting}a).
	%
	However, obtaining these priors requires additional experimental steps, and is often not trivial. 
	%
	Thus a tradeoff exists.
	%
	Without any calibration, \ours is able to accurately count in a subset of kinetic conditions ($\pon > \poff$).
	%
	But with an experimental calibration, the counting ability of \ours is expanded to a wider range of kinetic conditions. 

% qPAINT undercounts but blinx does not
Due to the stochastic nature of blinking, multiple emitters can 
be active at any given time-point, which becomes increasingly likely at higher counts.
	%
	This is not compatible with the qPAINT requirement of well separated, 
	single emitter blinking events.
	%
	As a result the measured average dark time is longer than the true average dark time
	and  qPAINT begins to underestimate molecular count 
	(especially noticeable above $\n=20$, see \figref{fig:results:qpaint_counting}c). 
	%
	The activity of multiple emitters at a single timepoint is explicitly modeled in both the 
	intensity and transition models previously derived (\eqref{eq:methods:p_c_given_z} and \eqref{eq:method:transition}). 
	%
	As a result, our model avoids underestimating, and provides an accurate measurement 
	of the molecular count up to $\n=30$.
	
%-------------
\subsection{Experimental Counting}
%-------------
% briefly describe experimental setup
To experimentally validate the counting performance of \ours, we used DNA-Origami, 
which provides nano-scale control over the number and location of emitters~\citep{rothemund_folding_2006}.
	% Why DNA origami
	DNA-Origamis were designed containing 1 and 4 DNA-PAINT docker strands, 
	spaced in a grid 20 \nanometer apart. 
	%
	This distance was specifically chosen, so that the true number of docker strands 
	could be visually confirmed through super-resolution post-processing.
	%
	Incorporation efficiency is roughly 80 percent for each docker site \cite{strauss_2018}, 
	so only a fraction of the origamis were expected to contain all 4 dockers. 
	% 
	Origamis were first imaged at 13\textdegree C, low laser power (1.5\%), and 20 nM imager concentration to 
	collect traces (4,000 frames at 200 ms exposure time) for counting with \ours (\figref{fig:results:experimental}a,b).
	%
	The system was then allowed to warm to 25\textdegree C, a buffer exchange performed, and new imager 
	added at 10 nM.
	%
	The origamis were imaged again at high laser power (40\%),
	and post-processed with Picasso \citep{schnitzbauer_2017} to obtain super-resolution ground truth (\figref{fig:results:experimental}c).
	%
	Only origamis that had a visual count matching the designed count (1 or 4) were selected for analysis with \ours.

Of the 131 traces with a known count of 1, \ours correctly counted 112 (85\%)
and the average the posterior distributions, is shown in \figref{fig:results:experimental}d.
	%
	Possible counts up to $\n=8$ were tested, but the likelihood for a count above $n=3$ was negligible for all traces.
	%
	For the traces with a known count of 4, 71/110 (65\%) were correctly identified as 4, 
	and 103/110 were identified as between 3 and 5. 
	%
	The average posterior distribution is shown in \figref{fig:results:experimental}e.
	Importantly, these traces were not preprocessed or filtered before undergoing \ours counting analysis.
