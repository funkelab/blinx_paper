\section{Results}

\subsection{Comparison to state of the art \lbfcs}

%
%Over time, the intensity contributions from independent emitters, combined for each timepoint, produce a fluctuating intensity trace.
	%
%	For a small number of emitters, \ie, $\truen = 2$, this trace contains series of easily distinguishable intensity steps corresponding to the number of emitters bound at each timepoint.
	%
  %However, for a larger number of emitters, \ie, $\truen = 20$ individual intensity steps can no longer be distinguished (see \figref{fig:results:comparison} A,B).

The current state of the art, \lbfcs, was benchmarked on simulated traces of a known number of emitters up to $\truen = 30$, with experimentally plausible parameters $(\mu=2000, \sigma=0.03, \pon=0.1, \poff=0.1)$.
	% 
While \lbfcs performed well for \truen \textless 12, performance quickly declined with increasing \truen, ultimately yielding invalid results for $\truen > 21$ (see \figref{fig:results:comparison}C).
	% lbfcs is limited by histogram fitting of mu
This drop in performance is a result of \textsc{lbFCS+}’s inability to identify the intensity of a single emitter ($\mu$ in \ours). \lbfcs is reliant on a histogram based method that works by identifying the $\y{}=1$ peak. However, for large \truen, observing this specific state becomes increasingly less likely. 

Holistic methods like \ours, on the other hand, allow inferring the intensity of a single emitter
jointly with the number of emitters active at every time-step, thus
circumventing the problem of having to observe a specific number of active emitters.
% blinx performance 
As a result \ours is able to count up to $\truen = 30$, effectively doubling the range of countable emitters (see \figref{fig:results:comparison} C).

% summary and transition 


\subsection{Informing experiment}

% Transition to effect of parameters on model accuracy
Next we investigated the robustness of \ours to the underlying blinking rates of individual emitters. 
% maximize signal to noise through microscope parameters
To determine the role of blinking rates on the accuracy of \ours, traces were simulated with \truen = 5 and \pon and \poff ranging from 0.01 to 0.2 (\figref{fig:results:regime}). 
% 3 regions in plot on>off on=off on<off
The results of these simulations can be sparated into 3 distinct regimes: \poff \textgreater \pon, \poff = \pon, and \poff \textless \pon. 
% on<off
In the regime where \poff \textgreater \pon, \ours displayed a very low accuracy. This poor performance is related to the hidden state \y{t} observed at each timepoint. 
%
As \poff increases relative to \pon the likelihood of observing \y{t}=0 increases while the likelihood for observing any other state decreases. As a result, the observed intensity traces in this regime only exhibit a small subset of the total possible states \y{}, leaving the model unable to determine \truen. 
%on=off
Interestingly, in the regime of \poff=\pon, an increase in \ours accuracy with increasing \pon is observed. Intuitively this makes sense, as both \pon and \poff increases the number of blinking events, providing more information in the signal. This result also suggests that if the observation time were increased, the accuracy of all regimes would increase as well. 
%on>off

Based on these results, we suspected that accuracy could be further
increased by shifting the experimental kinetics such that \pon $\gg$ \poff. Simulating
traces with \pon=0.17 and \poff= 0.01, yielded a substantially tighter distribution
around the correct count, compared to traces simulated with \pon =
\poff = 0.1 (\figref{fig:results:regime}B). While it is not possible to know the exact blinking
kinetics ahead of time, this finding can be used in experimental design: for maximizing the counting potential of the model, the blinking rates should be adjusted so that \pon is greater than \poff.