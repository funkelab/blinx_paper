\section{Results}

\subsection{Simulated Experiments}
% Simulated traces based on exerpimentally relevant parameters
Running blinx as a forward model, traces were generated for counts N=1 to 30
	(\figref{X} A) ,with experimentally relevent blinking and noise characteristics.
	%
	Camera parameters were determined from an empirical calibration of the
	sCMOS camera, and kinetic and emission parameters were determined through an
	initial experiment, further the kinetic parameters closely match those reported
	in literature.
	%
	Traces were generated for 4000 framnes, corresponding to an imaging time of 
	14 minutes.

% priors were placed on camera parameters, but all others were uniform
While fitting, tight gaussian priors were placed on the camera parameters,
	loose priors on the emission parameters, and flat/uniform priors on the
	kinetic parameters. 

% blinx can count!
blinx successfully fit all simulated traces with counts up to N=30 (\figref{}).
	For N=10 and below, blinx fit the correct count, with a significantly higher
	likelihood than all other possiblilites. Above N=10 the likelihood of other 
	possible counts increased, but the posterior distribution remained
	centered on the correct count.

% blinx outperforms lbFCS, counting up to 30 
This is a trippling in performance compared to lbFCS, which counted accurately up
	to N=7, but failed to estimate higher counts.
	% main limitation of lbFCS is estimation of the "step size" corresponding to r_e
	Upon further analysis, the main limitation of lbFCS is the estimation of the 
	intensity of a single emitter, corresponding to $r_e$ in blinx. If this value is 
	first estimated by blinx, then supplied to lbFCS, performance is singificantly 
	rescued (SI Figure)
% Longer traces lead to better fits
Next, we investigated the effect of trace length on counting accuracy (\figref{X} C).
	As expected accuracy increased and the variance of the posterior distribution decreased, 
	as the number of observations increased.
	% Also notice a underestimation of count with low trace length
	Iterestingly, for short trace lengths, blinx underestimated the true count. Underestimating 
	more serverly as the true count increased.
	% This is most likely due to the system only occupying a subset of all possible zs in this short time

% there is a lower bound to SNR, below a specific SNR, model maximally overcounts
Finally, the effect of SNR

\subsection{qPAINT Kinetic Regime}
% qPAINT operates in a different kinetic regime that lbFCS
qPAINT is based on an accurate measure of the average dark time between blinking events. 
	As a result, this method operates in an entirely different kinetic regime than lbFCS, where blinking 
	"on" events are short and infrequent (\figref{X}).
	% this presents challanges to blinx
	This regime presents a challange to blinx. If the only states ever observed are z=0 or 1 (off or on),
	there is not enough information in the system to estimate count without prior knowledge.
	% qPAINT relies on a calibration of kinetics
	qPAINT, faces the same limitation, and relies on a calibration of the blinking kinetics of a single binding site.
	% However, blinx is fully Bayesian and we can overcome these challanges by tightening priors
	blinx, as a fully Bayesian model can easily incorperate the same calibration as priors of the kinetic parameters
	and once again accurately count up to N=30 (\figref{X}).

% qPAINT undercounts but blinx does not
Due to the stochastic nature of blinking, multiple binding sites can blink at the same time, 
	which becomes increasingly more likely at higher counts.
	This is not compatible with the qPAINT assumption of well separated, single binding-site blinking events, 
	and as a result, qPAINT begins to underestimate molecular count, (especially noticable above N=20, \figref{X}). 
	% blinx avoids this problem
	The blinking of multiple binding sites at the same time point, 
	is enitrely accounted for in the transition distribution of blinx 
	and as a result, blinx accurately estimated the count even at higher N.
	

\subsection{Kinetic sweep}
% blinking kinetics can have a significant effect on blinx performance

% looking at kinetic performance without any priors

\subsection{Experimental Counting}
