\section{Results}

\begin{figure*}[ht]
  \includegraphics[width=\linewidth]{figures/comparison_lbfcs}
  \caption{TODO: add caption here}
  \label{fig:results:comparison}
\end{figure*}

\begin{figure}[ht]
  \includegraphics[width=\linewidth]{figures/kinetic_regime}
  \caption{TODO: add caption here}
  \label{fig:results:regime}
\end{figure}

\subsection{Comparison to state of the art \lbfcs}

\figref{fig:results:comparison} shows....

Blinx was first benchmarked on simulated traces of 4,000 frames with known N
values between 1 and 30, and with kinetic and intensity parameters closely
matching those seen in experiments (pon=poff=0.1) (example traces shown in
figure 1A and 1B). The performance of blinx was then compared to lbFCS2 (Figure
1B). Direct compassion to other models was not possible due to differences in
the data needed. Blinx and lbFCS2 both display good accuracy for N < 10
however, lbFCS quickly loses accuracy as N increases, while blinx retains a
high level of accuracy through counts up to N = 30. This drop off in
performance is caused by lbFCS2 reliance on a histogram-based method (??) to
estimate the intensity of a single fluorophore (mu). This becomes challenging
for larger numbers of binding sites, as this method relies on the detection of
distinct peaks in the histogram, and specifically the identification of the z =
1 peak, to accurately estimate the intensity of a single fluorophore. With
larger numbers of binding sites however, the z = 1 state becomes increasingly
less likely to be observed and harder to distinguish. Holistic methods like
blinx, on the other hand, allow inferring the intensity of a single fluorophore
jointly with the number of fluorophores bound at every timestep, thus
circumventing the problem of having to observe a specific number of bound
fluorophores.

\subsection{Informing experiment}

Next it was investigated how the accuracy of blinx depends on the blinking
kinetics of the individual labels. The label intensity and noise also have a
significant effect on model accuracy, but these are most heavily dependent on
microscope imaging conditions and can easily be adjusted during image
acquisition to maximize the signal to noise. Ideally the accuracy of blinx
would be invariant to different label kinetics, however some variance is
expected.

Blinx was tested on over 300 simulated traces with a known N of 5 and ranges of
pon and poff values from 0.01 to 0.2 (Figure 2A). It was found that while blinx
displayed a high accuracy in regions where pon>=poff, accuracy significantly
decreased as pon becomes much smaller than poff. This accuracy behavior
correlates well with label occupancy, or the average number of labels
illuminated at any time. In regions where poff >> pon, no matter the true N,
the occupancy approaches 0, and the only states observed are 0 and 1 labels
illuminated. This provides very little information to the model, leading to a
significant decrease in accuracy. Interestingly, this low occupancy region is
preferred for super-resolution microscopy with literature pon values ranging
from 0.001 to 0.05 and poff values ranging from 0.02 to 0.36. Another
interesting observation is the increase in accuracy along the pon = poff
diagonal as pon increases. As both pon and poff increase, transitions between
states become more common, providing more information for the model to fit.
This result also suggests that if the observation time were increased an
increase in accuracy for the low pon = poff region would increase.

Based on these results, it was hypothesized that accuracy could be further
increased by shifting the experimental kinetics such that pon>>poff. Simulating
traces with pon=0.17 and poff= 0.01, a significantly tighter distribution
around the correct count is observed, compared to traces simulated with pon =
poff = 0.1 (Figure 2B). While it is not always possible to know the blinking
kinetics ahead of time, this finding can be used in experimental design, so as
to engineer a system so that pon > poff and the counting potential of the model
is maximized.
