\section{Results}

\subsection{Simulated Experiments}

As a proof of concept, our model was first tested on simulated data (\figref{fig:results:sim_traces} A). 
	Traces were simualted by sampling a sequence of 4000 \states values from 
	the transition distribution \eqref{eq:method:transition}, 
	(effect of trace length investigated in SI),
	then randomly sampling from the intensity distribution \eqref{eq:method:intensity_distribution} given each timepoint in the \states sequence.
%
	Camera properties, $\mu=600$ ADU, $\sigma=49$ ADU, and $g=2.17$ ADU$/e^{-}$ were chosen based on the spec sheet provided by the camera manufacturer.
%
	The active emitter ($r_{E}=2.8$ $e^{-}/\text{ms}$) and background ($r_{BG}=16.6$ $e^{-}/\text{ms}$) photon emission rates, 
	as well as the transition probabilities ($\pon=0.046$ $/\Delta t$, $\poff=0.035$ $/\Delta t$)
	were estimated from a preliminary experiment (\figref{fig:si:preliminary_trace}).

To test the accuracy of our model, 10 traces were simulated for each true count $n=1-20$. 
	Possible counts of $n=1-25$ were then fit,
	and the posterior probability distribution ($\ndist$) is shown as a heatmap in \figref{a}. 
%
	For this experiment, $\mu$, $\sigma$, and $g$ were not fixed at the same values as the simulation and not fit through gradient descent, 
	as they could be experiemntally determined in future experiemnts.
%
	As shown in figure \figref{a}, our model is able to recover the correct \pon and \poff value with consistant accuracy even as the count
	increases up to 20 individual emitters.

Next, we compared the performance of our model, to that of the current state of the art lbFCS (source).
%

\subsection{informing experiment}
% One of the advantages of using DNA-PAINT to produce blinking (vs PALM or STORM) is the precise control over blinking kinetics

% did a parameter sweep of on and off rates to see their effect on model performance

\subsection{Experimental traces}

%Made DNA-origami with N=1 known binding site

% used blinx, with set N=1 to determine kinetics

% shifted to favorable regime by increasing concentration and decreasing temperature

% stacked N=1 traces to generate “half-simulated” traces up through N=20
% Figure: counting results for these stacked traces