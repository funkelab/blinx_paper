\section{Method}

Starting from the photo-physics of the system, we derive a probabilistic model
that accounts for both the observed intensity and the temporal fluctuation
kinetics.
  %
  Given an intensity trace, the output of our method is the posterior
  distribution over molecular counts, providing not only the most likely count,
  but also a comparison to all other possible counts. 

\subsection{Model}

We consider the case of multiple fluorescent emitters within a single
diffraction limited spot.
%
  As such, we can measure only their combined intensity contributions
  $\x{t}$ at time $t$.
  %
  We assume that each emitter blinks stochastically and independently of the
  others, yielding a fluctuating trace $\trace = (\x{1},\ldots,\x{T})$ of total
  intensity measurements over $T$ frames (\figref{fig:method:overview}a).
  %
  Our goal is to infer the number of emitters \n, given \trace.

Formally, we are modelling a posterior distribution of the
number of emitters \n given the intensity trace \trace, \ie:
  %
  \begin{equation*}
    p(\n | \trace) \propto p(n) p(\trace | \n)
    \text{.}
  \end{equation*}
  %
  % Can ignore p(n)
  Assuming a uniform prior on the number of emitters, $p(\n)$ can be ignored
  and it remains to model the likelihood $p(\trace|\n)$.
  %
  % choose to model as HMM
  %
  We model this likelihood as a \hmm where the hidden state, denoted by \z{t},
  represents the number of emitters that are active and emitting photons at
  time $t$ (see \figref{fig:method:overview}d).
  %
  The \hmm is conditioned on a set of parameters \parameters, including
    \parametersc camera parameters, \parameterse emission parameters, 
    and \parameterst transition parameters,
  %
  \begin{equation}
    p(\trace| \n, \parameters) =
      \prod_{t=1}^{T}
        p(\x{t} | \z{t}, \parameters)
        p(\z{t} | \z{t-1}, \parameters, \n)
    \text{,}
    \label{eq:method:HMM}
  \end{equation}
  %
  where we define $p(\z{1}|\z{0},\parameters,\n)=p(\z{}|\parameters,\n)$, \ie,
  the initial distribution of the hidden state $\z{}$ is the stationary
  distribution.

  Following this framework, the likelihood is the product of two probability
  distributions: an intensity model $p(\x{t}|\z{t},\parameters)$ and a
  transition model $p(\z{t}|\z{t-1},\parameters,\n)$.

\subsubsection{Intensity Model}

The distribution of observed intensity \x{t} given the number of
currently active emitters \z{t} can be directly derived from the photo-physics
of our microscope.
  %
  In the following section we omit the subscript $t$ for clarity.
  
Working backwards through the light path, the measured intensity value \x{}
is a function of the number of photons \photons detected
(\figref{fig:method:overview}c).
  %
  The detector itself contributes noise to the system, known as the readout
  noise, which we model as $p(\x{}|\photons,\parametersc)$.
  %
  The number of photons hitting the detector, in turn, depends on the number of
  active emitters \z{} and is distributed following a shot noise model
  $p(\photons|\z{},\parameterse)$.

The probability of observing an intensity \x{} given \z{} can then be obtained
by marginalizing over \photons, \ie:
  %
  \begin{equation}
    p(\x{}|\z{},\parameters) =
      \sum_\photons
        \underbrace{p(\x{}|\photons,\parametersc)}_{\text{readout}}
        \underbrace{p(\photons|\z{},\parameterse)}_{\text{shot}}
    \text{.}
    \label{eq:method:marginalized_intensity}
  \end{equation}

\paragraph{Readout Noise}

The readout noise from the detector is often assumed to be negligible (\ie when
using an EMCCD camera).
  %
  However, when using an sCMOS camera, its contribution is 
  significant ~\citep{huang_video-rate_2013}.
  %
  To accommodate both systems, here we include the readout noise in our model.
  %
  However, the readout noise contribution can be negated by setting $\camoffset=0$, 
  and $\camvar << 1$.
  %
  Given \photons photons, the readout intensity of the camera is normally
  distributed~\citep{huang_video-rate_2013}
  %
  \begin{equation}
    p(\x{}|\photons,\parametersc) = \mathcal{N}[\photons\camgain + \camoffset, \camvar](\x{})
    \text{,}
  \end{equation}
  %
  where $\parametersc = (\camgain, \camoffset, \camvar)$ are the camera's gain,
  offset, and variance, respectively.

  In the following, it will be beneficial to express the readout noise as a
  zero mean normal distribution that is independent of the number of photons
  detected. To that end, we transform our measurement \x{} into photon space
  by accounting for the camera gain and offset:
  %
  \begin{equation}
    \xp{} = \frac{\x{} - \camoffset}{\camgain}
    \;\;\;\;
    \x{} = \xp{}\camgain + \camoffset
    \text{.}
  \end{equation}
  %
  We can now express our readout noise distribution in terms of the transformed
  measurement \xp{}, which we further shift by $-\photons$ to obtain a
  zero-mean distribution:
  %
  \begin{align}
    p(\xp{}|\photons,\parametersc)
      &= \mathcal{N}[\photons, \frac{\camvar}{\camgain^2}](\xp{})  \\
      &= \mathcal{N}[0, \frac{\camvar}{\camgain^2}](\xp{} - \photons)
      \label{eq:method:transformed_readout_noise}
    \text{.}
  \end{align}

\paragraph{Shot Noise}

Due to shot noise, the number of photons detected over a time interval
$\deltat$ is Poisson distributed~\cite{mehta_poisson_2016}
(\figref{fig:method:overview}c).
  %
  Our model accounts for two sources of emitted photons: first, the \z{} active
  emitters produce photons at a rate $\z{}\re$, and second, out of plane
  emitters produce a relatively constant number of background photons \rb.
  %
  Combined, these sources produce the expected number of photons
  $\lambda=(\z{}\re + \rb)\deltat$ per frame.
  %
  \begin{equation}
    p(\photons|\z{},\parameterse)
      = \poisson[\underbrace{(\z{}\re + \rb)\deltat}_{\lambda}](\photons)
    \text{.}
    \label{eq:methods:p_c_given_z}
  \end{equation}
  %
  For large values of $\lambda$, the Poisson distribution approaches a normal
  distribution with both mean and variance of $\lambda$ and therefore we
  approximate our shot noise distribution as:
  %
  \begin{equation}
    p(\photons|\z{},\parameterse)
      \approx \mathcal{N}[\lambda, \lambda](\photons)
    \text{.}
    \label{eq:method:normal_approx_shot_noise}
  \end{equation}

\paragraph{Combined Intensity Model}

Taking together the readout noise transform
\eqref{eq:method:transformed_readout_noise} and shot noise approximation ~\eqref{eq:method:normal_approx_shot_noise}, the
intensity distribution \eqref{eq:method:marginalized_intensity} can now be
rewritten as
%
  \begin{equation}
    p(\xp{}|\z{},\parametersc, \parameterse)
      = \sum_\photons
        \mathcal{N}[0, \frac{\camvar}{\camgain^2}](\xp{} - \photons)
        \mathcal{N}[\lambda, \lambda](\photons)
    \text{.}
  \end{equation}
  %
  Similar to~\cite{huang_video-rate_2013}, we note that this expression
  resembles a convolution, \ie, the sum of two independent random variables
  (the photon count \photons and a zero-mean, photon-independent camera readout
  noise, ~\eqref{eq:method:transformed_readout_noise}). Since the two distributions are normal, we can rewrite
  the above as a single normal distribution with summed means and variances:
  %
  \begin{equation}
    p(\xp{}|\z{},\parametersc, \parameterse)
      = \mathcal{N}[\lambda, \lambda + \frac{\camvar}{\camgain^2}](\xp{})
    \text{.}
    \label{eq:method:combined_intensity}
  \end{equation}

\subsubsection{Transition Model}

To model the step-like temporal fluctuations in intensity observed in
the trace \trace, a distribution is needed that describes the change in the
number of active emitters \z{} over time.
  %
  To accomplish this, we assume that the process is Markovian and the number of active
  emitters \z{t} at time $t$ depends only on the number of active emitters at
  the previous time point \z{t-1}.
  %
  On the individual emitter level, we define \pon as the probability of an
  inactive emitter at time $t-1$ to become active at time $t$.
  Conversely, we define \poff as the probability of an active emitter at $t-1$
  to become inactive at time $t$ (\figref{fig:method:overview}b).

For the case of a single emitter (\ie, $\n=1$), the probability of observing
this emitter as active (\ie, $\z{t}=1$) if the emitter was
inactive in the previous frame, is \pon , or $(1-\poff)$ if 
it was previously active (and similarly for the probability of observing 
the emitter as inactive).
  %
  For multiple emitters, however, a transition from \z{t-1} to \z{t} can be
  caused by multiple events. Consider, for example the case of two emitters
  ($\n=2$). A transition from $\z{t-1}=1$ to $\z{t}=1$ can be caused either by
  no change occurring or by one emitter becoming active while the other emitter 
  deactivates.

In the general case ($\n\geq1$), we consider all possible events that can
lead to an observed change in \z{}. To that end, we marginalize over the
number $a$ of active emitters that deactivated from $t-1$ to $t$:
  %
  \begin{align}
    \label{eq:method:transition}
    p(z_{t} = z | z_{t-1}, \theta_{T}, n) &=\\
    \sum_{a = 0}^{z_{t-1}}
      {a \choose z_{t-1}}
      &\poff
      {z - z_{t-1} + a \choose \n - z_{t-1}}
      \pon
      \text{,} \notag
  \end{align}
  %
  We assume that all emitters share the same \pon and \poff, and that
  both probabilities remain constant over time.


\subsubsection{Inference}

The likelihood $p(\trace|\parameters,\n)$ of observing \trace, see
\eqref{eq:method:HMM}, depends on a total of seven parameters
\parameters and the total number of emitters \n.
  %
  However, we are interested in the posterior distribution $p(\trace|\n)$.
  %
  Accounting for the priors on \parameters in addition to the likelihood,
  this is known as the model evidence and is defined as:
  %
  \begin{equation*}
    p(\trace | n) = \int p(\trace | \theta, n) p(\theta | n) d\theta
    \text{.}
  \end{equation*}
  %
  Although no closed form solution exists for this integral, multiple approximations exist including 
  variational methods \cite{bronson_learning_2009} and the Laplace approximation~\citep{bishop_pattern_2006}.
  %
  We use the latter, which is the maximum likelihood solution corrected by the Occam factor:

  \begin{equation}
    \ln p(\trace | n) \approx
      \ln p(\trace | \parametersmap, \n) +
      \underbrace{O(\trace, \parametersmap)}_\text{Occam factor}
    \text{,}
    \label{eq:method:evidence_with_m}
  \end{equation}
  %
  where the Occam factor $O(\trace, \parametersmap)$ is defined as
  %
  \begin{align*}
    O(\trace, \parametersmap) &=
      \ln p(\parametersmap) +
      \frac{M}{2} \ln(2\pi) -
      \frac{1}{2} \ln |A|
    \\
    A &=
      -\nabla \nabla \ln p(\parametersmap | \trace)
    \text{.}
  \end{align*}
  %
  Here, $M$ is the dimensionality of \parameters, which is constant.
  %
  Because we are interested only in comparing likelihoods, this term can 
  safely be ignored, resulting in:
  %
  \begin{equation}
    \ln p(\trace | \n) \approx \ln p(\trace | \parametersmap, \n) + \ln p(\parametersmap) - \frac{1}{2} \ln |A|
    \text{.}
    \label{eq:method:evidence}
  \end{equation}
  %
  Therefore, to estimate the posterior distribution over \n, we first need to 
  find the maximum likelihood parameters $\parametersmap$ for each value of \n. 

  This model is fully differentiable, allowing us to estimate 
  the maximum likelihood parameters through gradient ascent:
  %
  \begin{align}
    \parameters_{k+1} = \parameters_{k} + \nu \frac{\partial p(\trace | \parameters, \n)}{\partial \parameters}
    \bigg\rvert_{\parameters_{k}}
    \notag
    \text{,}
  \end{align}
  %
  where $\nu$ is the step size, dynamically calculated by the Adam optimizer, and 
  the gradients of \parameters are taken with respect to the likelihood: $p(\trace | \n, \parameters)$.
  %
  The maximum likelihood parameters $\parametersmap$ are then obtained,
  \begin{equation}
    \parametersmap = \text{arg}\max_{\parameters} p(\trace | \parameters, \n)
    \text{,}
  \end{equation}
  %
  thus allowing us to estimate the posterior distribution $p(\trace|\n)$. 
  %
  Additionally, we can obtain an estimate of the maximum likelihood molecular count:
  %
  \begin{equation}
      \truen =
      \text{arg}\max_{n}
        \ln p(\trace | \n)
    \text{.}
    \label{eq:method:optimization}
  \end{equation}
