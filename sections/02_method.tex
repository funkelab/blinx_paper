\section{Method}

\subsection{Model}
Consider multiple fluorescent emitters, smaller than can be visually separated.
  %
  Each blinking stochastically and independently of the others, 
  yielding a fluctuating intensity signal over time \figref{fig:method:overview}a.
  %
  The goal of this method is to determine the posterior distribution over 
  possible number of emitters \ndist given the observed intensity trace \trace.
  %
  Because \ndist is discrete, models must be fit independently for each possible value of $n$.
  Beginning with Bayes law, the posterior distribution can be written as:
  % Bayes Law
  \begin{equation*}
    p(n | \trace) \propto p(n) p(\trace | n)
  \end{equation*}
  % Can ignore p(n)
  Assuming a uniform prior on the number of emitters, $p(n)$ can be dropped.
  % choose to model as HMM
  Because \trace is a series of sequential intensity measurements, 
    we choose to model this distribution as a Hidden Markov Model, parameterized by $\theta$,
     with hidden state $z$, which represents the number of emitters active and emitting photons at time $t$.
    
  \begin{equation}
    p(\trace| n) = \prod_{t=0}^{T} p(x_{t} | z_{t}, \theta) p(z_{t} | z_{t-1}, \parameters, n)
    \label{eq:method:HMM}
  \end{equation}

  Following this framework, the likelihood is the product of two probability distributions,
  an intensity distribution and a transition distribution. 

\subsubsection{Intensity Model}

To accurately model the observed intensity from a microscope system with an sCMOS camera, 
  the readout noise from the detector, and the shot noise of photon count must both be accounted for~\cite{huang_video-rate_2013} \figref{fig:method:overview}c.
  The probability of observing intensity \x{} can then be defined as the convolution of these two noise distributions.

  \begin{equation}
    p(\x{}|\z{},\parameters) = \sum_\photons p(\x{}|\photons,\parametersc)p(\photons|\z{},\parameterse)
  \end{equation}

\textbf{Camera readout noise}:
  %
  Given \photons photons, the readout intensity of the detector is normally distributed
  %
  \begin{equation}
    p(\x{}|\photons,\parametersc) = \mathcal{N}[\photons\camgain + \camoffset, \camvar](\x{})
  \end{equation}
  %
  Next, a change of variables is introduced, moving the distribution into photon space
  %
  \begin{equation}
    \xp{} = \frac{\x{} - \camoffset}{\camgain}
    \;\;\;\;
    \x{} = \xp{}\camgain + \camoffset
  \end{equation}
  %
  \begin{align}
    p(\xp{}|\photons,\parametersc)
      &= \mathcal{N}[\photons, \frac{\camvar}{\camgain^2}](\xp{}) \\
      &= \mathcal{N}[0, \frac{\camvar}{\camgain^2}](\xp{} - \photons)
  \end{align}
  %
  Shifting our distribution by the number of photons \photons, the readout noise of the detector can be expressed as a 0 mean 
  normal distribution, that is independent of the number of photons detected.

\textbf{Emission shot noise}:
  Due to shot noise, the number of photons detected over a time interval $\deltat$ is Poisson
  distributed~\cite{mehta_poisson_2016} \figref{fig:method:overview}c. In this system, there are two sources of emitted photons, a number of emitters in the active state $\z{}\re$,
  and other out of plane emitters producing a relatively constant amount of background photons $\rb$. Combined, 
  these sources produce the expected number of photons per exposure $\lambda$.
  \begin{align}
    p(\photons|\z{},\parameterse)
      &= \poisson[\underbrace{(\z{}\re + \rb)\deltat}_{\lambda}](\photons) \\
      &\approx \mathcal{N}[\lambda, \lambda](\photons)
  \end{align}
  % 
  For large values of $\lambda$ the poisson distribution approaches a normal distribution with both mean and variance of $\lambda$.

\textbf{Putting it back together:}
This convolution can now be seen as the sum of two random variables.
  \begin{align}
    p(\xp{}|\z{},\parameters)
      &= \sum_\photons
        \mathcal{N}[0, \frac{\camvar}{\camgain^2}](\xp{} - \photons)
        \mathcal{N}[\lambda, \lambda](\photons) \\
      &= \mathcal{N}[\lambda, \lambda + \frac{\camvar}{\camgain^2}](\xp{})
  \end{align}
  %
  In this case where both are normally distributed, the sum is also normally distributed with 
  a mean and variance that are the sum of the original random variables mean and variance respectively. 
  %

\subsubsection{Transition Model}
Next, to model the step-like temporal fluctuations in intensity, observed in the trace \trace, 
  a distribution is needed that describes the change in the number of active emitters \z{} over time. 
  %
  To do this, we assume that the process is Markovian and the number of active emitters \z{t} 
  at time $t$ is only dependent on the number of active emitters at the previous time point \z{t-1}.
  %
  On the individual emitter level, we define \pon as the probability of an emitter inactive at time $t-1$ becoming active at time $t$. 
  Conversely,  we define \poff as the probability of an emitter active at $t-1$ becoming inactive at time $t$ \figref{fig:method:overview}b.
  %
  Finally, assuming that all emitters share the same \pon and \poff, and that both remain constant over time, 
  the transition distribution can be written as:
  % 
  \begin{align}
    \label{eq:method:transition}
    p(z_{t} = z | z_{t-1}, n, \theta_{T}) &=\\
    \sum_{a = 0}^{z_{t-1}}
      {a \choose z_{t-1}}
      &\poff
      {z - z_{t-1} + a \choose \n - z_{t-1}}
      \pon
      \text{,} \notag\
  \end{align}
  %
  Note that this distribution depends on the total possible number of emitters $n$.

\subsubsection{Combined Model}

Substituting the intensity distribution \eqref{eq:method:intensity_distribution} 
and transition distribution \eqref{eq:method:transition} 
into \eqref{eq:method:HMM} and marginalizing over all possible \states
% 
\begin{align}
  \label{eq:method:likelihood}
  p(\trace|\n,\parameters) &=\\
    \sum_{\states}
      p(&\tilde{x}_{1}|z_{1}, \parameters)
      p(z_{1}|\n, \parameters)
      \prod_{t=2}^{T}
        p(\tilde{x}_{t}|z_{t}, \parameters)
        p(z_{t}|z_{t-1},\n, \parameters)
    \notag
  \text{.}
\end{align}
%
Where \parameters = $(\theta_{E}, \theta_{C}, \theta_{T})$

\subsubsection{Inference}

\eqref{eq:method:likelihood} describes the likelihood of observing \trace given seven fittable parameters.
  However, because the complexity of this expression is dependant on \n, the likelihoods for each \n are not directly comparable to one another.
  Rather, the model evidence, also known as the marginal likelihood, must be used to compare different values of \n and build the posterior. 

  \begin{equation*}
    p(\trace | n) = \int p(\trace | \theta, n) p(\theta | n) d\theta
  \end{equation*}

  This integral over all possible parameters $\theta$ can be approximated using the 
  Laplace approximation also called an Occam factor \cite{bishop_pattern_2006}
  %
  \begin{align*}
    \ln p(\trace | n) \approx \ln p(\trace | \parameters_{MAP}) + \ln p(\parameters_{MAP}) &\\
    + \frac{M}{2} \ln(2\pi) - \frac{1}{2} \ln |A|
    \label{eq:method:evidence_with_m}
  \end{align*}

  \begin{equation*}
    A = -\nabla \nabla \ln p(\parameters_{MAP} | \trace)
  \end{equation*}

  $M$ is the dimensionality of $\theta$, which in \ours is constant
  for all $n$ (only the size of the transition matrix changes).
  Therefore this term becomes a constant and can be dropped from 
  \eqref{eq:method:evidence_with_m}, resulting in:
  \begin{equation}
    \ln p(\trace | n) \approx \ln p(\trace | \parameters_{MAP}, n) + \ln p(\parameters_{MAP}) - \frac{1}{2} \ln |A|
    \label{eq:method:evidence}
  \end{equation}
  %
  Therefore, to estimate the posterior distribution over $n$ we first need to 
  find the maximum a posteriori (MAP) parameters for each value of $n$. 

  % blinx if fully differentiable
  \ours is fully differentiable, so $p(\trace | \parameters_{MAP}, n)$ can be efficiently 
  fit through gradient ascent.
  %
  Finally, the molecular count can be obtained:
  %
  \begin{equation}
      \estimatedn =
      \text{arg}\max_{n}(
      \max_\parameters
      p(\trace|\n,\parameters))
    \text{.}
    \label{eq:method:optimization}
  \end{equation}

\subsection{Experimental}
\subsubsection{DNA Origami}
DNA origami with 20 nm spaced docking strands, were designed with the Picasso-design module \cite{X}, 
  and a list of all single-strand DNA used as well as a detailed folding procedure can be found in \cite{X}.
  %
  A repetitive 11 nt docker sequence (CTCCTCCTCCT) was conjugated to the 5' end of select staple strands to form the grid.
  %
  Such repetitive sequences have been shown to increase \pon, while being short enough to prevent double binding events \cite{X}.

  DNA origamis samples were prepared for imaging following the procedure described in \cite{}. 
  %
  Briefly, a chamber (ibidi $\mu$-Slide 8-well glass bottom) was pacified with BSA, coated in streptavidin, and biotinylated origamis conjugated. 

\subsubsection{DNA-PAINT}
DNA-PAINT imaging was performed following the standard procedure detailed in \cite{}. 
  %
  Imaging solution contained buffer B (5 mM Tris-HCl, 10 mM MgCl$_2$ 1 mM EDTA, pH 8) and an oxygen scavenging system consisting of PCA, PCD, and Trolox.
  %
  For super-resolution experiments, 10nM of a 7 nt imager (GAGGAGG) was used, while for \ours counting analysis 
  20 nM of an 8 nt imager (GAGGAGGA) was used.
  %
  Both imagers were conjugated to at the 3' end to a Cy3B fluorophore, and a 200 ms exposure time was used for all experiments.
  %
  Images were post-processed in the Picasso-localize module to detect spots and correct drift over the time course. 
  %
  Super-resolution images were visualized in Picasso-render.

\subsubsection{Microscopy}
% Microscopy system
TIRF imaging was performed on a Zeiss Elyra 7 microscope, equipped with a pco.edge 4.2 sCMOS camera (pixel size 6.5 $\mu$m)
  %
  An $\alpha$ Plan-Apochromat 63x/1.46 TIRF oil immersion objective was used.
  % Chiller system
  For temperature control, an okolab H101-Cryo-BL system was used. Temperature readings were taken using a thermocouple from a water filled well neighboring 
  that of the sample in the 8 well chamber slide 
  % 
  It was found that this temperature control system was not compatible with super-resolution experiments, 
  due to vibrations caused by the pump, and transferred to the objective through the cooling jacket.
  % 
  Therefore for super-resolution experiments, the temperature control system was turned off, 
  and the sample temperature allowed to increase to room temp (25 C) before imaging.
  %
  % Further complications of DNA-PAINT at cool temperatures are discussed in the SI


