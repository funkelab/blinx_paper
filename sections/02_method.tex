\section{Method}

\todo{add summary of this section here}

\subsection{Model}
Consider multiple fluorescent emitters, who's combined intensity contributions are measured over time.
  % old version: smaller than can be visually separated,
  %\todo{the emitters could still be far from each other}
  %
  Each emitter blinks stochastically and independently of the others, 
  yielding a fluctuating sequence of total intensity measurements (see \figref{fig:method:overview}a).
  %
  We are interested in inferring the number of emitters \n, given this trace of intensity \trace.
  % Because the number of emitters is discrete, models must be fit independently for each possible value of $n$.
  % \todo{that is not generally true}\todo{\n was not introduced}\todo{also, use our macros :)}
  %
  % \todo{before we dive in, I am missing a simple sentence like ``We are interested in inferring the number \n of [...], given a trace of intensity \trace''}
  
  Beginning with Bayes law, the posterior distribution can be written as:
  % Bayes Law
  \begin{equation*}
    p(\n | \trace) \propto p(n) p(\trace | \n)
  \end{equation*}
  % Can ignore p(n)
  Assuming a uniform prior on the number of emitters, $p(\n)$ can be dropped.
  % choose to model as HMM
  We model this distribution as a \hmm, parameterized by \parameters.
    %
    The hidden state \z, represents the number of emitters active and emitting photons at time $t$
    % old version: with hidden state \z, which represents the number of emitters active and emitting photons at time $t$.
    % \todo{this sentence is too long, break it up and introduce the parts one by one}
    
  \begin{equation}
    p(\trace| n) = \prod_{t=0}^{T} p(x_{t} | z_{t}, \theta) p(z_{t} | z_{t-1}, \parameters, n)
    \label{eq:method:HMM}
  \end{equation}

  Following this framework, the likelihood is the product of two probability distributions,
  an intensity model and a transition model.
  %\todo{I'd say ``model'', that is also the title of the next sections}

\subsubsection{Intensity Model}
The distribution of observed intensity measurements can be derived directly from the photo-physics of our microscope system.
  %
  % Because the detector of our system is an sCMOS camera, both the readout noise from the camera, 
  % and the shot noise of photon counts must both be accounted for
  % ~\cite{huang_video-rate_2013} (\figref{fig:method:overview}c).
  %
  Working backwards through the light path, the measured intensity value \x{} is a function of the number of photons \photons detected.
  %
  The detector itself contributed noise to the system, known as the readout noise. 
  %
  The number of photons hitting the detector, is dependant on the number of emitters \z, and is distributed following the shot noise.
  %
  The readout noise from the detector is often assumed to be negligible (\ie when using an EMCCD camera).
  %
  However, because the detector of our system is an sCMOS camera, both noise distributions must be accounted for
  ~\cite{huang_video-rate_2013} (\figref{fig:method:overview}c).

  Marginalizing over \photons, the probability of observing intensity \x{} can then be defined as:
  %:\todo{\photons is not introduced}\todo{Generally, we are moving too fast here. Take your time explaining how we measure intensity and use Fig 1 as a guide. Then dive into the math.}

  \begin{equation}
    p(\x{}|\z{},\parameters) = \sum_\photons \underbrace{p(\x{}|\photons,\parametersc)}_{\text{readout}} 
    \underbrace{p(\photons|\z{},\parameterse)}_{\text{shot}}
    \label{eq:method:marginalized_intensity}
  \end{equation}

  For brevity, the subscript $t$ has been dropped from \eqref{eq:method:marginalized_intensity} through \eqref{eq:method:approx_nomral}.

\textbf{Readout noise (camera)}:
  %
  Given \photons photons, the readout intensity of the camera is normally distributed
  %
  \begin{equation}
    p(\x{}|\photons,\parametersc) = \mathcal{N}[\photons\camgain + \camoffset, \camvar](\x{})
  \end{equation}
  %
  Next, we introduce a change of variables, moving the distribution into photon space\todo{why do we do that?}
  %
  \begin{equation}
    \xp{} = \frac{\x{} - \camoffset}{\camgain}
    \;\;\;\;
    \x{} = \xp{}\camgain + \camoffset
  \end{equation}
  %
  \begin{align}
    p(\xp{}|\photons,\parametersc)
      &= \mathcal{N}[\photons, \frac{\camvar}{\camgain^2}](\xp{}) \\
      &= \mathcal{N}[0, \frac{\camvar}{\camgain^2}](\xp{} - \photons)
  \end{align}
  %
  Shifting our distribution by the number of photons \photons, the readout noise of the detector can be expressed as a zero mean 
  normal distribution, that is independent of the number of photons detected.\todo{Aha! That should be said before we do the change of variable.}


\textbf{Shot noise (emission)}:
  Due to shot noise, the number of photons detected over a time interval $\deltat$ is Poisson
  distributed~\cite{mehta_poisson_2016} (\figref{fig:method:overview}c). 
  %
  In this system, there are two sources of emitted photons, from emitters in the active state $\re$,
  and from out of plane emitters producing a relatively constant amount of background photons $\rb$. Combined, 
  these sources produce the expected number of photons per frame $\lambda$. \todo{$\lambda$ is the exposure? ;)}
  %
  \begin{align}
    p(\photons|\z{},\parameterse)
      &= \poisson[\underbrace{(\z{}\re + \rb)\deltat}_{\lambda}](\photons) \\
      &\approx \mathcal{N}[\lambda, \lambda](\photons)
  \end{align}
  % 
  For large values of $\lambda$ the poisson distribution approaches a normal distribution with both mean and variance of $\lambda$.\todo{say that this is an approximation}

\textbf{Putting it back together:}\todo{I wouldn't literally say that in a paper}
The intensity distribution can now be seen as a convolution of the readout and shot noise distributions.\todo{missing link to sum of independent random variables (and thus justification for our variable change)}
  %
  \begin{align}
    p(\xp{}|\z{},\parametersc, \parameterse)
      &= \sum_\photons
        \mathcal{N}[0, \frac{\camvar}{\camgain^2}](\xp{} - \photons)
        \mathcal{N}[\lambda, \lambda](\photons) \\
      &= \mathcal{N}[\lambda, \lambda + \frac{\camvar}{\camgain^2}](\xp{})
      \label{eq:method:approx_nomral}
  \end{align}
  %
  In this case where both are normally distributed, the convolution is also normally distributed with 
  a mean and variance that are the sum of the original random variables mean and variance respectively. 
  %

\subsubsection{Transition Model}
Next, to model the step-like temporal fluctuations in intensity, observed in the trace \trace, 
  a distribution is needed that describes the change in the number of active emitters \z{} over time. 
  %
  To do this, we assume that the process is Markovian and the number of active emitters \z{t} 
  at time $t$ is only dependent on the number of active emitters at the previous time point \z{t-1}.
  %
  On the individual emitter level, we define \pon as the probability of an emitter inactive at time $t-1$ becoming active at time $t$. 
  Conversely,  we define \poff as the probability of an emitter active at $t-1$ becoming inactive at time $t$ (\figref{fig:method:overview}b).
  %
  Finally, assuming that all emitters share the same \pon and \poff, and that both remain constant over time, 
  the transition distribution can be written as:
  % 
  \begin{align}
    \label{eq:method:transition}
    p(z_{t} = z | z_{t-1}, \theta_{T}, n) &=\\
    \sum_{a = 0}^{z_{t-1}}
      {a \choose z_{t-1}}
      &\poff
      {z - z_{t-1} + a \choose \n - z_{t-1}}
      \pon
      \text{,} \notag\
  \end{align}
  %
  Note that this distribution depends on the total possible number of emitters $n$.

\subsubsection{Inference}

\eqref{eq:method:HMM} describes the likelihood of observing \trace given seven fittable parameters.
  However, because the complexity of this expression is dependant on \n, the likelihoods for each \n are not directly comparable to one another.
  Rather, the model evidence, also known as the marginal likelihood, must be used to compare different values of \n and build the posterior. 

  \begin{equation*}
    p(\trace | n) = \int p(\trace | \theta, n) p(\theta | n) d\theta
  \end{equation*}

  This integral over all possible parameters \parameters = (\parametersc, \parameterse, \parameterst) can be approximated using the 
  Laplace approximation also called an Occam factor \cite{bishop_pattern_2006}
  %
  \begin{align*}
    \ln p(\trace | n) \approx \ln p(\trace | \parameters_{MAP}) + \ln p(\parameters_{MAP}) &\\
    + \frac{M}{2} \ln(2\pi) - \frac{1}{2} \ln |A|
    \label{eq:method:evidence_with_m}
  \end{align*}

  \begin{equation*}
    A = -\nabla \nabla \ln p(\parameters_{MAP} | \trace)
  \end{equation*}

  $M$ is the dimensionality of $\theta$, which in \ours is constant for all $n$ (only the size of the transition matrix changes).
  %
  Therefore this term becomes a constant and can be dropped from \eqref{eq:method:evidence_with_m}, resulting in:
  \begin{equation}
    \ln p(\trace | n) \approx \ln p(\trace | \parameters_{MAP}, n) + \ln p(\parameters_{MAP}) - \frac{1}{2} \ln |A|
    \label{eq:method:evidence}
  \end{equation}
  %
  Therefore, to estimate the posterior distribution over $n$ we first need to 
  find the maximum a posteriori (MAP) parameters for each value of $n$. 

  % blinx if fully differentiable
  \ours is fully differentiable, so $p(\trace | \parameters_{MAP}, n)$ can be efficiently 
  fit through gradient ascent.
  %
  Finally, the molecular count can be obtained:
  %
  \begin{equation}
      \estimatedn =
      \text{arg}\max_{n}(
      \max_\parameters
      p(\trace|\n,\parameters))
    \text{.}
    \label{eq:method:optimization}
  \end{equation}

\subsection{Experimental}
\subsubsection{DNA Origami}
DNA origami with 20 nm spaced docking strands, were designed with the Picasso-design module \cite{schnitzbauer_2017}, 
  and a list of all single-strand DNA used as well as a detailed folding procedure can be found in \cite{schnitzbauer_2017}.
  %
  A repetitive 11 nt docker sequence (CTCCTCCTCCT) was conjugated to the 5' end of select staple strands to form the grid.
  %
  Such repetitive sequences have been shown to increase \pon, while being short enough to prevent double binding events \cite{civitci_2020}.
  %
  DNA origamis samples were prepared for imaging following the procedure described in \cite{schnitzbauer_2017}. 
  %
  Briefly, a chamber (ibidi $\mu$-Slide 8-well glass bottom) was pacified with BSA, coated in streptavidin, and biotinylated origamis conjugated. 

\subsubsection{DNA-PAINT}
DNA-PAINT imaging was performed following the standard procedure detailed in \cite{schnitzbauer_2017}. 
  %
  Imaging solution contained buffer B (5 mM Tris-HCl, 10 mM MgCl$_2$ 1 mM EDTA, pH 8) and an oxygen scavenging system consisting of PCA, PCD, and Trolox.
  %
  For super-resolution experiments, 10nM of a 7 nt imager (GAGGAGG) was used, while for \ours counting analysis 
  20 nM of an 8 nt imager (GAGGAGGA) was used.
  %
  Both imagers were conjugated to at the 3' end to a Cy3B fluorophore, and a 200 ms exposure time was used for all experiments.
  %
  Images were post-processed in the Picasso-localize module to detect spots and correct drift over the time course. 
  %
  Super-resolution images were visualized in Picasso-render.

\subsubsection{Microscopy}
% Microscopy system
TIRF imaging was performed on a Zeiss Elyra 7 microscope, equipped with a pco.edge 4.2 sCMOS camera (pixel size 6.5 $\mu$m)
  %
  An $\alpha$ Plan-Apochromat 63x/1.46 TIRF oil immersion objective was used.
  % Chiller system
  For temperature control, an okolab H101-Cryo-BL system was used. Temperature readings were taken using a thermocouple from a water filled well neighboring 
  that of the sample in the 8 well chamber slide 
  % 
  It was found that this temperature control system was not compatible with super-resolution experiments, 
  due to vibrations caused by the pump, and transferred to the objective through the cooling jacket.
  % 
  Therefore for super-resolution experiments, the temperature control system was turned off, 
  and the sample temperature allowed to increase to room temp (25 C) before imaging.
  %
  % Further complications of DNA-PAINT at cool temperatures are discussed in the SI


