\section{Introduction}


% Challenges of fluorescence microscopy
Fluorescence microscopy is a foundational tool in the field of biology,
  providing useful information on the presence and localization 
  of labeled structures.
%
  However, it is often difficult to extract precise, quantitative information
  from fluorescent images, such as the molecular count.
  % Counting strucutres is important
  Determining molecular count can give important insights to the fucntion
  of protein complexes such as X, which ...

% Sometimes we can visually seperate structures,
% but often they are too close together
When the structures of interest are large, such as whole cells, organelles,
  or even X, they can be spatially separated, and precise counting becomes trivial.
  However many structures are smaller than the diffraction limited resoltution of 200 - 250 nm.
%
% While we cant separate spatially, we can temporally
While these small structures can no longer be directly spatially resolved, many super-resolution 
  imaging techniques utilize stochastic blinking to seperate these structures in time,
  and reconstuct a composite with a resolution well below the diffraction limit. 
  %
  PALM, and STORM, and other single-molecule localization microscopy techniques
  have been shown to achieve resolutions as high as 5-30 nm (seeing beyond the limit)
  % Even with this separation there are things that are smaller
  However, there are still smaller structures such as X, X, or X, that 
  cannot be seperated and quantified, even with these advanced techniques. 
  %
  Transition: new methods of molecular counting are needed

% Related Work
% --------------
% Overview of other counting methods
Perhaps the simplest method of molecular counting is correlating fluorescent
    intensity with number of labels. i.e. the brighter it is, the more there are. 
    This method works well for qualitative measures, but lacks the precision needed 
    to report exact counts. 
    % temporal information is useful for counting
    Similar to PALM and STORM, moleculr counting methods can make use of temporal information
    to achieve higher precision.
    % utilize random fluctuations
    Methods such as fluorescence correlation spectroscopy (FCS) and 
    balanced super-resolution optical fluctuation imaging (bSOFI)
    utilize random flucutations in fluorescent intensity over time to estimate the molecular count.
    % structured fluctuations in time are even more useful
    Using blinking fluorophores, a predictable structure can be introduced 
    to the intensity fluctuations, boosting the performance of counting algorithms.
    % limited by bleaching
    However, these models are limited by the complex behaviour of the fluorophores and their
    tendency to photobleach after a few blinking cycles.


% DNA-PAINT is extra useful because its emitters dont bleach either
In contrast, DNA-PAINT is a method of producing blinking fluorescence that is functioanlly
    immune to photobleaching due to the continuous replenishment of fluorophores from solution.
    % lbFCS
    Localization based FCS (lbFCS), combines the strucutred blinking of DNA-PAINT, with the 
    principles of FCS, fitting the autocorrelation function of intensity over time to produce a count,
    and is able to count up to 6 molecules.
    % qPAINT
    Quantitative DNA-PAINT (qPAINT) estiamtes the molecular count based on the frequency of blinking events. i.e.
    a blinking rate of 1 event per second is calibrated to 1 molecule, therefore an obervation of 10 events per second
    produces a count of 10 molecules.
    % Limitations
    Both of these methods are limited in that they fit summary statistics,
    rather than the data directly. 
    %
    Further, all of the methods mentioned provide a single estimate of molecular count.
    A well calibrated probabilistic estimate would provide 
    
% Our Model
% ----------
% fully Bayesian
We propose blinx, a fully Bayesian model to estimate the molecular 
  count directly from a fluctuating sequence of fluorescent intensity measurements over time.
  % Based on a fully differentiable markov chain
  Based on a fully differentiable Hidden Markov Model, blinx fits 7 parameters
  directly to the sequence of measurements, estimating a likelihood for each possible count. 
  % probabilistic
  These likelihoods can be directly compared, producing a posterior distribution of counts 
  given the observation sequence. 
  % more accurate than existing methods 

Here we first validate blinx on simualted experiments. Next