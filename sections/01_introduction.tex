\section{Introduction}

\begin{figure*}

  \begin{panel}{(b)}{\textwidth}
    \vspace{4mm}
    \tikzsetnextfilename{figure_1_model}
    \input{figures/model.tikz}%
  \end{panel}
\end{figure*}

\begin{figure*}
  \includegraphics[width=\linewidth]{figures/placeholders/figure_1_overview.png}
  \caption{A) Overview of the blinx method (scale bar: 1 $\mu$m) B) blinx is based on a Hidden Markov Model who's parameters are optimized to build the final posterior
  distribution}
  \label{fig:method:overview}
\end{figure*}


% molecular counting: what is it and why do we care?
%
Molecular counting aims to determine the absolute number of units 
in an complex, a quantity that is often essential to understanding
the underlying biology of a system.
  %
  The activation of T cells, for example, is sensitive to single ligands, and
  quantifying the molecular count of ligand-receptor interactions elucidates 
  system sensitivity and mechanisms of signalling pathways~\citep{irvine_2002}.
  %
  Furthermore, molecular counting enables identification of oligomeric states, 
  allowing differentiation of monomers, dimers, trimers, and higher order oligomers.
  %
  Several biological processes depend on oligomer quantity: TGF-$\beta$ signaling,
  for example, depends on the oligomeric state of Smad~\citep{inman_2002,
  moustakas_2002}.
  %
  Similarly, the oligomeric state of G-protein coupled receptors influences 
  GPCR signaling~\citep{felce_2018, breitwieser_2004}.

% fluorescence microscopy alone is not enough
%
Units of interest are often separated by only a few
nanometers and thus cannot be quantified through standard fluorescence
microscopy techniques.
  %
  While traditional fluorescence microscopy is diffraction limited to a resolution 
  of approximately 200 \nanometer, super-resolution 
  techniques~\citep{betzig_2006,rust_2006,rittweger_sted_2009} 
  are able to discern individual \smallobjects up to a resolution of 10 \nanometer 
  apart~\citep{valli_seeing_2021}.
  %
  Below that threshold, it is no longer possible to quantify the molecular count 
  through visual separation.
  %

% summary of what the problem is and how we approach it
%
We present an alternative method of molecular counting 
that does not rely on super-resolution microscopy.
%
  We attempt to estimate the molecular count directly through observation of 
  fluorescence dynamics, rather than by trying to optically separate signals.
  %
  We developed a Bayesian solution to estimate the number of
  fluorescently labelled \smallobjects directly from a diffraction limited
  spot.
  %
  Our solution is based on a probabilistic model that incorporates the
  photo-physics of blinking fluorescent emitters and that models their dynamics over
  time as a Markov chain.
  %
  Given a temporal record of the combined intensity of all \smallobjects within a
  diffraction limited spot (\figref{fig:method:overview}a), this model provides the 
  most likely number of emitters within the spot: the molecular count.

%
%
%Although other methods exist to quantify molecular count at this scale, a
%robust, accurate, \todo{needs citations} fluorescence based method would make
%quantification much more accessible, while maintaining the benefits of
%fluorescence imaging, such as spatial localization.

%%%%%%%%%%%%%%%%
% Related Work %
%%%%%%%%%%%%%%%%

% fluorescence intensity based counting
%
Perhaps the simplest method of molecular counting is to correlate the combined
fluorescent intensity of a spot with the number of \smallobjects, \ie, the more subunits 
located within a spot, the brighter the spot is expected to be ~\citep{schmied_2012,
tolar_2005}.
  %
  This method works well for qualitative measurements, but, due to the noise in
  intensities measured from any single fluorophore, this approach lacks the precision 
  required for accurate estimates of molecular counts. 

% Other methods work by separating things out in time
%
Molecular counting methods that incorporate temporal fluctuations of intensity
are more accurate than correlation approaches.
  %
  % FCS and bSOFI
  Methods such as fluorescence correlation spectroscopy
  (FCS)~\citep{otsuka_2023,wachsmuth_2015,politi_2018} and balanced
  super-resolution optical fluctuation imaging
  (bSOFI)~\citep{geissbuehler_2012} fit higher order statistics to fluctuations
  in fluorescent intensity over time to quantify the molecular concentration.
  % Bleaching based
  In other methods, temporal variations in intensity are induced rather than
  just observed; for example, counting distinct bleaching
  steps can provide estimates of the number of fluorescent
  emitters~\citep{ulbrich_2007,jain_2011,hummert_2021, garry_bayesian_2020}.

% Some of those use blinking fluorophores
%
Blinking fluorophores, such as those used in
PALM~\citep{sengupta_pcPALM_2011,lee_counting_2012} and
STORM~\citep{patel_blinking_2021}, can be used to facilitate solving the counting
problem~\citep{rollins_stochastic_2015,nino_2017}.
  %
  % Some of these methods are also based on HMMs, should we mention that here?
  The blinking behavior allows for modeling at the single-molecule level, compared 
  to the bulk measurements of FCS, while the repeated transitions in
  intensity provide more information than the irreversible switches that occur 
  in bleaching-based counting.
  %
  % Limitations
  A major limitation of these methods is the complex photophysical properties of 
  these blinking fluorophores, making it difficult to differentiate between possible 
  dark states and photo-bleaching. 

% DNA-PAINT is extra useful because its emitters don't bleach either
%
In contrast, DNA-PAINT~\citep{schnitzbauer_2017} achieves blinking through transient 
DNA-binding, effectively decoupling blinking from photophysics. This has 
the additional advantage of producing blinking fluorescence that is functionally immune to 
photo-bleaching, due to the continuous replenishment of fluorophores from 
solution~\citep{stehr_2021}.

% In contrast, DNA-PAINT~\citep{schnitzbauer_2017} provides
% blinking fluorescence that is functionally immune to photo-bleaching due to the
% continuous replenishment of fluorophores from solution~\citep{stehr_2021}.
  % qPAINT
  Quantitative DNA-PAINT (qPAINT)~\citep{jungmann_2016} estimates the molecular
  count based on the frequency of blinking events. For example, if a blinking rate of one
  event per second is calibrated to one molecule, an observation of
  ten events per second corresponds to a count of ten molecules. 
  qPAINT is a relative counting method, and as a result is entirely dependant on the quality of 
  the calibration, which can be a non-trivial endeavor.
  %
  % lbFCS
  Localization based FCS (\lbfcs)~\citep{stein_2019,stein_2021} combines the
  structured blinking of DNA-PAINT with the principles of FCS, fitting the
  autocorrelation function of intensity over time to produce a count. \lbfcs 
  can accurately count up to six molecules in a diffraction
  limited spot, without calibration.
  %
  % Limitations
  However, both of these methods are limited because they fit summary
  statistics, rather than the data directly.
  %
  In contrast, we fit the model to every measurement, 
  fully utilizing both the time and intensity information.
  
Furthermore, all prior methods provide a single estimate of molecular count, even though 
a probabilistic count may be preferable for downstream analysis.
  %
  Bayesian approaches can estimate a likelihood for each possible condition 
  and have been used previously to infer the number of FRET conformational 
  states \citep{hon_bayesian-estimated_2019, bronson_learning_2009}, 
  the number of photo-bleaching steps \citep{garry_bayesian_2020},
  the assignment of blinking events to specific fluorophores \citep{gabitto_bayesian_2021, fazel_bayesian_2019},
  and the molecular count from observed intensity \citep{nino_2017}.

% Our Model
% ----------
% fully Bayesian
We propose \ours, a Bayesian model to estimate the molecular count
directly from a trace of fluctuating diffraction limited fluorescent
intensity measurements.
  %
  % Based on a fully differentiable markov chain
  Based on a fully differentiable \hmm, \ours fits
  seven parameters that describe the photo-physics and kinetics of the system, 
  to the sequence of measurements and
  estimates a likelihood for each possible count.
  %
  % probabilistic
  These likelihoods can be compared directly, producing a posterior
  distribution of counts given the observed sequence.
  % more accurate than existing methods 
  We first run \ours as a forward model, to simulate
  different experimental conditions and to determine the effect of 
  each condition on counting ability.
  %
  We find that \ours provides a substantial improvement in calibration-free 
  (compared to \lbfcs) 
  and calibrated (compared to \qpaint), counting accuracy, and can 
  count across different kinetic regimes.
  %
  Finally, we prove the counting ability of \ours experimentally by validating 
  the estimated count with ground truth measured by super-resolution DNA-PAINT imaging.
