\section{Introduction}

\begin{figure*}

  \begin{panel}{(b)}{\textwidth}
    \vspace{4mm}
    \tikzsetnextfilename{figure_1_model}
    \input{figures/model.tikz}%
  \end{panel}
\end{figure*}

\begin{figure*}
  \includegraphics[width=\linewidth]{figures/placeholders/figure_1_overview.png}
  \caption{A) Overview of the blinx method (scale bar: 1 $\mu$m) B) blinx is based on a Hidden Markov Model who's parameters are optimized to build the final posterior
  distribution}
  \label{fig:method:overview}
\end{figure*}


% molecular counting: what is it and why do we care?
%
Molecular counting aims to determine the absolute number of \smallobjects
associated in an \object, a quantity that is often essential to understanding
the underlying biology of a system.
  %
  The activation of T-cells, for example, is sensitive to single ligands, and
  quantifying the molecular count of ligand-receptor interactions informs
  understanding of downstream signalling pathways and
  sensitivity~\citep{irvine_2002}.
  %
  Furthermore, molecular counting enables identifying oligomeric states by
  counting the subunits in a cluster, differentiating monomers, dimers,
  trimers, and higher order oligomers.
  %
  Several biological processes depend on this quantity: TGF-$\beta$ signaling,
  for example, depends on the oligomeric state of Smad~\citep{inman_2002,
  moustakas_2002}.
  %
  Similarly, the oligomeric state of G-protein coupled receptors influences
  downstream GPCR signaling~\citep{felce_2018, breitwieser_2004}.

% fluorescence microscopy alone is not enough
%
Often, as in the aforementioned examples, the \smallobjects of interest are only a few
nanometers apart and as such can not be quantified through standard fluorescence
microscopy techniques.
  %
  %Fluorescence microscopy can provide precise localization information of
  %molecular species, but its ability to determine quantitative information, such
  %as counts of nearby molecules, remains limited.
  %
  In contrast to larger structures (whole cells or organelles), which can
  easily be discerned in fluorescence microscopy, individual molecules in
  diffraction limited spots below a resolution of 200 \nanometer can no longer
  be separated.
  %
  Even super-resolution techniques~\citep{betzig_2006,rust_2006} that surpass
  the diffraction limit are not suitable to discern \smallobjects that are
  closer than 10 \nanometer apart~\citep{valli_seeing_2021}.
  %
  However, many \objects of interest are smaller still, or physically
  positioned closer together.

% summary of what the problem is and how we approach it
%
We present a Bayesian solution to estimate the number of fluorescently labelled
\smallobjects in a single diffraction limited spot.
  %
  Our solution is based on a probabilistic model that incorporates the
  photo-physics of blinking fluorescent emitters and models their dynamics over
  time as a Markov chain.
  %
  Given a trace of the combined intensity of all \smallobjects contained in a
  spot over time (\figref{fig:method:overview}a), this model can be used to obtain the 
  most likely number of emitters contained in that spot: the molecular count.

% Motivate fluorescence based methods (doesn't say a lot IMO -Jan)
%
%
%Although other methods exist to quantify molecular count at this scale, a
%robust, accurate, \todo{needs citations} fluorescence based method would make
%quantification much more accessible, while maintaining the benefits of
%fluorescence imaging, such as spatial localization.

%%%%%%%%%%%%%%%%
% Related Work %
%%%%%%%%%%%%%%%%

% fluorescence intensity based counting
%
Perhaps the simplest method of molecular counting is to correlate the combined
fluorescent intensity of a spot with the number of \smallobjects, \ie, the
brighter the spot is, the more \smallobjects are contained~\citep{schmied_2012,
tolar_2005}.
  %
  This method works well for qualitative measures, but, due to the noise in
  intensities measured from any single fluorophore, lacks the precision needed
  to report exact counts.

% Other methods work by separating things out in time
%
Molecular counting can be more reliable if temporal fluctuations of intensity
are considered.
  %
  % FCS and bSOFI
  Methods such as fluorescence correlation spectroscopy
  (FCS)~\citep{otsuka_2023,wachsmuth_2015,politi_2018} and balanced
  super-resolution optical fluctuation imaging
  (bSOFI)~\citep{geissbuehler_2012} fit higher order statistics to fluctuations
  in fluorescent intensity over time to estimate the molecular count.
  % Bleaching based
  In other methods, temporal variations in intensity are induced rather than
  just observed, \eg, by measuring the number of distinct bleaching
  steps that provide cues about the number of fluorescent
  emitters~\citep{ulbrich_2007,jain_2011,hummert_2021}.

% Some of those use blinking fluorophores
%
Blinking fluorophores, such as those used in
PALM~\citep{sengupta_pcPALM_2011,lee_counting_2012} and
STORM~\citep{patel_blinking_2021}, can be used to facilitate the counting
problem~\citep{rollins_stochastic_2015,nino_2017}.
  %
  % Some of these methods are also based on HMMs, should we mention that here?
  The known blinking behavior allows for more precise models than the random
  fluctuations measured in FCS and bSOFI, while the repeated transitions in
  intensity provide more information than the irreversible switch in bleaching
  based counting.
  %
  % Limitations
  A major limitation of these methods is the effect of photo-bleaching, as this
  limits the amount of time a single fluorophore can be observed, and makes it
  difficult to differentiate blinking from photo-bleaching.

% DNA-PAINT is extra useful because its emitters dont bleach either
%
In contrast, DNA-PAINT~\citep{schnitzbauer_2017} is a method of producing
blinking fluorescence that is functionally immune to photo-bleaching due to the
continuous replenishment of fluorophores from solution~\citep{stehr_2021}.
  %
  % lbFCS
  Localization based FCS (\lbfcs)~\citep{stein_2019,stein_2021} combines the
  structured blinking of DNA-PAINT with the principles of FCS, fitting the
  autocorrelation function of intensity over time to produce a count. As such,
  it is able to accurately count up to six molecules in a single diffraction
  limited spot.
  %
  % qPAINT
  Quantitative DNA-PAINT (qPAINT)~\citep{jungmann_2016} estimates the molecular
  count based on the frequency of blinking events. For example, if a blinking rate of one
  event per second is calibrated to one molecule, an observation of
  ten events per second corresponds to a count of ten molecules.
  % Limitations
  However, both of these methods are limited in that they fit summary
  statistics, rather than the data directly.
  %
  In contrast, our solution fits the model to each frame of the measurement 
  in sequence, fully utilizing both the temporal and intensity information available.
  
Further, all of the methods mentioned provide a single estimate of molecular count. 
  %
  A probabilistic estimate would identify cases where several different counts are likely,
  providing a more detailed quantitative understanding of the system. 

% Our Model
% ----------
% fully Bayesian
We propose \ours, a Bayesian model to estimate the molecular count
directly from a fluctuating sequence of fluorescent intensity measurements over
time.
  %
  % Based on a fully differentiable markov chain
  Based on a fully differentiable \hmm, \ours fits
  seven parameters, describing the photo-physics and kinetics of the system, 
  directly to the sequence of measurements, 
  estimating a likelihood for each possible count.
  %
  % probabilistic
  These likelihoods can be directly compared, producing a posterior
  distribution of counts given the observation sequence.
  % more accurate than existing methods 
  We first run \ours as a forward model, to simulate
  different experimental conditions and determine their effect on counting ability.
  %
  We demonstrate a substantial improvement in counting accuracy compared to both \lbfcs and \qpaint
  and an ability to count across different kinetic regimes.
  %
  Finally, we prove the counting ability of \ours experimentally by validating 
  the estimated count with ground truth measured by super-resolution DNA-PAINT imaging.

