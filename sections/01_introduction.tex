\section{Introduction}

Fluorescence microscopy is a foundational tool in the field of biology and
provides useful information on the abundance and localization of labeled
structures.
%
  However, determining the exact molecular count of these structures remains a
  challenge. Super-resolution techniques such as PALM \cite{betzig_imaging_2006}, 
  STORM \cite{rust_sub-diffraction-limit_2006}
  , and DNA-PAINT \cite{schnitzbauer_super-resolution_2017} have
  addressed this problem by achieving resolutions as high as 20nm, often
  spatially separating structures allowing for visual counting, but still many
  structures of interest exist even below this threshold prompting the need for
  new methods of molecular counting.

These super-resolution techniques are achieved by distributing the fluorescence
signal from each individual fluorophore over time.
%
  Only a sparse subset of fluorophores are illuminated in each frame, allowing
  for precise fitting of spatially distinct point spread functions, and once
  localization from all frames are combined, a sub-diffraction limit
  reconstruction of the image. Many molecular counting approaches similarly
  make use of imaging over time to extract additional information from the
  system. Some of the simplest systems count events that are designed to happen
  only once per label such as photobleaching steps \cite{Ulbrich_subunit_2007} or
  blinks from a 1 time photo-switching fluorophore
  \cite{gunzenhauser_quantitative_2012}. However, these events are often difficult to
  detect, and can easily be missed leading to severe undercounting.

Other methods make use of fluorophores that blink repeatedly to reduce the
dependence on individual events.
%
  By modeling the photoswitching kinetics of STORM dyes as a continuous time
  Markov process, \cite{patel_blinking_2021, rollins_stochastic_2015} one can extract the total count of molecules
  in a given area. However, these models are limited in accuracy by the complex
  photobleaching nature of the dyes used. Another temporal counting method is
  qPAINT \cite{jungmann_quantitative_2016} which utilizes DNA-PAINT to minimize photobleaching and
  correlates the frequency of repeated binding and unbinding events to the
  molecular count. This method shows an excellent accuracy for counts under 15,
  but relies heavily on previous knowledge of binding kinetics and is
  susceptible to simultaneous binding events.

Although not precise enough to determine molecular counts, the linear
relationship observed between fluorescence intensity and fluorophore abundance
\cite{schmied_fluorescence_2012}, provides additional information to improve the
accuracy of these methods.
%
  Many methods use correlation functions to combine both the temporal and
  intensity information to estimate the molecular count, such as balanced
  super-resolution optical fluctuation imaging (bSOFI) and fluorescence
  correlation spectroscopy (FCS) which has been used to estimate the copy
  number of specific proteins within the nuclear pore complex
  \cite{otsuka_quantitative_2023}. Similar techniques have also been applied to the
  consistent blinking behavior of DNA-PAINT in lbFCS+ \cite{stein_calibration-free_2021}, which has
  accurately counted up to ~8 molecules and does not rely on previous
  calibrations. These models work by fitting the entire observation series to a
  single equation, in the process greatly simplifying the information. We
  hypothesize that by fitting the entire observation series our model will be
  able to use this additional information to greatly improve the counting
  accuracy of our model.

Our approach, blinx, fits a series of hidden Markov models to the entire
observation sequence.
%
  By fitting kinetic and intensity parameters to the observations at each time
  point blinx will use all the information available to calculate a
  likelihood that the observations arose from an exact count. A further
  advantage of blinx is that it produces a likelihood distribution across
  all possible counts, that could be useful for further downstream
  applications.
