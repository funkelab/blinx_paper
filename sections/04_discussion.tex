\section{Discussion} \label{discussion}

Our analysis of different kinetic regimes highlights a trade-off 
between resolution and counting.
  %
  Relatively fast kinetics are required to generate temporal separation required for single 
  molecule localization microscopy.
  % 
  In contrast, a slower kinetic regime, where multiple emitters are active in a single frame, 
  is optimal for counting.
  %
  \ours can accurately estimate molecular count in both regimes, either calibration-free in the slow regime 
  or with a calibration in the fast regime.

% Limitations
While \ours is robust to a range of experimental conditions, 
we identified several limitations.
  %
  % overestimation of count
  First, \ours tends to overestimate the molecular
  count in time series with low SNR.
  %
  The model expects intensities to be observed in evenly spaced, normally
  distributed peaks (see histograms in \figref{fig:results:experimental}a,b).
  %
  With low SNR, noise can generate intensity between peaks, and adding additional 
  \z{}-states can sometimes yield higher likelihoods than correctly absorbing 
  this noise into the fitted distribution.
  %
  In the limit, an infinite number of states could perfectly
  explain any intensity time series.
  %
  % mention histogram comparison to identify over-counting?
  %
  This can be avoided by specifying a prior on \re, the photon
  emission rate of a single fluorophore, which effectively increases the cost
  of adding additional states, and therefore decreases the likelihood of these models.

A second limitation is that the intensity model does not capture all of the noise
in the system.
    %
    As seen in \figref{fig:results:experimental}a, there are occasional
    frames where the measured intensity is greater than expected and 
    fits the model distribution poorly.
    % 
    To compensate for these outliers, we incorporated a baseline outlier
    probability into the intensity model, \ie, a fixed uniform likelihood for
    observing any intensity within the recorded range of values.
    %
    To further reduce the influence of these observed outliers on counting accuracy, we
    excluded the highest 0.5\% of intensities values when calculating the
    likelihood (the specific value is adjustable as a hyperparameter).

% Assumptions
In this work, we used DNA-PAINT to test the performance of \ours 
due to its resistance to photobleaching and consistent, 
predictable blinking behavior.
  %
  However, \ours is not specific to DNA-PAINT and could be 
  applied to any experimental system that meets the following three assumptions.

  First, all sub-units are assumed to behave independently.
  % 
  In the intensity model, this means that observed intensity will scale
  linearly with the number of active emitters $z$.
  %
  In the transition model, this means that the blinking kinetics of one spot
  have no effect on the blinking kinetics of any other.
  %
  In practice, sub-units in close proximity might 
  interact and violate this assumption~\citep{helmerich_photoswitching_2022}. 
  %
  Other methods, like qPAINT, also make this assumption and have been shown to
  work well in experimental systems \citep{fischer_quantitative_2021,
  jayasinghe_true_2018}. 

  Second, all sub-units within a spot are assumed to have identical properties
  (\pon, \poff, \re).
  %
  Rather than model each sub-unit individually, we model the summed behavior
  of all sub-units combined. 
  %
  In other words, we assume that each sub-unit is
  interchangeable with any other within the same spot.
  %
  In some applications, where non-uniform behavior among subunits may occur, \eg,
  if one unit is less accessible to diffusion of
  imager than the others~\citep{civitci_2020}. 
  %
  In these cases, we expect a decrease in model performance.

Finally, all properties are assumed to remain constant over time. 
  %
  Experimentally, drift in emission properties (\re, \rb) can result from 
  an unstable focus and should be corrected before any processing with
  \ours.
  % 
  Changes in kinetic properties over time are also sometimes observed and could
  be caused by photo-bleaching, damage to emitters or temperature changes. 
  %
  Unlike the first two assumptions, this assumption is not fundamental to the
  structure of \ours and slight extensions of our model could account for these
  time dependant parameters.
  %
  In fact, we anticipate that intentional temporal changes in parameters could be used to gain more
  information from the system and further increase model performance.

% Outlook
In principle, these assumptions could be relaxed within the Bayesian framework of \ours,
providing opportunities to apply \ours to new systems and questions.
    % 
    For example, the simple modification of setting $\pon=0$ could support the
    counting of photobleaching events.
    %
    Additionally, by incorporating dynamic priors, \ours could also capture
    changing conditions over time, potentially increasing counting
    ability beyond what we demonstrate here.
    % extend to include PALM and STORM
    Further, with minor modifications to the transition distribution to account
    for photobleaching, this method could be extended to other stochastically
    blinking emitters, such as those used in PALM or STORM, opening the door
    for molecular counting in living samples.
    %
