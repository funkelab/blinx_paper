\section{Discussion} \label{discussion}

Our analysis of counting in different kinetic regimes highlights a trade-off 
that exists between counting ability and resolution.
  %
  For counting, a slower kinetic regime, where multiple emitters are active in a single frame, is optimal. 
  %
  For resolution, a fast kinetics regime is required to obtain the temporal separation required for SMLM.
  % 
  \ours is able to accurately estimate molecular count in both regimes, either calibration-free in the slow regime, 
  or with a calibration in the fast regime.

% Limitations
While \ours has been designed to be robust and applicable to a range of 
different experimental systems, there are a few limitations that should be considered.
  %
  % overestimation of count
  The primary limitation of \ours is the tendency to overestimate the molecular
  count in traces with low SNR.
  %
  This is caused by the distribution at the heart of the intensity model. 
  %
  The model expects intensities to be observed in evenly spaced, normally
  distributed peaks (see histograms in \figref{fig:results:experimental}a,b).
  %
  When intensity is observed between two of these peaks, it can sometimes be
  cheaper (in terms of likelihood) to add a new \z{}-state to the model, rather
  than account for that intensity with the given distribution.
  %
  In the extreme limit, an infinite number of states could be used to perfectly
  explain any intensity trace.
  %
  % mention histogram comparison to identify over-counting?
  %
  This limitation can be avoided by specifying a prior on \re, the photon
  emission rate of a single fluorophore, which effectively increases the cost
  of adding additional states.

Another limitation is that the intensity model does not capture all the noise
in the system.
    %
    As seen in \figref{fig:results:experimental}a, there are occasional
    frames where the measured intensity is greater than expected and 
    fits the model distribution poorly.
    % 
    To compensate for those outliers we incorporated a baseline outlier
    probability into the intensity model, \ie, a fixed uniform likelihood for
    observing any intensity within the recorded range of values.
    %
    To further reduce the influence of these observed outliers on counting accuracy, we
    excluded the highest 0.5\% of intensities values when calculating the
    likelihood (the specific value is adjustable as a hyperparameter).

% Assumptions
In this work, we used DNA-PAINT due to its resistance to photobleaching and consistent, 
predictable blinking behavior.
  %
  However, \ours is not specific to DNA-PAINT by design, and could be 
  applied to any experimental system, so long as a few essential assumptions are met:

  Firstly, we assume that all subunits behave independently.
  % 
  In the intensity model, this means that observed intensity will scale
  linearly with the number of active emitters $z$.
  %
  In the transition model, this means that the blinking kinetics of one spot
  have no effect on the blinking kinetics of any other.
  %
  In practice, subunits in extremely close proximity could
  potentially interact and violate this assumption~\citep{helmerich_photoswitching_2022}. 
  %
  Other methods like qPAINT make this same assumption and have been shown to
  work well in experimental systems \citep{fischer_quantitative_2021,
  jayasinghe_true_2018}. 

  Second, we assume that all subunits within a spot have identical properties
  (\pon, \poff, \re).
  %
  Rather than model each subunit individually, we model the summed behavior
  of all subunits combined. 
  %
  In other words, we assume that each subunit is
  interchangeable with any other within the same spot.
  %
  In some applications, where non-uniform behavior among subunits may occur, \eg,
  if one unit is less accessible to diffusion of
  imager than the others~\citep{civitci_2020}. 
  %
  In these cases, we would expect a decrease in model performance.

Finally, we also assume that all properties remain constant over time. 
  %
  Experimentally, drift in emission properties (\re, \rb) can result from 
  an unstable focus and should be corrected before any processing with
  \ours.
  % 
  Changes in kinetic properties over time are also sometimes observed and could
  be caused by photo-bleaching, damage to emitters or changes in temperature. 
  %
  Unlike the other two assumptions, this assumption is not fundamental to the
  structure of \ours and slight extensions of our model could account for these
  time dependant parameters.
  %
  In fact, we anticipate that intentional temporal changes in parameters could be used to gain more
  information from the system and further increase model performance.

% Outlook

In the future, through the Bayesian framework of \ours, these assumptions could be relaxed, 
opening opportunities for applications to new systems.
% The Bayesian framework of \ours opens this method up to many future 
% opportunities.
    % 
    For example, the simple modification of setting $\pon=0$ could support the
    counting of photobleaching events.
    %
    Additionally, by incorporating dynamic priors, our model could also capture
    changing conditions over time, potentially even increasing the counting
    ability.
    % extend to include PALM and STORM
    Further, with minor modifications to the transition distribution to account
    for photobleaching, this method could be extended to other stochastically
    blinking emitters, such as those used in PALM or STORM, opening the door
    for molecular counting in living samples.
    %
