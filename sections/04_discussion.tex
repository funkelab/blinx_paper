\section{Discussion} \label{discussion}
% Overview of highlights
Here, we presented \ours a probabilistic model, to estimate the absolute 
number of subunits in an area to small to be spatially separated. 
    %
    Our model was validated on both synthetic and experimental intensity traces,
    and benchmarked against the current state of the art methods \lbfcs \cite{stein_2021} and \qpaint \cite{jungmann_2016}.
    %
    Although experimentally DNA-PAINT was used, our model is not 
    limited to DNA-PAINT, and is compatible with any system that generates repeated fluorescent blinking.
    %
    We anticipate that \ours will be adaptable to any experimental molecular counting pipeline,
    and provide an increase in accuracy as well as probabilistic results, benefiting downstream analysis.

% Assumptions
There are a few assumptions that are essential to \ours.
    %
    Firstly, it is assumed that all subunits behave independently.
    % 
    In the intensity model, this means that observed intensity will scale linearly with the number of emitters active $z$.
    %
    In the transition model, this means that the blinking kinetics of one spot have no effect on the blinking kinetics of any other.
    %
    In practice, although not common, subunits in extremely close proximity could potentially 
    have steric hinderance effects on neighboring subunits, violating this assumption.
    %
    Other methods like qPAINT make this same assumption and have been shown to work well in experimental 
    systems \cite{fischer_quantitative_2021, jayasinghe_true_2018}. 
    
Secondly, it is assumed that all subunits within a spot have identical properties (\pon, \poff, \re).
    %
    Rather than model each subunit individually, \ours models the sum total behavior of all subunits combined, 
    and as a result each subunit is assumed to be interchangeable with any other within the same spot.
    %
    In applications where non-uniform behavior among subunits is expected, \eg due to 3D conformation, 
    1 subunit might be less accessible to diffusion of imager than the others \cite{civitci_2020}, a decrease in model performance is expected.
    %
    Finally, it is also assumed that all properties remain constant over time. 
    %
    Experimentally, drift in emission properties (\re, \rb) is commonly seen due to an unstable focus, and should be corrected before any processing with \ours.
    % 
    Change in kinetic properties over time is also sometimes observed and could be caused by damage to emitters, or changes in temperature. 
    %
    Unlike the other two, this assumption is not fundamental to the structure of \ours. 
    %
    Future versions of this model could account for these changing parameters.
    % 
    Intentional changes in parameters could even be used to gain more information from the system and further increase model performance. 
    

% Limitations
While, \ours has been designed to be applicable to as many different systems as possible, 
there are a few limitations that should be considered.
    % overestimation of count
    The primary limitation of \ours is the tendency to overestimate the molecular count, 
    especially in traces with low SNR. 
    %
    This is caused by the distribution at the heart of the intensity model. 
    %
    The model expects intensities to be observed in evenly spaced, normally distributed 
    peaks (see histograms in \figref{fig:results:experimental}a,b).
    %
    When an intensity is observed between two of these peaks, it can sometimes be cheaper, 
    in terms of likelihood, to add a new \z{}-state to the model,
    rather than account for that intensity with the given distribution.
    %
    In the extreme limit, an infinite number of states could be used to perfectly explain any intensity trace.
    %
    % mention histogram comparison to identify over-counting?
    %
    This limitation can be avoided by specifying a prior on \re, the photon emission rate of a single fluorophore. 
    %
    A prior on \re effectively increases the cost of adding additional states.

Another limitation is that the intensity model does not capture all the noise in the system.
    %
    As seen in \figref{fig:results:experimental}a, there are a significant amount of frames where the 
    measured intensity is greater than $10^4$ and out of our model's distribution.
    % 
    The frequency of this effect was dependant on the imaging temperature (data not shown).
    %
    As a result, we hypothesize that this effect is associated with the DNA-PAINT imaging system.
    % 
    To compensate for this behavior a baseline probability was incorporated into the intensity model 
    \ie there is a fixed minimum likelihood for observing any intensity.
    %
    Further, to reduce the influence of this effect on counting accuracy, the highest 0.5\% of 
    intensities values were excluded when calculating likelihood (the specific percent is adjustable as a hyperparameter).

% Outlook
Finally, the Bayesian framework of \ours, opens this method to many opportunities of future expansion.
    % 
    For example with the simple modification of setting $\pon=0$, this 
    model could support the counting of photobleaching events. 
    %
    Additionally, by incorporating dynamic priors, our model could 
    also capture changing conditions over time, potentially even increasing the counting ability.
    %extend to include PALM and STORM
    Further, with minor modifications to the transition distribution to account for photobleaching,
    this method could be extended to other stochastically blinking emitters, 
    such as those used in PALM or STORM, opening the door for molecular counting in living samples.
