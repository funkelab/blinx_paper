\section{Conclusions}
Here, we present blinx, a maximum likelihood estimate model to count the number of fluorescent emitters within a diffraction limited area.
%
Because blinx, estimates all parameters jointly, it allows for counting 
While blinx displays great accuracy on this simulated data, it relies on a few assumptions that could be relaxed in experimental data.
%
The first assumption is that all emitters in an area have a common \pon and \poff. 
%
While this assumption holds in controlled environments, local environment has been shown to significantly effect fluorescent emitter kinetics, and if the emitter kinetics vary significantly, this could have a detrimental effect on the counting ability of blinx. 
%
Another assumption is that the emitter intensities are log-normally distributed, and the mean intensity remains constant over time. While, both these assumptions are experimentally supported, the emitter intensity is highly variable with imaging conditions and can change significantly over the time-course if experimental parameters are not carefully managed. 
%


