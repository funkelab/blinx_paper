\section{Conclusions}

Here we present blinx, a maximum likelihood estimate model of the number of
labeled subunits within a spatially unresolvable area. One advantage of blinx
over other counting methods is that it returns a likelihood, allowing
comparison between different counts, which can supply additional information to
downstream models. Another advantage is blinx does not rely on histogram
methods, and therefore is able to count higher, up to 30. Finally, bcan be used
to inform experimental design, suggesting that designing labels where pon >>
poff would result in a further increase in counting ability.

\todo{add failure modes}
