\section{Intensity Model}

In the ideal case of a single fluorescent emitter, the number of photons given off over a period of time $\Delta t$ varies because of shot noise and can be described by the poisson distribution:

\begin{equation}
	p(c) = \text{Poisson}(\lambda) \;\;\;\;\;\;\; \lambda=r_{E}\Delta t
\end{equation}

Where $c$ is the photon count, $\lambda$ is the total expected number of photons emitted and $r_{E}$ is the emission rate in photons per second.
Because the sum of independent Poisson distributed random variables, is also Poisson distributed, the number of photons given off by multiple emitters can also be described as a single Poisson Distribution. Taking into account $z$ emitters and background fluorescence $r_{BG}$:

\begin{equation}
	p(c) = \text{Poisson}(\lambda) \;\;\;\;\;\;\; \lambda = (z r_{E} + r_{BG})\Delta t
\end{equation}

When measuring this system with an scMOS camera, the sensor introduces readout noise, thermal noise, and an amplification factor, which combined can be modeled as a gaussian for each pixel $i$ in the sensor, where $c_{i}$ is the number of photons detected by pixel $i$:

\begin{equation}
	p(x_{i} | c_{i}) = N( c_{i} g_{i} + \mu_{i}, \sigma_{i})
\end{equation}
\begin{equation*}
	= \frac{1}{\sqrt{2\pi \sigma_{i}} } \exp{ \left[
	- \frac{(x - c_{i}g_{i} - \mu_{i})^{2}}{2 \sigma_{i}} \right]}
\end{equation*}

Where $x$ is the sensor readout in ADU, $\mu$ and $\sigma$ are the mean and variance of the readout noise (after amplification), and $g$ is the gain or amplification factor.
%
When measured by the sensor, the emitted photons are spread across multiple pixels according to their point spread function (PSF). 
Summing this distribution to include all $N$ pixels covered by the PSF:

\begin{equation*}
	p(\sum_{i=0}^{N}x_{i} | \sum_{i=0}^{N}c_{i}) = N\left( \sum_{i=0}^{N}c_{i} g_{i} + \mu_{i}, \sum_{i=0}^{N}\sigma_{i} \right)
\end{equation*}

\begin{equation}
	p(x | c) = N( c g + \mu, \sigma)
\end{equation}

Accounting for the shot (Poisson) and the readout (Guassian) noise, the probability of observing an output value of $x$ can be described as the convolution of equations 2 and 5:

\begin{equation}
	p(x) = \sum_{c=0}^{\inf} p(x | c)p(c)
\end{equation}

\begin{equation*}
	p(x) = A \sum_{c=0}^{\inf} \frac{1}{c!} e^{-\lambda} \lambda^{c} 
	\frac{1}{\sqrt{2\pi \sigma} } \exp{ \left[
	- \frac{(x - cg - \mu)^{2}}{2 \sigma} \right]}
\end{equation*}


Following the approximation in \textit{Video-rate nanoscopy using scMOS camera-specific single-molecule localization algorithms} (Huang et. al., Supplementary note 4). This distribution can be approximated as:

\begin{equation}
	\frac{(x - \mu)}{g} + \frac{\sigma}{g^{2}} = Poisson \left(\lambda + \frac{\sigma}{g^{2}} \right)
\end{equation}

Defining $\tilde{x} = (x - \mu)/g + \sigma/g^{2}$ and $\tilde{\lambda} = \lambda +  \sigma/g^{2}$

\begin{equation}
	p(\tilde{x}) = \text{Poisson}\left(\tilde{\lambda} \right)
\end{equation}

At large $\lambda$s the Poisson distribution can be well approximated by a normal distribution with both a mean and variance of the poisson expected value $\lambda$.

\begin{equation}
	p(\tilde{x}) = N \left( \lambda, \tilde{\lambda} \right)
\end{equation}

Expanding terms:

\begin{equation}
	\frac{(x - \mu)}{g} + \frac{\sigma}{g^{2}}  = N\left((z r_{E} + r_{BG})\Delta t\;\;,\;\; (z r_{E} + r_{BG})\Delta t + \frac{\sigma}{g^{2}}\right)
\end{equation}

\bigskip
Where:

$x$ is the sum of measured pixel values in the spot (ADU)

$z$ is the number of emitters (iterated over in the model)

$r_{E}$ is the emitter photon rate (fit)

$r_{BG}$ is the background photon rate (fit)

$\Delta t$ is the frame exposure time (known constant)

$\sigma_{i}$ is the variance of the readout noise (measured camera property)

$\mu_{i}$ is the sum of the pixel readout noise means / offsets (measured camera property)

$g$ is the mean gain / amplification factor of pixels in the spot (measured camera property)







