\paragraph{Experimental Methods}


\paragraph{DNA Origami}
DNA origami with 20 nm spaced docking strands, were designed with the Picasso-design module \cite{schnitzbauer_2017}
and assembled following the procedure in \cite{schnitzbauer_2017}, with single-strand DNA oligos purchased from IDT.
  %
  A repetitive 11 nt docker sequence (CTCCTCCTCCT) was added to the 3' end of select staples as the docker sequence.
  %
  Such repetitive sequences have been shown to increase \pon~\citep{civitci_2020}.
  %
  DNA origamis samples were prepared for imaging in ibidi $\mu$-Slide 8-well glass bottom slides, 
  again following the procedure described in \cite{schnitzbauer_2017}. 
  
\paragraph{DNA-PAINT}
DNA-PAINT imaging was performed following the standard procedure detailed in \cite{schnitzbauer_2017}. 
  %
  Imaging solution contained buffer B (5 mM Tris-HCl, 10 mM MgCl$_2$ 1 mM EDTA, pH 8) and an oxygen scavenging system consisting of PCA, PCD, and Trolox.
  %
  Super-resolution images were acquired at an with a 7 nt imager (GAGGAGG-Cy3B, Biosyn) at a concentration of 10 nM,
  while counting analysis images were acquired with a 8nt imager (GAGGAGGA-Cy3B, Biosyn) at a concentration of 20 nM.
  %
  All images were post-processed in the Picasso-localize module to detect spots and correct drift over the time course. 

\paragraph{Microscopy}

% Microscopy system
TIRF imaging was performed on a Zeiss Elyra 7 microscope, equipped with 
a pco.edge 4.2 sCMOS camera (pixel size 6.5 $\mu$m), and an $\alpha$ 
Plan-Apochromat 63x/1.46 TIRF oil immersion objective.
  % Chiller system
  An okolab H101-Cryo-BL system provided temperature control. 
  % 
  It was found that this temperature control system was not compatible with super-resolution experiments, 
  due to vibrations caused by the pump, and transferred to the objective through the cooling jacket.
  % 
  Therefore for super-resolution experiments, the temperature control system was turned off, 
  and the sample temperature allowed to increase to room temp (25 C) before imaging.