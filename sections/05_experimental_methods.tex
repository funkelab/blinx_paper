\paragraph{Experimental Methods}


\paragraph{DNA Origami}
DNA origami with 20 nm spaced docking strands, were designed with the Picasso-design module \cite{schnitzbauer_2017}, 
  and a list of all single-strand DNA used as well as a detailed folding procedure can be found in \cite{schnitzbauer_2017}.
  %
  A repetitive 11 nt docker sequence (CTCCTCCTCCT) was conjugated to the 5' end of select staple strands to form the grid.
  %
  Such repetitive sequences have been shown to increase \pon, while being short enough to prevent double binding events \cite{civitci_2020}.
  %
  DNA origamis samples were prepared for imaging following the procedure described in \cite{schnitzbauer_2017}. 
  %
  Briefly, a chamber (ibidi $\mu$-Slide 8-well glass bottom) was pacified with BSA, coated in streptavidin, and biotinylated origamis conjugated. 

\paragraph{DNA-PAINT}
DNA-PAINT imaging was performed following the standard procedure detailed in \cite{schnitzbauer_2017}. 
  %
  Imaging solution contained buffer B (5 mM Tris-HCl, 10 mM MgCl$_2$ 1 mM EDTA, pH 8) and an oxygen scavenging system consisting of PCA, PCD, and Trolox.
  %
  For super-resolution experiments, 10nM of a 7 nt imager (GAGGAGG) was used, while for \ours counting analysis 
  20 nM of an 8 nt imager (GAGGAGGA) was used.
  %
  Both imagers were conjugated to at the 3' end to a Cy3B fluorophore, and a 200 ms exposure time was used for all experiments.
  %
  Images were post-processed in the Picasso-localize module to detect spots and correct drift over the time course. 
  %
  Super-resolution images were visualized in Picasso-render.

\paragraph{Microscopy}

% Microscopy system
TIRF imaging was performed on a Zeiss Elyra 7 microscope, equipped with a pco.edge 4.2 sCMOS camera (pixel size 6.5 $\mu$m)
  %
  An $\alpha$ Plan-Apochromat 63x/1.46 TIRF oil immersion objective was used.
  % Chiller system
  For temperature control, an okolab H101-Cryo-BL system was used. Temperature readings were taken using a thermocouple from a water filled well neighboring 
  that of the sample in the 8 well chamber slide 
  % 
  It was found that this temperature control system was not compatible with super-resolution experiments, 
  due to vibrations caused by the pump, and transferred to the objective through the cooling jacket.
  % 
  Therefore for super-resolution experiments, the temperature control system was turned off, 
  and the sample temperature allowed to increase to room temp (25 C) before imaging.
  %
  % Further complications of DNA-PAINT at cool temperatures are discussed in the SI