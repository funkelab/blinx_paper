\begin{abstract}

  We address the problem of inferring the number of independently blinking
  fluorescent light emitters, when only their combined intensity contributions
  can be observed at each timepoint.
  %
  This problem occurs regularly in light microscopy of \objects that are
  smaller than the diffraction limit, where one wishes to count the number of
  fluorescently labelled \smallobjects.
  %
  Our proposed solution directly models the photo-physics of the system, as
  well as the blinking kinetics of the fluorescent emitters as a fully
  differentiable \hmm.
  %
  Given a trace of intensity over time, our model jointly estimates the
  parameters of the intensity distribution per emitter, their blinking rates,
  as well as a posterior distribution of the total number of fluorescent
  emitters.
  %
  We show that our model is
  consistently more accurate and increases the range of countable subunits by a
  factor of two, compared to current state-of-the-art methods, which count based on
  autocorrelation and blinking frequency, 
  %
  Furthermore, we demonstrate that our model can be used to investigate the
  effect of blinking kinetics on counting ability, and therefore inform
  experimental conditions that will maximize counting accuracy.


  % We address the problem of inferring the number of independently blinking
  % \introduce{fluorescent light emitters}, when only their combined
  % \introduce{intensity} contributions can be observed at each
  % \introduce{timepoint}.
  % %
  % % why do we want to do that?
  % %
  % This problem occurs regularly in light microscopy of \introduce{complexes}
  % that are smaller than the diffraction limit, where one wishes to count the
  % number of \introduce{fluorescently} labelled \introduce{subunits}.
  % %
  % % how do we do that?
  % %   differentiable markov chain
  % %   blinx estimates number of active emitters for each timestep
  % %   holistic model
  % %
  % We estimate the number of subunits of a complex by modeling the light
  % emitted per complex over time as a Markov-chain.
  % %
  % Crucially, our model is differentiable, which allows us to jointly estimate
  % the parameters of the intensity distribution per emitter as well as their
  % blinking rates through maximum likelihood optimization.
  % %
  % % how good is it?
  % %
  % Compared to the current state of the art, which uses histogram-based
  % heuristics to infer distribution parameters, our holistic model is
  % consistently more accurate and extends the range of countable subunits by a
  % factor of two.
  % %
  % Furthermore, our model allows us to inform experimentalists about optimal
  % imaging parameters to increase accuracy.

\end{abstract}
