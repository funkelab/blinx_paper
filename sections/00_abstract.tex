\begin{abstract}

  We address the problem of inferring the number of independently blinking
  \introduce{fluorescent light emitters}, when only their combined
  \introduce{intensity} contributions can be observed at each
  \introduce{timepoint}.
  %
  % why do we want to do that?
  %
  This problem occurs regularly in light microscopy of \introduce{complexes}
  that are smaller than the diffraction limit, where one wishes to count the
  number of \introduce{fluorescently} labelled \introduce{subunits}.
  %
  % how do we do that?
  %   differentiable markov chain
  %   blinx estimates number of active emitters for each timestep
  %   holistic model
  %
  We estimate the number of subunits of a complex by modeling the light
  emitted per complex over time as a Markov-chain.
  %
  Crucially, our model is differentiable, which allows us to jointly estimate
  the parameters of the intensity distribution per emitter as well as their
  blinking rates through maximum likelihood optimization.
  %
  % how good is it?
  %
  Compared to the current state of the art, which uses histogram-based
  heuristics to infer distribution parameters, our holistic model is
  consistently more accurate and extends the range of countable subunits by a
  factor of two.
  %
  Furthermore, our model allows us to inform experimentalists about optimal
  imaging parameters to increase accuracy.

\end{abstract}
