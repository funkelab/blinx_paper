\begin{abstract}

Here, we aim to increase the quantitative precision of fluorescent microscopy 
  by accurately counting the number of single-molecules within a spatially limited area.
  %
  Focusing on a resolution limited spot, \ours models the temporal fluctuations of combined intensity over time
  to uncover the number of individual subunits within the spot. 
  %
  \ours directly models the photo-physics of the system, as well as the blinking kinetics of the fluctuations
  as a fully differentiable Markov Chain.
  %
  The result is a probabilistic model, that not only reports an accurate molecular count, but 
  also the likelihood of other possible counts, which can provide important information for downstream processing. 
  %
  \ours is then used to investigate the effect of blinking kinetics on counting ability, and to inform 
  experimental conditions that will maximize counting potential.
  %
  Compared to the current state of the art, which count based on autocorrelation and blinking frequency,
  \ours is consistently more accurate and increases the range of countable subunits by a factor of two.
  %



  % We address the problem of inferring the number of independently blinking
  % \introduce{fluorescent light emitters}, when only their combined
  % \introduce{intensity} contributions can be observed at each
  % \introduce{timepoint}.
  % %
  % % why do we want to do that?
  % %
  % This problem occurs regularly in light microscopy of \introduce{complexes}
  % that are smaller than the diffraction limit, where one wishes to count the
  % number of \introduce{fluorescently} labelled \introduce{subunits}.
  % %
  % % how do we do that?
  % %   differentiable markov chain
  % %   blinx estimates number of active emitters for each timestep
  % %   holistic model
  % %
  % We estimate the number of subunits of a complex by modeling the light
  % emitted per complex over time as a Markov-chain.
  % %
  % Crucially, our model is differentiable, which allows us to jointly estimate
  % the parameters of the intensity distribution per emitter as well as their
  % blinking rates through maximum likelihood optimization.
  % %
  % % how good is it?
  % %
  % Compared to the current state of the art, which uses histogram-based
  % heuristics to infer distribution parameters, our holistic model is
  % consistently more accurate and extends the range of countable subunits by a
  % factor of two.
  % %
  % Furthermore, our model allows us to inform experimentalists about optimal
  % imaging parameters to increase accuracy.

\end{abstract}
