\begin{figure*}

  \begin{panel}{(a)}{0.58\textwidth}
    \small%
    \setlength\plotwidth{42mm}%
    \setlength\plotheight{60mm}%
    \begin{tikzpicture}
  \begin{axis}[
    width=\plotwidth,
    height=\plotheight,
    scale only axis=true,
    name=sweep,
    xlabel=\pon,
    ylabel=\poff,
    enlarge x limits=false,
    enlarge y limits=false,
    scaled ticks=false,
    ticklabel style={/pgf/number format/.cd,fixed,precision=3},
    xtick distance=0.04,
    xtick align=outside,
    xtick pos=lower,
    ytick align=outside,
    ytick pos=lower,
    colorbar,
  ]

    \pgfplotsset{
      colormap={posteriorcolormap}{
        color(0.0)=(funkey_color_2!50!white)
        color(0.1)=(funkey_color_2)
        color(0.9)=(funkey_color_1)
        color(1.0)=(funkey_color_1!50!black)
      }
    }

    \addplot[
      matrix plot*,
      mesh/cols=10,
      point meta=explicit,
    ] table [
      col sep=comma,
      x=p_on,
      y=p_off,
      meta=error,
    ] {\kineticsheatmapcsv};

    % highlight qpaint and lbfcs points
    \node[circle,draw=funkey_color_9,thick] (qpaint) at (axis cs:0.01, 0.2) {};
    \node[circle,draw=funkey_color_9,thick] (lbfcs) at (axis cs:0.02, 0.02) {};

  \end{axis}

  \node[anchor=north] (qpaint_zs) at ($(sweep.north east)+(2.8,0)$) {
    \def\zhistogramcsv{figures/data/kinetics_grid/state_histogram.csv}%
    \def\zhistogramcol{qpaint}%
    \def\noylabels{}%
    \setlength\plotwidth{14mm}%
    \setlength\plotheight{16mm}%
    \tikz{\@ifundefined{noylabels}{}{%
  \pgfplotsset{yticklabel=\empty}%
}%
\@ifundefined{noxlabel}{%
  \pgfplotsset{xlabel=\z{}}%
}{%
  \pgfplotsset{xlabel=\empty}%
}%
\begin{axis}[
  width=\plotwidth,
  height=\plotheight,
  scale only axis=true,
  grid=major,
  grid style={dashed, very thin},
  enlarge x limits=0.1,
  enlarge y limits={value=0.2,upper},
  ymin=0,
  scaled ticks=false,
]

  \addplot+[
    ybar,
    bar width=1,
    mark=none,
    fill=ztracecolor,
    fill opacity=0.6,
    draw=ztracecolor,
  ] table [
    col sep=comma,
    y=\zhistogramcol,
    x=bin,
  ] {\zhistogramcsv};

\end{axis}
}%
  };
  \node[fit=(qpaint_zs),inner sep=0,rounded corners,draw=funkey_color_9,thick] (qpaint_box) {};

  \node[anchor=south] (lbfcs_zs) at (sweep.south-|qpaint_zs.south) {
    \def\zhistogramcsv{figures/data/kinetics_grid/state_histogram.csv}%
    \def\zhistogramcol{lbfcs}%
    \def\noylabels{}%
    \setlength\plotwidth{14mm}%
    \setlength\plotheight{16mm}%
    \tikz{\@ifundefined{noylabels}{}{%
  \pgfplotsset{yticklabel=\empty}%
}%
\@ifundefined{noxlabel}{%
  \pgfplotsset{xlabel=\z{}}%
}{%
  \pgfplotsset{xlabel=\empty}%
}%
\begin{axis}[
  width=\plotwidth,
  height=\plotheight,
  scale only axis=true,
  grid=major,
  grid style={dashed, very thin},
  enlarge x limits=0.1,
  enlarge y limits={value=0.2,upper},
  ymin=0,
  scaled ticks=false,
]

  \addplot+[
    ybar,
    bar width=1,
    mark=none,
    fill=ztracecolor,
    fill opacity=0.6,
    draw=ztracecolor,
  ] table [
    col sep=comma,
    y=\zhistogramcol,
    x=bin,
  ] {\zhistogramcsv};

\end{axis}
}%
  };
  \node[fit=(lbfcs_zs),inner sep=0,rounded corners,draw=funkey_color_9,thick] (lbfcs_box) {};

  \draw[funkey_color_3,thick] (qpaint) -- (qpaint-|qpaint_box.west);
  \draw[funkey_color_3,thick] (lbfcs) -- (lbfcs-|lbfcs_box.west);

\end{tikzpicture}

  \end{panel}
  \begin{panel}{(b)}{0.42\textwidth}
    \setlength\plotwidth{56mm}%
    \setlength\plotheight{60mm}%
    \pgfplotsset{/pgf/number format/.cd,fixed,precision=3}%
\begin{tikzpicture}
  \begin{axis}[
    width=\plotwidth,
    height=\plotheight,
    scale only axis=true,
    xlabel=\pon,
    ylabel=\poff,
    grid=major,
    grid style={dashed},
    scaled ticks=false,
    ticklabel style={font=\small},
  ]
    \pgfplotsset{
      colormap={posteriorcolormap}{
        color(0.0)=(funkey_color_7)
        color(0.5)=(funkey_color_2)
        color(1.0)=(funkey_color_1!50!red)
      }
    }

    \addplot[
      patch,
      patch type=rectangle,
      shader=interp,
      draw=black,
      fill opacity=0.4,
    ] table[
      x=pon_mean,
      y=poff_mean,
      point meta=\thisrow{temperature},
    ] {
      pon_mean    pon_var       poff_mean   poff_var      temperature concentration
      0.02813685  8.5348165e-06 0.07232066  9.0882335e-05 25          10
      0.02797574  1.8311839e-05 0.04775255  5.822931e-05  18          10
      0.05270538  6.0997234e-05 0.04954245  4.0005598e-05 18          20
      0.059536304 5.9824146e-05 0.08741606  0.00014375064 25          20

      0.05270538  6.0997234e-05 0.04954245  4.0005598e-05 18          20
      0.059536304 5.9824146e-05 0.08741606  0.00014375064 25          20
      0.07053028  4.2747815e-05 0.10869306  8.929498e-05  25          30
      0.072243474 0.00016735448 0.05488883  0.0012168097  18          30

      0.02797574  1.8311839e-05 0.04775255  5.822931e-05  18          10
      0.05270538  6.0997234e-05 0.04954245  4.0005598e-05 18          20
      0.045684416 6.31415e-05   0.03472516  8.919405e-05  13          20
      0.024452658 2.4062621e-05 0.036942866 4.378218e-05  13          10

      0.05270538  6.0997234e-05 0.04954245  4.0005598e-05 18          20
      0.072243474 0.00016735448 0.05488883  0.0012168097  18          30
      0.052961048 0.00010537513 0.03289487  6.0903745e-05 13          30
      0.045684416 6.31415e-05   0.03472516  8.919405e-05  13          20
    };

    \addplot[
      scatter,
      mark=*,
      only marks,
      point meta=\thisrow{temperature},
      scatter/@pre marker code/.append style={
        /tikz/mark size=\size,
      },
      visualization depends on={\thisrow{concentration} \as \concentration},
      visualization depends on={2+\thisrow{concentration}*0.1 \as \size},
    ] table [
      col sep=comma,
      x=pon_mean,
      y=poff_mean,
    ] {\kineticscsv};

    \coordinate (n_25_10) at (axis cs:0.02813685,0.07232066) {};
    \coordinate (n_18_10) at (axis cs:0.02797574,0.04775255) {};
    \coordinate (n_13_10) at (axis cs:0.024452658,0.036942866) {};
    \coordinate (n_25_20) at (axis cs:0.059536304,0.08741606) {};
    \coordinate (n_18_20) at (axis cs:0.05270538,0.04954245) {};
    \coordinate (n_13_20) at (axis cs:0.045684416,0.03472516) {};
    \coordinate (n_25_30) at (axis cs:0.07053028,0.10869306) {};
    \coordinate (n_18_30) at (axis cs:0.072243474,0.05488883) {};
    \coordinate (n_13_30) at (axis cs:0.052961048,0.03289487) {};

    \begin{pgfonlayer}{background}
      \draw (n_13_10) -- node[pos=0.5,above,sloped] {\tiny10nM} (n_18_10) -- (n_25_10);
      \draw (n_13_20) -- node[pos=0.5,above,sloped] {\tiny20nM} (n_18_20) -- (n_25_20);
      \draw (n_13_30) -- node[pos=0.5,above,sloped] {\tiny30nM} (n_18_30) -- (n_25_30);
      \draw (n_13_10) -- node[pos=0.5,above,sloped] {\tiny$T=13$} (n_13_20) -- (n_13_30);
      \draw (n_18_10) -- node[pos=0.5,above,sloped] {\tiny$T=18$} (n_18_20) -- (n_18_30);
      \draw (n_25_10) -- node[pos=0.5,above,sloped] {\tiny$T=25$} (n_25_20) -- (n_25_30);
    \end{pgfonlayer}
  \end{axis}

\end{tikzpicture}%

  \end{panel}

  \begin{panel}{(c)}{\textwidth}
    \small%
    \setlength\plotwidth{\textwidth}%
    \setlength\plotheight{36mm}%
    \def\tracecsv{figures/data/kinetics_grid/traces.csv}%
\def\traceintensitycol{qpaint_trace}%
\def\zcol{qpaint_fit}%
\def\histogramcsv{figures/data/kinetics_grid/intensity_histogram.csv}%
\def\histbincol{qpaint_bins}%
\def\histcountcol{qpaint_measured}%
\def\modelfitcol{qpaint_model}%
\def\posteriorcsv{figures/data/kinetics_grid/posteriors.csv}%
\def\posteriorcol{qpaint}%
\tikzsetnextfilename{sim_trace_qpaint}%
\begin{tikzpicture}
  \def\plotlabel{$\n = 10$}
  \@ifundefined{nolabels}{
  \pgfplotsset{xlabel=frames}
  \pgfplotsset{ylabel=intensity}
  \pgfplotsset{grid=major}
  \pgfplotsset{grid style={dashed}}
}{
  \pgfplotsset{xticklabel=\empty}
  \pgfplotsset{yticklabel=\empty}
  \pgfplotsset{ticks=none}
  \pgfplotsset{scaled ticks=false}
}
\@ifundefined{histogramcsv}{}{
  \setlength{\plotwidth}{0.73\plotwidth}
}
\begin{axis}[
  name=trace,
  width=\plotwidth,
  height=\plotheight,
  scale only axis=true,
  enlarge x limits=false,
  xtick distance=500,
  ticklabel style={font=\small},
  legend style={fill=black, nodes={scale=0.6, transform shape}},
]

  \addplot [
    color=tracecolor,
    mark=*,
    mark size=0.7pt,
    mark options={line width=0},
    fill opacity=0.8,
    draw opacity=0.0,
  ] table [
    col sep=comma,
    x=frames,
    y=\traceintensitycol
  ] {\tracecsv};
  \@ifundefined{nolabels}{\addlegendentry{intensity trace}}{}

  \@ifundefined{zcol}{}{
    \addplot [
      color=ztracecolor,
      thick
    ] table [
      col sep=comma,
      x=frames,
      y=\zcol
    ] {\tracecsv};
    \@ifundefined{nolabels}{\addlegendentry{inferred state}}{}
  }

  % remember min/max y-axis values for next plot
  \pgfplotsextra{
    \pgfmathparse{\pgfkeysvalueof{/pgfplots/ymin}}
    \global\let\ymin\pgfmathresult
    \pgfmathparse{\pgfkeysvalueof{/pgfplots/ymax}}
    \global\let\ymax\pgfmathresult
  }

\end{axis}
\@ifundefined{plotlabel}{}{
  \node[anchor=north west,rectangle,draw,fill=black] at (trace.north west) {\plotlabel};
}

\@ifundefined{histogramcsv}{}{
  \begin{axis}[
    at={($(trace.east) + (4mm,0)$)},
    name=histogram,
    anchor=west,
    width=0.25\plotwidth,
    height=\plotheight,
    scale only axis=true,
    ylabel=\empty,
    yticklabel=\empty,
    xtick distance=0.01,
    xlabel=probability,
    grid=major,
    grid style={dashed, very thin},
    enlarge x limits={value=0.4,upper},
    scaled ticks=false,
    ymin=\ymin,
    ymax=\ymax,
    ticklabel style={font=\small},
    legend style={fill=black, nodes={scale=0.6, transform shape}},
  ]

    \addplot+[
      xbar interval,
      mark=none,
      color=tracecolor,
      fill=tracecolor,
      fill opacity=0.6,
      draw=none,
    ] table [
      col sep=comma,
      y=\histbincol,
      x=\histcountcol,
    ] {\histogramcsv};
    \addlegendentry{intensity histogram}

    \addplot[
      color=intensitymodelcolor!80!black,
      thick
    ] table [
      col sep=comma,
      y=\histbincol,
      x=\modelfitcol
    ] {\histogramcsv};
    \addlegendentry{inferred model}

  \end{axis}
}

  \def\noylabels{}
  \def\noxlabel{}
  \setlength\plotwidth{0.1\plotwidth}
  \setlength\plotheight{\plotwidth}
  \pgfplotsset{every axis/.style={
    anchor=below south east,
    at={($(histogram.south east)-(2mm,-4mm)$)},
    xtick distance=10
  }}
  \def\eps{0.001}%  skip bars for values below eps
\@ifundefined{noylabels}{}{
  \pgfplotsset{yticklabel=\empty}
  \pgfplotsset{ylabel=\empty}
}
\@ifundefined{noxlabel}{
  \pgfplotsset{xlabel=$n$}
}{
  \pgfplotsset{xlabel=\empty}
}
\begin{axis}[
  width=\plotwidth,
  height=\plotheight,
  scale only axis=true,
  enlarge x limits={abs=1.5},
  enlarge y limits=0,
  ymin=0,
  ymax=1,
  scaled ticks=false,
  ticklabel style={font=\tiny},
  xtick distance=1,
  axis background/.style={fill=white},
]

  \addplot+[
    ybar,
    bar width=1,
    mark=none,
    fill=posteriorcolor,
    fill opacity=0.6,
    draw=posteriorcolor,
    y filter/.expression={
      y < \eps ? nan : y
    },
  ] table [
    col sep=comma,
    y=\posteriorcol,
    x=n,
  ] {\posteriorcsv};

  \ifdefined\posteriorcolextra
    \addplot+[
      ybar,
      bar width=1,
      mark=none,
      fill=posteriorcolor!60!black,
      fill opacity=0.6,
      draw=posteriorcolor,
      y filter/.expression={
        y < \eps ? nan : y
      },
    ] table [
      col sep=comma,
      y=\posteriorcolextra,
      x=n,
    ] {\posteriorcsv};
  \fi

\end{axis}

\end{tikzpicture}
%
    \vspace{-4mm}%
  \end{panel}

  \begin{panel}{(d)}{\textwidth}
    \small%
    \setlength\plotwidth{\textwidth}%
    \setlength\plotheight{36mm}%
    \def\tracecsv{figures/data/kinetics_grid/traces.csv}%
\def\traceintensitycol{lbfcs_trace}%
\def\zcol{lbfcs_fit}%
\def\histogramcsv{figures/data/kinetics_grid/intensity_histogram.csv}%
\def\histbincol{lbfcs_bins}%
\def\histcountcol{lbfcs_measured}%
\def\modelfitcol{lbfcs_model}%
\def\posteriorcsv{figures/data/kinetics_grid/posteriors.csv}%
\def\posteriorcol{lbfcs}%
\tikzsetnextfilename{sim_trace_lbfcs}%
\begin{tikzpicture}
  \def\plotlabel{$\n = 10$}
  \@ifundefined{nolabels}{
  \pgfplotsset{xlabel=frames}
  \pgfplotsset{ylabel=intensity}
  \pgfplotsset{grid=major}
  \pgfplotsset{grid style={dashed}}
}{
  \pgfplotsset{xticklabel=\empty}
  \pgfplotsset{yticklabel=\empty}
  \pgfplotsset{ticks=none}
  \pgfplotsset{scaled ticks=false}
}
\@ifundefined{histogramcsv}{}{
  \setlength{\plotwidth}{0.73\plotwidth}
}
\begin{axis}[
  name=trace,
  width=\plotwidth,
  height=\plotheight,
  scale only axis=true,
  enlarge x limits=false,
  xtick distance=500,
  ticklabel style={font=\small},
  legend style={fill=black, nodes={scale=0.6, transform shape}},
]

  \addplot [
    color=tracecolor,
    mark=*,
    mark size=0.7pt,
    mark options={line width=0},
    fill opacity=0.8,
    draw opacity=0.0,
  ] table [
    col sep=comma,
    x=frames,
    y=\traceintensitycol
  ] {\tracecsv};
  \@ifundefined{nolabels}{\addlegendentry{intensity trace}}{}

  \@ifundefined{zcol}{}{
    \addplot [
      color=ztracecolor,
      thick
    ] table [
      col sep=comma,
      x=frames,
      y=\zcol
    ] {\tracecsv};
    \@ifundefined{nolabels}{\addlegendentry{inferred state}}{}
  }

  % remember min/max y-axis values for next plot
  \pgfplotsextra{
    \pgfmathparse{\pgfkeysvalueof{/pgfplots/ymin}}
    \global\let\ymin\pgfmathresult
    \pgfmathparse{\pgfkeysvalueof{/pgfplots/ymax}}
    \global\let\ymax\pgfmathresult
  }

\end{axis}
\@ifundefined{plotlabel}{}{
  \node[anchor=north west,rectangle,draw,fill=black] at (trace.north west) {\plotlabel};
}

\@ifundefined{histogramcsv}{}{
  \begin{axis}[
    at={($(trace.east) + (4mm,0)$)},
    name=histogram,
    anchor=west,
    width=0.25\plotwidth,
    height=\plotheight,
    scale only axis=true,
    ylabel=\empty,
    yticklabel=\empty,
    xtick distance=0.01,
    xlabel=probability,
    grid=major,
    grid style={dashed, very thin},
    enlarge x limits={value=0.4,upper},
    scaled ticks=false,
    ymin=\ymin,
    ymax=\ymax,
    ticklabel style={font=\small},
    legend style={fill=black, nodes={scale=0.6, transform shape}},
  ]

    \addplot+[
      xbar interval,
      mark=none,
      color=tracecolor,
      fill=tracecolor,
      fill opacity=0.6,
      draw=none,
    ] table [
      col sep=comma,
      y=\histbincol,
      x=\histcountcol,
    ] {\histogramcsv};
    \addlegendentry{intensity histogram}

    \addplot[
      color=intensitymodelcolor!80!black,
      thick
    ] table [
      col sep=comma,
      y=\histbincol,
      x=\modelfitcol
    ] {\histogramcsv};
    \addlegendentry{inferred model}

  \end{axis}
}

  \def\noylabels{}
  \def\noxlabel{}
  \setlength\plotwidth{0.1\plotwidth}
  \setlength\plotheight{\plotwidth}
  \pgfplotsset{every axis/.style={
    anchor=below south east,
    at={($(histogram.south east)-(2mm,-4mm)$)},
    xtick distance=2
  }}
  \def\eps{0.001}%  skip bars for values below eps
\@ifundefined{noylabels}{}{
  \pgfplotsset{yticklabel=\empty}
  \pgfplotsset{ylabel=\empty}
}
\@ifundefined{noxlabel}{
  \pgfplotsset{xlabel=$n$}
}{
  \pgfplotsset{xlabel=\empty}
}
\begin{axis}[
  width=\plotwidth,
  height=\plotheight,
  scale only axis=true,
  enlarge x limits={abs=1.5},
  enlarge y limits=0,
  ymin=0,
  ymax=1,
  scaled ticks=false,
  ticklabel style={font=\tiny},
  xtick distance=1,
  axis background/.style={fill=white},
]

  \addplot+[
    ybar,
    bar width=1,
    mark=none,
    fill=posteriorcolor,
    fill opacity=0.6,
    draw=posteriorcolor,
    y filter/.expression={
      y < \eps ? nan : y
    },
  ] table [
    col sep=comma,
    y=\posteriorcol,
    x=n,
  ] {\posteriorcsv};

  \ifdefined\posteriorcolextra
    \addplot+[
      ybar,
      bar width=1,
      mark=none,
      fill=posteriorcolor!60!black,
      fill opacity=0.6,
      draw=posteriorcolor,
      y filter/.expression={
        y < \eps ? nan : y
      },
    ] table [
      col sep=comma,
      y=\posteriorcolextra,
      x=n,
    ] {\posteriorcsv};
  \fi

\end{axis}

\end{tikzpicture}

  \end{panel}

  \caption{%
    \panelref{a} The counting ability of \ours varies significantly with
      blinking kinetics of the emitters, showing increased model accuracy when
      $\pon > \poff$. Top inset: when $\pon < \poff$, the distribution of
      observed \z{} states is shifted towards 0. Bottom inset: for
      $\pon = \poff$, the distribution of \z{} states is centered at $1/2$ $n$.
    %
    \panelref{b} In DNA-PAINT \pon and \poff are governed by DNA binding kinetics and therefore are 
      highly sensitive to changes temperature and concentration, allowing experimental tuning of these parameters. 
    %
    \panelref{c} Trace with \pon=0.01, \poff=0.2, and true count of $n$=10, representative of (a) top inset.
    %
    \panelref{d} Trace with \pon=0.02, \poff=0.02, and true count of $n$=10, representative of (a) bottom inset.
  }
  \label{fig:results:campare_kinetics}
\end{figure*}
