\begin{figure*}

  \begin{panel}{(a)}{\textwidth}
    \def\histogramcsv{figures/data/sim_counting/intensity_histogram.csv}
    \def\tracecsv{figures/data/sim_counting/trace_and_fit_N10.csv}
    \def\traceintensitycol{trace}
    \def\zcol{fit}
    \def\histbincol{N10_bins}
    \def\histcountcol{N10_measured}
    \def\modelfitcol{N10_model}
    \tikzsetnextfilename{figure_2_trace_N10}
    \begin{tikzpicture}
  \begin{axis}[
    name=trace,
    width=0.8\textwidth,
    height=5cm,
    xlabel=frames,
    ylabel=intensity,
    enlarge x limits=false,
    xtick distance=500,
    grid=major,
    grid style={dashed},
    scaled ticks=false,
    ymin=\minintensity,
    ymax=\maxintensity,
    ticklabel style={font=\small},
  ]

    \addplot [
      color=tracecolor,
      mark=*,
      mark size=0.7pt,
      mark options={line width=0},
      fill opacity=0.8,
      draw opacity=0.2,
    ] table [
      col sep=comma,
      x=frames,
      y=\traceintensitycol
    ] {\tracecsv};

    \addplot [
      color=ztracecolor,
      thick
    ] table [
      col sep=comma,
      x=frames,
      y=\zcol
    ] {\tracecsv};

  \end{axis}

  \begin{axis}[
    at={($(trace.east) + (4mm,0)$)},
    anchor=west,
    width=0.3\textwidth,
    height=5cm,
    yticklabel=\empty,
    xtick distance=0.005,
    xlabel=probability,
    grid=major,
    grid style={dashed, very thin},
    enlarge x limits=0,
    scaled ticks=false,
    ymin=\minintensity,
    ymax=\maxintensity,
    ticklabel style={font=\small},
  ]

    \addplot+[
      xbar interval,
      mark=none,
      color=tracecolor,
      fill=tracecolor,
      fill opacity=0.6,
      draw=none,
    ] table [
      col sep=comma,
      y=\histbincol,
      x=\histcountcol,
    ] {\histogramcsv};

    \addplot[
      color=intensitymodelcolor!80!black,
      thick
    ] table [
      col sep=comma,
      y=\histbincol,
      x=\modelfitcol
    ] {\histogramcsv};

  \end{axis}
\end{tikzpicture}

%
  \end{panel}

  \begin{panel}{(b)}{\textwidth}
    \def\histogramcsv{figures/data/sim_counting/intensity_histogram.csv}
    \def\tracecsv{figures/data/sim_counting/trace_and_fit_N20.csv}
    \def\traceintensitycol{trace}
    \def\zcol{fit}
    \def\histbincol{N20_bins}
    \def\histcountcol{N20_measured}
    \def\modelfitcol{N20_model}
    \tikzsetnextfilename{figure_2_trace_N20}
    \begin{tikzpicture}
  \begin{axis}[
    name=trace,
    width=0.8\textwidth,
    height=5cm,
    xlabel=frames,
    ylabel=intensity,
    enlarge x limits=false,
    xtick distance=500,
    grid=major,
    grid style={dashed},
    scaled ticks=false,
    ymin=\minintensity,
    ymax=\maxintensity,
    ticklabel style={font=\small},
  ]

    \addplot [
      color=tracecolor,
      mark=*,
      mark size=0.7pt,
      mark options={line width=0},
      fill opacity=0.8,
      draw opacity=0.2,
    ] table [
      col sep=comma,
      x=frames,
      y=\traceintensitycol
    ] {\tracecsv};

    \addplot [
      color=ztracecolor,
      thick
    ] table [
      col sep=comma,
      x=frames,
      y=\zcol
    ] {\tracecsv};

  \end{axis}

  \begin{axis}[
    at={($(trace.east) + (4mm,0)$)},
    anchor=west,
    width=0.3\textwidth,
    height=5cm,
    yticklabel=\empty,
    xtick distance=0.005,
    xlabel=probability,
    grid=major,
    grid style={dashed, very thin},
    enlarge x limits=0,
    scaled ticks=false,
    ymin=\minintensity,
    ymax=\maxintensity,
    ticklabel style={font=\small},
  ]

    \addplot+[
      xbar interval,
      mark=none,
      color=tracecolor,
      fill=tracecolor,
      fill opacity=0.6,
      draw=none,
    ] table [
      col sep=comma,
      y=\histbincol,
      x=\histcountcol,
    ] {\histogramcsv};

    \addplot[
      color=intensitymodelcolor!80!black,
      thick
    ] table [
      col sep=comma,
      y=\histbincol,
      x=\modelfitcol
    ] {\histogramcsv};

  \end{axis}
\end{tikzpicture}


  \end{panel}

  \vspace{4mm}
  \begin{panel}{(c)}{0.4\textwidth}
    \def\posteriormatrixcsv{figures/data/sim_counting/heatmap.csv}
    \def\lbfcscsv{figures/data/sim_counting/heatmap_lbfcs.csv}
    \tikzsetnextfilename{figure_2_lbfcs_comparison}
    \begin{tikzpicture}
  \begin{axis}[
    name=trace,
    width=\textwidth,
    height=\textwidth,
    xlabel=Estimated \n,
    ylabel=True \n,
    enlarge x limits=false,
    enlarge y limits=false,
    grid=major,
    grid style={dashed},
    scaled ticks=false,
    ticklabel style={font=\small},
    colorbar,
  ]

    \pgfplotsset{
      colormap={posteriorcolormap}{
        color(0.0)=(white)
        color(0.2)=(funkey_color_2)
        color(1.0)=(funkey_color_2!50!black)
      }
    }

    \addplot[
      matrix plot*,
      mesh/cols=35,
      point meta=explicit,
    ] table [
      col sep=comma,
      x=n,
      y=true_n,
      meta=posterior,
    ] {\posteriormatrixcsv};

  \end{axis}

  \begin{axis}[
    name=trace,
    width=\textwidth,
    height=\textwidth,
    xmin=0.5,
    xmax=35.5,
    ymin=0.5,
    ymax=30.5,
    grid=major,
    grid style={dashed},
    scaled ticks=false,
    domain=0:35,
    ticklabel style={font=\small},
  ]

    \addplot[
      mark=*,
      only marks,
      mark size=1.4pt,
      mark options={draw=white,fill=funkey_color_1,draw opacity=0.6,fill opacity=0.7},
    ] table [
      col sep=comma,
      x=lbfcs_count,
      y=n,
    ] {\lbfcscsv};

    \addplot[no marks,thick,gray] {x};

  \end{axis}

\end{tikzpicture}

  \end{panel}
  \hspace{1cm}
  \begin{panel}{(d)}{0.2\textwidth}
    \def\posteriorcsv{figures/data/sim_counting/posteriors.csv}
    \def\posteriorcol{posterior_10}
    \def\posteriorcolextra{posterior_20}
    \tikzsetnextfilename{figure_2_posterior_n10_n20}
    \begin{tikzpicture}

  \begin{axis}[
    width=0.3\textwidth,
    height=0.3\textwidth,
    xlabel=\n,
    xlabel=$p(\n|\trace)$,
    grid=major,
    grid style={dashed, very thin},
    enlarge x limits=1.4,
    enlarge y limits=0,
    ymin=0,
    ymax=1,
    scaled ticks=false,
    ticklabel style={font=\small},
  ]

    \addplot+[
      ybar,
      bar width=1,
      mark=none,
      fill=posteriorcolor,
      fill opacity=0.6,
      draw=posteriorcolor,
      y filter/.expression={
        y < 0.000001 ? nan : y
      },
    ] table [
      col sep=comma,
      y=\posteriorcol,
      x=n,
    ] {\posteriorcsv};

  \end{axis}

\end{tikzpicture}

  \end{panel}
  \begin{panel}{(e)}{0.4\textwidth}
    \def\tracelengthcsv{figures/data/trace_length/trace_length_results.csv}
    \def\mapcol{max_likelihoods_20}
    \def\varcol{variance_20}
    \tikzsetnextfilename{figure_2_trace_length}
    \def\intervalplot#1#2{%
  \addplot[%
    name path=lower,%
    draw=none,%
    fill=none,%
  ] table [%
    col sep=comma,%
    x=length,%
    y expr=\thisrow{max_likelihoods_#1} - \thisrow{variance_#1}%
  ] {\tracelengthcsv};%
  \addplot[%
    name path=upper,%
    draw=none,%
    fill=none,%
  ] table [%
    col sep=comma,%
    x=length,%
    y expr=\thisrow{max_likelihoods_#1} + \thisrow{variance_#1}%
  ] {\tracelengthcsv};%
  \addplot [%
    color=#2,%
    opacity=0.6,%
  ] fill between[of=lower and upper];%
  \addplot[%
    color=#2,%
  ] table [%
    col sep=comma,%
    x=length,%
    y=max_likelihoods_#1,%
  ] {\tracelengthcsv};%
}%
\begin{tikzpicture}
  \begin{axis}[
    width=\textwidth,
    height=\textwidth,
    xlabel={length [frames]},
    ylabel=estimated count,
    enlarge x limits=false,
    enlarge y limits=false,
    xtick distance=4000,
    ytick distance=5,
    ymin=0,
    grid=major,
    grid style={dashed},
    scaled ticks=false,
    ticklabel style={font=\small},
    legend style={nodes={scale=0.6, transform shape}},
  ]
  \intervalplot{20}{funkey_color_1}
  \addlegendentry{$\n = 20$}%
  \intervalplot{15}{funkey_color_2}
  \addlegendentry{$\n = 15$}%
  \intervalplot{10}{funkey_color_3}
  \addlegendentry{$\n = 10$}%
  \intervalplot{5}{funkey_color_4}
  \addlegendentry{$\n = 5$}%
  \end{axis}
\end{tikzpicture}%

  \end{panel}

\end{figure*}


\begin{figure*}
  \includegraphics[width=\linewidth]{figures/placeholders/figure_2_simulated_counting.png}
  \caption{A) Trace simulated from 10 emitters, and the posterior distribution estimated by blinx B) simulated from 20 emitters
  C) The blinx posterior is able to accurately estimate significantly higher molecular counts than the current state of the art, lbFCS
  D) As trace length increases, the variance of the blinx posterior decreases E) As the signal to noise ratio increases the variance of 
  the blinx posterior decreases. }
  \label{fig:method:overview}
\end{figure*}
