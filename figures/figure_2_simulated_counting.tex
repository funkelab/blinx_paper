\begin{figure*}

  \begin{panel}{(a)}{\textwidth}
    \hspace{-2mm}%
    \setlength\plotwidth{\textwidth}%
    \setlength\plotheight{36mm}%
    \def\histogramcsv{figures/data/sim_counting/intensity_histogram.csv}%
\def\tracecsv{figures/data/sim_counting/trace_and_fit_N10.csv}%
\def\traceintensitycol{trace}%
\def\zcol{fit}%
\def\histbincol{N10_bins}%
\def\histcountcol{N10_measured}%
\def\modelfitcol{N10_model}%
\def\posteriorcsv{figures/data/sim_counting/posteriors.csv}%
\def\posteriorcol{posterior_10}%
\tikzsetnextfilename{sim_trace_n10}%
\begin{tikzpicture}
  \@ifundefined{nolabels}{
  \pgfplotsset{xlabel=frames}
  \pgfplotsset{ylabel=intensity}
  \pgfplotsset{grid=major}
  \pgfplotsset{grid style={dashed}}
}{
  \pgfplotsset{xticklabel=\empty}
  \pgfplotsset{yticklabel=\empty}
  \pgfplotsset{ticks=none}
  \pgfplotsset{scaled ticks=false}
}
\@ifundefined{histogramcsv}{}{
  \setlength{\plotwidth}{0.73\plotwidth}
}
\begin{axis}[
  name=trace,
  width=\plotwidth,
  height=\plotheight,
  scale only axis=true,
  enlarge x limits=false,
  xtick distance=500,
  ticklabel style={font=\small},
  legend style={fill=black, nodes={scale=0.6, transform shape}},
]

  \addplot [
    color=tracecolor,
    mark=*,
    mark size=0.7pt,
    mark options={line width=0},
    fill opacity=0.8,
    draw opacity=0.0,
  ] table [
    col sep=comma,
    x=frames,
    y=\traceintensitycol
  ] {\tracecsv};
  \@ifundefined{nolabels}{\addlegendentry{intensity trace}}{}

  \@ifundefined{zcol}{}{
    \addplot [
      color=ztracecolor,
      thick
    ] table [
      col sep=comma,
      x=frames,
      y=\zcol
    ] {\tracecsv};
    \@ifundefined{nolabels}{\addlegendentry{inferred state}}{}
  }

  % remember min/max y-axis values for next plot
  \pgfplotsextra{
    \pgfmathparse{\pgfkeysvalueof{/pgfplots/ymin}}
    \global\let\ymin\pgfmathresult
    \pgfmathparse{\pgfkeysvalueof{/pgfplots/ymax}}
    \global\let\ymax\pgfmathresult
  }

\end{axis}
\@ifundefined{plotlabel}{}{
  \node[anchor=north west,rectangle,draw,fill=black] at (trace.north west) {\plotlabel};
}

\@ifundefined{histogramcsv}{}{
  \begin{axis}[
    at={($(trace.east) + (4mm,0)$)},
    name=histogram,
    anchor=west,
    width=0.25\plotwidth,
    height=\plotheight,
    scale only axis=true,
    ylabel=\empty,
    yticklabel=\empty,
    xtick distance=0.01,
    xlabel=probability,
    grid=major,
    grid style={dashed, very thin},
    enlarge x limits={value=0.4,upper},
    scaled ticks=false,
    ymin=\ymin,
    ymax=\ymax,
    ticklabel style={font=\small},
    legend style={fill=black, nodes={scale=0.6, transform shape}},
  ]

    \addplot+[
      xbar interval,
      mark=none,
      color=tracecolor,
      fill=tracecolor,
      fill opacity=0.6,
      draw=none,
    ] table [
      col sep=comma,
      y=\histbincol,
      x=\histcountcol,
    ] {\histogramcsv};
    \addlegendentry{intensity histogram}

    \addplot[
      color=intensitymodelcolor!80!black,
      thick
    ] table [
      col sep=comma,
      y=\histbincol,
      x=\modelfitcol
    ] {\histogramcsv};
    \addlegendentry{inferred model}

  \end{axis}
}

  \def\noylabels{}
  \def\noxlabel{}
  \setlength\plotwidth{0.1\plotwidth}
  \setlength\plotheight{\plotwidth}
  \pgfplotsset{every axis/.style={anchor=below south east,at={($(histogram.south east)-(2mm,0)$)}}}
  \def\eps{0.001}%  skip bars for values below eps
\@ifundefined{noylabels}{}{
  \pgfplotsset{yticklabel=\empty}
  \pgfplotsset{ylabel=\empty}
}
\@ifundefined{noxlabel}{
  \pgfplotsset{xlabel=$n$}
}{
  \pgfplotsset{xlabel=\empty}
}
\begin{axis}[
  width=\plotwidth,
  height=\plotheight,
  scale only axis=true,
  enlarge x limits={abs=1.5},
  enlarge y limits=0,
  ymin=0,
  ymax=1,
  scaled ticks=false,
  ticklabel style={font=\tiny},
  xtick distance=1,
  axis background/.style={fill=white},
]

  \addplot+[
    ybar,
    bar width=1,
    mark=none,
    fill=posteriorcolor,
    fill opacity=0.6,
    draw=posteriorcolor,
    y filter/.expression={
      y < \eps ? nan : y
    },
  ] table [
    col sep=comma,
    y=\posteriorcol,
    x=n,
  ] {\posteriorcsv};

  \ifdefined\posteriorcolextra
    \addplot+[
      ybar,
      bar width=1,
      mark=none,
      fill=posteriorcolor!60!black,
      fill opacity=0.6,
      draw=posteriorcolor,
      y filter/.expression={
        y < \eps ? nan : y
      },
    ] table [
      col sep=comma,
      y=\posteriorcolextra,
      x=n,
    ] {\posteriorcsv};
  \fi

\end{axis}

\end{tikzpicture}

    \vspace{-4mm}
  \end{panel}

  \begin{panel}{(b)}{\textwidth}
    \hspace{-2mm}
    \setlength\plotwidth{\textwidth}%
    \setlength\plotheight{36mm}%
    \def\histogramcsv{figures/data/sim_counting/intensity_histogram.csv}%
\def\tracecsv{figures/data/sim_counting/trace_and_fit_N20.csv}%
\def\traceintensitycol{trace}%
\def\zcol{fit}%
\def\histbincol{N20_bins}%
\def\histcountcol{N20_measured}%
\def\modelfitcol{N20_model}%
\def\posteriorcsv{figures/data/sim_counting/posteriors.csv}%
\def\posteriorcol{posterior_20}%
\tikzsetnextfilename{sim_trace_n20}%
\begin{tikzpicture}
  \@ifundefined{nolabels}{
  \pgfplotsset{xlabel=frames}
  \pgfplotsset{ylabel=intensity}
  \pgfplotsset{grid=major}
  \pgfplotsset{grid style={dashed}}
}{
  \pgfplotsset{xticklabel=\empty}
  \pgfplotsset{yticklabel=\empty}
  \pgfplotsset{ticks=none}
  \pgfplotsset{scaled ticks=false}
}
\@ifundefined{histogramcsv}{}{
  \setlength{\plotwidth}{0.73\plotwidth}
}
\begin{axis}[
  name=trace,
  width=\plotwidth,
  height=\plotheight,
  scale only axis=true,
  enlarge x limits=false,
  xtick distance=500,
  ticklabel style={font=\small},
  legend style={fill=black, nodes={scale=0.6, transform shape}},
]

  \addplot [
    color=tracecolor,
    mark=*,
    mark size=0.7pt,
    mark options={line width=0},
    fill opacity=0.8,
    draw opacity=0.0,
  ] table [
    col sep=comma,
    x=frames,
    y=\traceintensitycol
  ] {\tracecsv};
  \@ifundefined{nolabels}{\addlegendentry{intensity trace}}{}

  \@ifundefined{zcol}{}{
    \addplot [
      color=ztracecolor,
      thick
    ] table [
      col sep=comma,
      x=frames,
      y=\zcol
    ] {\tracecsv};
    \@ifundefined{nolabels}{\addlegendentry{inferred state}}{}
  }

  % remember min/max y-axis values for next plot
  \pgfplotsextra{
    \pgfmathparse{\pgfkeysvalueof{/pgfplots/ymin}}
    \global\let\ymin\pgfmathresult
    \pgfmathparse{\pgfkeysvalueof{/pgfplots/ymax}}
    \global\let\ymax\pgfmathresult
  }

\end{axis}
\@ifundefined{plotlabel}{}{
  \node[anchor=north west,rectangle,draw,fill=black] at (trace.north west) {\plotlabel};
}

\@ifundefined{histogramcsv}{}{
  \begin{axis}[
    at={($(trace.east) + (4mm,0)$)},
    name=histogram,
    anchor=west,
    width=0.25\plotwidth,
    height=\plotheight,
    scale only axis=true,
    ylabel=\empty,
    yticklabel=\empty,
    xtick distance=0.01,
    xlabel=probability,
    grid=major,
    grid style={dashed, very thin},
    enlarge x limits={value=0.4,upper},
    scaled ticks=false,
    ymin=\ymin,
    ymax=\ymax,
    ticklabel style={font=\small},
    legend style={fill=black, nodes={scale=0.6, transform shape}},
  ]

    \addplot+[
      xbar interval,
      mark=none,
      color=tracecolor,
      fill=tracecolor,
      fill opacity=0.6,
      draw=none,
    ] table [
      col sep=comma,
      y=\histbincol,
      x=\histcountcol,
    ] {\histogramcsv};
    \addlegendentry{intensity histogram}

    \addplot[
      color=intensitymodelcolor!80!black,
      thick
    ] table [
      col sep=comma,
      y=\histbincol,
      x=\modelfitcol
    ] {\histogramcsv};
    \addlegendentry{inferred model}

  \end{axis}
}

  \def\noylabels{}
  \def\noxlabel{}
  \setlength\plotwidth{0.1\plotwidth}
  \setlength\plotheight{\plotwidth}
  \tikzset{every axis/.style={anchor=below south east,at={($(histogram.south east)-(2mm,0)$)}}}
  \def\eps{0.001}%  skip bars for values below eps
\@ifundefined{noylabels}{}{
  \pgfplotsset{yticklabel=\empty}
  \pgfplotsset{ylabel=\empty}
}
\@ifundefined{noxlabel}{
  \pgfplotsset{xlabel=$n$}
}{
  \pgfplotsset{xlabel=\empty}
}
\begin{axis}[
  width=\plotwidth,
  height=\plotheight,
  scale only axis=true,
  enlarge x limits={abs=1.5},
  enlarge y limits=0,
  ymin=0,
  ymax=1,
  scaled ticks=false,
  ticklabel style={font=\tiny},
  xtick distance=1,
  axis background/.style={fill=white},
]

  \addplot+[
    ybar,
    bar width=1,
    mark=none,
    fill=posteriorcolor,
    fill opacity=0.6,
    draw=posteriorcolor,
    y filter/.expression={
      y < \eps ? nan : y
    },
  ] table [
    col sep=comma,
    y=\posteriorcol,
    x=n,
  ] {\posteriorcsv};

  \ifdefined\posteriorcolextra
    \addplot+[
      ybar,
      bar width=1,
      mark=none,
      fill=posteriorcolor!60!black,
      fill opacity=0.6,
      draw=posteriorcolor,
      y filter/.expression={
        y < \eps ? nan : y
      },
    ] table [
      col sep=comma,
      y=\posteriorcolextra,
      x=n,
    ] {\posteriorcsv};
  \fi

\end{axis}

\end{tikzpicture}

    \vspace{-4mm}
  \end{panel}

  \begin{panel}{(c)}{0.45\textwidth}
    \setlength\plotwidth{72mm}%
    \setlength\plotheight{72mm}%
    \def\posteriormatrixcsv{figures/data/sim_counting/heatmap.csv}%
\def\lbfcscsv{figures/data/sim_counting/heatmap_lbfcs.csv}%
\tikzsetnextfilename{sim_counting_estimates}%
\begin{tikzpicture}
  \begin{axis}[
    name=trace,
    width=\plotwidth,
    height=\plotheight,
    xlabel=true \n,
    ylabel=estimated \n,
    enlarge x limits=false,
    enlarge y limits=false,
    grid=major,
    grid style={dashed},
    scaled ticks=false,
    ticklabel style={font=\small},
    xtick align=outside,
    xtick pos=lower,
    ytick align=outside,
    ytick pos=lower,
    colorbar,
  ]

    \pgfplotsset{
      colormap={posteriorcolormap}{
        color(0.0)=(white)
        color(0.2)=(funkey_color_2)
        color(1.0)=(funkey_color_2!50!black)
      }
    }

    \addplot[
      matrix plot*,
      mesh/cols=30,
      point meta=explicit,
    ] table [
      col sep=comma,
      x=true_n,
      y=n,
      meta=posterior,
    ] {\posteriormatrixcsv};

  \end{axis}

  \begin{axis}[
    name=trace,
    width=\plotwidth,
    height=\plotheight,
    xmin=0.5,
    xmax=30.5,
    ymin=0.5,
    ymax=35.5,
    grid=major,
    grid style={dashed},
    xticklabel=\empty,
    yticklabel=\empty,
    scaled ticks=false,
    domain=0:30,
    ticklabel style={font=\small},
    legend pos=north west,
  ]

    \addlegendimage{mark=square*,only marks,funkey_color_2}
    \addlegendentry{\ours}
    \addlegendimage{mark=*,only marks,funkey_color_1}
    \addlegendentry{\lbfcs}

    \addplot[
      mark=*,
      only marks,
      mark size=1.4pt,
      mark options={draw=white,fill=funkey_color_1,draw opacity=0.6,fill opacity=0.7},
    ] table [
      col sep=comma,
      x=n,
      y=lbfcs_count,
    ] {\lbfcscsv};

    \addplot[no marks,thick,gray] {x};

  \end{axis}

\end{tikzpicture}

  \end{panel}
  \hspace{1cm}
  \begin{panel}{(d)}{0.55\textwidth}
    \begin{panel}{}{\textwidth}
      \hspace{5mm}%
      \small%
      \setlength\plotwidth{68mm}%
      \setlength\plotheight{22mm}%
      \def\tracevarcsv{figures/data/trace_length/trace_length_results.csv}%
\def\xcol{length}%
\def\xlabel{length [frames]}%
\tikzsetnextfilename{trace_var_length}%
\begin{tikzpicture}
  \def\intervalplot#1#2{
  \addplot[
    color=#2,
    very thick,
  ] table [
    col sep=comma,
    x=\xcol,
    y=max_likelihoods_#1,
  ] {\tracevarcsv};
  \addplot[
    name path=lower,
    draw=none,
    fill=none,
    forget plot,
  ] table [
    col sep=comma,
    x=\xcol,
    y expr=\thisrow{max_likelihoods_#1} - \thisrow{variance_#1}
  ] {\tracevarcsv};
  \addplot[
    name path=upper,
    draw=none,
    fill=none,
    forget plot,
  ] table [
    col sep=comma,
    x=\xcol,
    y expr=\thisrow{max_likelihoods_#1} + \thisrow{variance_#1}
  ] {\tracevarcsv};
  \addplot [
    color=#2,
    opacity=0.4,
    forget plot,
  ] fill between[of=lower and upper];
  \@ifundefined{nolegend}{
    \addlegendentry{$n=#1$}
  }{}
}
\begin{axis}[
  width=\plotwidth,
  height=\plotheight,
  scale only axis,
  xlabel=\xlabel,
  enlarge x limits=false,
  enlarge y limits=false,
  xtick distance=4000,
  ytick distance=5,
  ymin=0,
  grid=major,
  grid style={dashed},
  scaled ticks=false,
  legend columns=4,
  legend style={nodes={scale=0.6, transform shape}},
]
  \intervalplot{20}{funkey_color_1}
  \intervalplot{15}{funkey_color_2}
  \intervalplot{10}{funkey_color_3}
  \intervalplot{5}{funkey_color_4}
\end{axis}

\end{tikzpicture}%
%
    \end{panel}
    \begin{panel}{(e)}{\textwidth}
      \hspace{5mm}%
      \small%
      \setlength\plotwidth{68mm}%
      \setlength\plotheight{22mm}%
      \def\tracevarcsv{figures/data/trace_snr/trace_snr_results.csv}%
\def\xcol{snr}%
\def\xlabel{SNR}%
\def\nolegend{}%
\tikzsetnextfilename{trace_var_snr}%
\begin{tikzpicture}
  \def\xtickdistance{1}
  \def\intervalplot#1#2{
  \addplot[
    color=#2,
    very thick,
  ] table [
    col sep=comma,
    x=\xcol,
    y=max_likelihoods_#1,
  ] {\tracevarcsv};
  \addplot[
    name path=lower,
    draw=none,
    fill=none,
    forget plot,
  ] table [
    col sep=comma,
    x=\xcol,
    y expr=\thisrow{max_likelihoods_#1} - \thisrow{variance_#1}
  ] {\tracevarcsv};
  \addplot[
    name path=upper,
    draw=none,
    fill=none,
    forget plot,
  ] table [
    col sep=comma,
    x=\xcol,
    y expr=\thisrow{max_likelihoods_#1} + \thisrow{variance_#1}
  ] {\tracevarcsv};
  \addplot [
    color=#2,
    opacity=0.4,
    forget plot,
  ] fill between[of=lower and upper];
  \@ifundefined{nolegend}{
    \addlegendentry{$n=#1$}
  }{}
}
\begin{axis}[
  width=\plotwidth,
  height=\plotheight,
  scale only axis,
  xlabel=\xlabel,
  enlarge x limits=false,
  enlarge y limits=false,
  xtick distance=4000,
  ytick distance=5,
  ymin=0,
  grid=major,
  grid style={dashed},
  scaled ticks=false,
  legend columns=4,
  legend style={nodes={scale=0.6, transform shape}},
]
  \intervalplot{20}{funkey_color_1}
  \intervalplot{15}{funkey_color_2}
  \intervalplot{10}{funkey_color_3}
  \intervalplot{5}{funkey_color_4}
\end{axis}

\end{tikzpicture}%
%
    \end{panel}
  \end{panel}

  \caption{
    \panelref{a, b} Traces simulated from $\n=10/20$ emitters, and a density plot of measured intensities.
    The inferred state and inferred model demonstrate accurate fitting of both the intensity and transition distributions.
    Inset: posterior distribution estimated by \ours.
    %
    \panelref{c} The \ours posterior is able to accurately estimate
    significantly higher molecular counts than the current state of the art,
    \lbfcs.
    %
    \panelref{d} As trace length increases, the variance of the \ours posterior
    decreases, with limited improvement above 4000 frames.
    %
    \panelref{e} At low signal-to-noise, \ours severely over estimates the count, estimating the maximum count tested 
    as the most likely.
  }
  \label{fig:results:sim_counting}

\end{figure*}
