\def\tikzmath#1#2{\tikz[remember picture,baseline=(#1.base),inner sep=0pt] \node (#1) {$\displaystyle #2$};}

\newenvironment{panel}[2]{%
  \begin{minipage}[t][][t]{#2}%
    \strut#1%

    \vspace{-\baselineskip}%
}{%
  \end{minipage}%
}%
\newenvironment{panelcolumn}[1]{%
  \begin{minipage}[t][][t]{#1}%
}{%
  \end{minipage}%
}%

% useful for setting/computing the width of plots
\newlength{\plotwidth}
\newlength{\plotheight}

% draw 5 example fluorophores, args are "on" or "off"
\def\blinky#1#2#3#4#5{
  \begin{tikzpicture}[node distance=3mm]
    \node[site#1] (s1) {};
    \node[site#2,right of=s1] (s2) {};
    \coordinate (s12) at ($(s1)!.5!(s2)$);
    \node[site#3,right of=s2] (s3) {};
    \node[site#4,below of=s12] (s4) {};
    \node[site#5,right of=s4] (s5) {};
    \foreach \angle in {0, 45, 90, 135, 180, 225, 270, 315}{
      \begin{scope}[rotate=\angle]
        \draw[sitelight#1] ($(s1)+(0,0.13)$) -- ($(s1)+(0,0.16)$);
        \draw[sitelight#2] ($(s2)+(0,0.13)$) -- ($(s2)+(0,0.16)$);
        \draw[sitelight#3] ($(s3)+(0,0.13)$) -- ($(s3)+(0,0.16)$);
        \draw[sitelight#4] ($(s4)+(0,0.13)$) -- ($(s4)+(0,0.16)$);
        \draw[sitelight#5] ($(s5)+(0,0.13)$) -- ($(s5)+(0,0.16)$);
      \end{scope}
    }
  \end{tikzpicture}
}
