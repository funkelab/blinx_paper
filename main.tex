\documentclass[twocolumn]{article}

\usepackage{amsmath}
\usepackage{amssymb}
\usepackage{graphicx}
\usepackage{gensymb}
\usepackage{tikz}
\usepackage{pgfplots}
\usepackage[textwidth=1.7cm,textsize=tiny,linecolor=orange!60!white,bordercolor=orange!60!white,color=orange!20!white]{todonotes}
\setlength{\marginparwidth}{1.7cm}
\usepackage{xspace}
\usepackage{soul}
\usepackage{siunitx}
\usepackage{natbib}
\usepackage{bm}
\usepackage[title]{appendix}
\usepackage[hidelinks]{hyperref}
\usetikzlibrary{external}
\tikzexternalize[prefix=figures/tikzexternal/]
\makeatletter
\renewcommand{\todo}[2][]{\tikzexternaldisable\@todo[#1]{#2}\tikzexternalenable}
\makeatother
\input{colors}
\usetikzlibrary{arrows.meta}
\usetikzlibrary{backgrounds}
\usetikzlibrary{calc}
\usetikzlibrary{decorations.pathreplacing}
\usetikzlibrary{fit}
\usetikzlibrary{spy}
\pgfplotsset{compat=1.18}
\usepgfplotslibrary{fillbetween}

\tikzstyle{var}=[circle,draw,minimum width=8mm,fill=funkey_color_2!20!white]
\tikzstyle{state}=[var,fill=funkey_color_1!20!white]
\tikzstyle{observation}=[var,fill=funkey_color_3!20!white]
\tikzstyle{plate}=[rectangle,rounded corners,draw,fill=gray!20!white]
\tikzstyle{siteon}=[circle,inner sep=2pt,fill=orange!80!white,draw=orange]
\tikzstyle{siteoff}=[circle,inner sep=2pt,fill=gray]
\tikzstyle{sitelighton}=[orange,thick,line cap=round]
\tikzstyle{sitelightoff}=[draw=none]
\tikzstyle{arrow}=[->,-{Latex}]
\tikzstyle{image}=[rectangle,draw,inner sep=0]

\def\tikzmath#1#2{\tikz[remember picture,baseline=(#1.base),inner sep=0pt] \node (#1) {$\displaystyle #2$};}


% references
\def\figref#1{Fig.~\ref{#1}}
\def\panelref#1{{\bf(#1)}}
\def\secref#1{Section~\ref{#1}}
\def\tabref#1{Table~\ref{#1}}
\def\eqref#1{(Eq.~\ref{#1})}

% common abbreviations
\newcommand{\eg}{\emph{e.g.}\xspace}
\newcommand{\ie}{\emph{i.e.}\xspace}


% introduce a new word
\def\introduce#1{{\color{funkey_color_1}\ul{#1}}}
%\def\introduce#1{#1}  % uncomment this for final version

\def\mathlet#1#2{\pgfmathparse{#2}\let#1\pgfmathresult}

% General Math Notation
\def\vct#1{\bm{#1}}  % vectors in bold
\def\trans{\intercal}  % matrix transpose symbol

% Our Method
\def\ours{{blinx}\xspace}
\def\lbfcs{{lbFCS}\xspace}
\def\qpaint{{qPAINT}\xspace}

% Variables
\def\n{\ensuremath{n}\xspace}                 % count
\def\ndist{\ensuremath{\vct{n}}\xspace}       % distribution of predictions
\def\truen{\ensuremath{n^*}\xspace}           % true count
\def\estimatedn{\ensuremath{\hat{n}}\xspace}  % estimated count
\def\z#1{\ensuremath{z_{#1}}\xspace}          % number of active emitters at time t: \y{t}
\def\x#1{\ensuremath{x_{#1}}\xspace}          % intensity at time t: \x{t}
\def\xp#1{\ensuremath{\tilde{x}_{#1}}\xspace} % intensity at time t in photon space: \xp{t}
\def\states{\ensuremath{\vct{z}}\xspace}      % all active emitters over time
\def\trace{\ensuremath{\vct{x}}\xspace}       % all intensities over time
\def\pon{\ensuremath{p_\text{on}}\xspace}     % prob. for activating
\def\poff{\ensuremath{p_\text{off}}\xspace}   % prob. for deactivating
\def\re{\ensuremath{r_e}\xspace}              % fluorophore emission rate
\def\rb{\ensuremath{r_b}\xspace}              % background emission rate
\def\camoffset{\ensuremath{\mu}\xspace}       % camera offset
\def\camvar{\ensuremath{\sigma^2}\xspace}       % camera variance
\def\camgain{\ensuremath{g}\xspace}           % camera gain
\def\parameters{\ensuremath{\theta}\xspace}   % all parameters (mu, sigma, ...)
\def\parameterst{\ensuremath{\parameters_T}\xspace}  % transition parameters
\def\parameterse{\ensuremath{\parameters_E}\xspace}  % emission parameters
\def\parametersc{\ensuremath{\parameters_C}\xspace}  % camera parameters
\def\photons{\ensuremath{c}\xspace}           % number of photons per spot
\def\deltat{\ensuremath{\Delta t}\xspace}     % length of frame

\def\pchain{\ensuremath{\alpha\xspace}}       % recursive probability for forward algorithm
\def\pchainnorm{\ensuremath{\tilde{\alpha}\xspace}}
                                              % \pchain normalized
\def\norms{\ensuremath{\beta\xspace}} % normalization factors for \pchain
\def\object{object\xspace} % what to call things we want to count
\def\objects{objects\xspace} % what to call things we want to count plural
\def\smallobjects{subunits\xspace} % what to call things we want to count plural

\DeclareMathOperator{\poisson}{Poisson}

% units
\def\micrometer{\unit{\micro\meter}\xspace}
\def\nanometer{\unit{\nano\meter}\xspace}

% Functions
%\newcommand{\argmax}[1]{\mathop{\arg\max}_{#1}\hspace{0.5em}}


\begin{document}

\title{A Basyian approach to the molecular counting problem}

\author{
  Alexander Hillsley$^{1}$,
  Johannes Stein$^{2}$,
  Paul Tillberg$^{1}$,
  David Stern$^{1}$,
  Jan Funke$^{1}$\thanks{Corresponding Author}

  \normalsize{$^1$ HHMI},
  \normalsize{$^2$ Wyss}
}

\maketitle

\begin{abstract}

  We address the problem of inferring the number of independently blinking
  fluorescent light emitters, when only their combined intensity contributions
  can be observed at each timepoint.
  %
  This problem occurs regularly in light microscopy of \objects that are
  smaller than the diffraction limit, where one wishes to count the number of
  fluorescently labelled \smallobjects.
  %
  Our proposed solution directly models the photo-physics of the system, as
  well as the blinking kinetics of the fluorescent emitters as a fully
  differentiable \hmm.
  %
  Given a trace of intensity over time, our model jointly estimates the
  parameters of the intensity distribution per emitter, their blinking rates,
  as well as a posterior distribution of the total number of fluorescent
  emitters.
  %
  We show that our model is
  consistently more accurate and increases the range of countable subunits by a
  factor of two compared to current state-of-the-art methods, which count based on
  autocorrelation and blinking frequency.
  %
  Furthermore, we demonstrate that our model can be used to investigate the
  effect of blinking kinetics on counting ability, and therefore can inform
  experimental conditions that will maximize counting accuracy.


  % We address the problem of inferring the number of independently blinking
  % \introduce{fluorescent light emitters}, when only their combined
  % \introduce{intensity} contributions can be observed at each
  % \introduce{timepoint}.
  % %
  % % why do we want to do that?
  % %
  % This problem occurs regularly in light microscopy of \introduce{complexes}
  % that are smaller than the diffraction limit, where one wishes to count the
  % number of \introduce{fluorescently} labelled \introduce{subunits}.
  % %
  % % how do we do that?
  % %   differentiable markov chain
  % %   blinx estimates number of active emitters for each timestep
  % %   holistic model
  % %
  % We estimate the number of subunits of a complex by modeling the light
  % emitted per complex over time as a Markov-chain.
  % %
  % Crucially, our model is differentiable, which allows us to jointly estimate
  % the parameters of the intensity distribution per emitter as well as their
  % blinking rates through maximum likelihood optimization.
  % %
  % % how good is it?
  % %
  % Compared to the current state of the art, which uses histogram-based
  % heuristics to infer distribution parameters, our holistic model is
  % consistently more accurate and extends the range of countable subunits by a
  % factor of two.
  % %
  % Furthermore, our model allows us to inform experimentalists about optimal
  % imaging parameters to increase accuracy.

\end{abstract}

\section{Introduction}

% counting structures is important
% super resolution to the rescue
%
Fluorescence microscopy is a foundational tool in the field of biology and
provides useful information on the abundance and localization of labeled
structures.
%
  However, determining the exact molecular count of subunits in a complex smaller than the diffraction limit
  remains a challenge. Single Molecule Localization Microscopy (SMLM) such as PALM
  \cite{betzig_imaging_2006}, STORM \cite{rust_sub-diffraction-limit_2006}, and
  DNA-PAINT \cite{schnitzbauer_super-resolution_2017} have circumvented the diffraction limit
  by achieving resolutions as high as \SI{20}{\nm}.

% key concept of super-resolution: fluorescence over time
% doesn't work for many very small structures
%
The underlying principle of SMLM is the
seperation of fluorescence signals from individual fluorophores over time:
%
  only a sparse subset of fluorescent emitters are \introduce{\emph{active}}
  (\ie, emitting light) at each timepoint.
  %
  Each spatially distinct emitter can then be localized with sub-pixel accuracy
  by fitting its point spread function. Subsequently, all localizations over
  all frames can be combined into a super-resolved reconstruction.
  %
  Although this method allows to spatially separate the subunits of larger complexes, the resolution is insufficient to determine the number of subunits of complexes at the nano-scale, prompting the need for new methods of molecular
  counting.

% intro to molecular counting for computer scientists: what is the data, what
% do we want to predict?
%
At that scale the challenge for molecular counting is to disentangle the contributions
of multiple fluorescent emitters that overlap in space.
%
  % multiple emitters, single spot
  % intensity trace of combined intensity
  % unknown number of active emitters per timepoint
  % noisy measurements, randomly distributed with unknown parameters
  % dependency on "kinetics" of emitters
  %
  % -> joint inference problem: number of active emitters per timepoint and
  %    parameters
  This problem is ill-posed in a static image, where the intensities of all
  emitters are combined into a single \introduce{spot}. However, for blinking fluorophores the
  fluctuating \introduce{intensity trace} over time can provide enough
  information to infer the number of emitters:
  %
  %While this problem is ill-posed in a static image, the fluctuating
  %\introduce{intensity trace} of the combined intensity of independently
  %blinking emitters over time in a single \introduce{spot} can provide enough
  %information to infer the number of emitters:
  %
  At each timepoint, a subset of all emitters will be active, and thus
  contribute to the combined intensity. Between timepoints, the number of
  active emitters changes accordingly to their underlying \introduce{blinking
  rates}.
  %
  None of the parameters of this process (\eg, intensity of a single emitter,
  probability of activating or deactivating) are known \emph{a-priori} and need
  to be inferred jointly with the total number of emitters.

Here we propose \ours, a differentiable Markov-chain model for intensity
traces, which is conditioned on the total number of emitters, the parameters of
the intensity distribution, and the blinking rates.
%
  \ours allows joint estimation of the parameters of the intensity distribution
  per emitter as well as their blinking rates through maximum likelihood
  optimization.
  %
  The most likely number of emitters per trace is then found as the maximum
  \emph{a-posteriori} solution.

We compare \ours on synthetic traces (where the true number of emitters is
known) and compare its predictions against the current state-of-the-art method,
\lbfcs~\citep{stein_calibration-free_2021}.
%
  \ours is consistently more accurate than \lbfcs and extends the range of
  countable subunits up to 30.
  %
  Furthermore, we show that \ours can be used to inform experimental design by
  identifying conditions that facilitate molecular counting.

\paragraph{Related Work}

% molecular counting
Many molecular counting approaches similarly
  make use of imaging over time to extract additional information from the
  system. 
  %Some of the simplest systems count events that are designed to happen
  %only once per emitter such as photobleaching steps \cite{Ulbrich_subunit_2007} or
  %blinks from a 1 time photo-switching emitter
  %\cite{gunzenhauser_quantitative_2012}. However, these events are often difficult to
 % detect, and can easily be missed leading to severe undercounting.
Some of the simplest systems count events that happen only a single time per emitter \cite{Ulbrich_subunit_2007}; \cite{gunzenhauser_quantitative_2012}.
Other methods make use of emitters that blink repeatedly to reduce the
dependence on individual events.
%
  By modeling the blinking rates of STORM emitters as a continuous time
  Markov process \cite{patel_blinking_2021, rollins_stochastic_2015}, one can extract the total count of emitters
  in a spot. However, these models are limited in accuracy by the complex
  photobleaching nature of the emitters used. Another temporal counting method is
  qPAINT \cite{jungmann_quantitative_2016} which utilizes DNA-PAINT to minimize photobleaching and
  correlates the frequency of repeated binding and unbinding events to the
  molecular count. This method shows an excellent accuracy for counts under 15,
  but relies heavily on previous knowledge of binding rates and is
  susceptible to simultaneous binding events.

%Although not precise enough to determine molecular counts, the linear
%relationship observed between fluorescence intensity and emitter abundance
%\cite{schmied_fluorescence_2012}, provides additional information to improve the
%accuracy of these methods.
%
  Still other methods use correlation functions to combine both the temporal and
  intensity information to estimate the molecular count, such as balanced
  super-resolution optical fluctuation imaging and fluorescence
  correlation spectroscopy (FCS) which has been used to estimate the copy
  number of specific proteins within the nuclear pore complex
  \cite{otsuka_quantitative_2023}. Similar techniques have also been applied to the
  consistent blinking behavior of DNA-PAINT in \lbfcs \cite{stein_calibration-free_2021}, which has
  accurately counted up to ~ eight molecules and without relying on previous
  calibrations. These models work by fitting the entire observation series to a
  single equation, in the process greatly simplifying the information. 

\section{Method}

\subsection{Model}

Consider a complex smaller than the fluorescence spatial resolution limit,
consisting of \n individually labeled subunits.

Each subunit is labeled such that it stochastically fluoresces over time,
independently of the others, resulting in $\y{t} \in [0, \n]$ fluorophores
illuminated at any given timepoint $t$, producing a fluorescent signal of
intensity \x{t}. We can then estimate \n, from an observed intensity sequence
$\trace = (\x{1},\ldots,\x{T})$ through maximum likelihood estimation.

\begin{equation}
  \estimatedn =
    \argmax{\n}
    \max_\parameters
    p(\trace|\n,\parameters)
  \text{,}
\end{equation}
%
where \parameters are parameters of the model that we will introduce below.

Assuming that each state \y{t} depends only on the previous state \y{t-1}, this process can be modeled as a hidden Markov model. Expanding to include $\states = \left( \y{1}, \ldots, \y{T} \right)$:

\begin{equation}
  p(\trace|\n,\parameters) =
    \sum_{\states}
    p_\parameters(\x{1}|\y{1}) p_\parameters(\y{1}|\n)
    \prod_{t=2}^{T}
    p_\parameters(\x{t}|\y{t}) p_\parameters(\y{t}|\y{t-1},\n)
\end{equation}

The emission distribution $p(\x{t}|\y{t})$ describes the relationship
between the observations \trace, and the hidden state \states, and can be
approximated by a log-normal distribution (??), \ie,
%
\begin{equation}
  p(\x{t}|\y{t}) =
    \frac{1}{\x{t}\sigma\sqrt{2\pi}}
    \exp \left(
      - \frac{|\log(\x{t}) - \y{t}\mu|^2}{2\sigma^2}
    \right)
  \text{,}
\end{equation}
%
where $\mu$ and $\sigma$ are the mean and standard deviation of the intensity
of a single activated emitter.

The transition distribution $p(\y{t}|\y{t-1},\n)$ describes the kinetics of
arriving at any hidden state \y{t} from any other hidden state \y{t-1}. The
probability of a single subunit which is dark at time $t-1$ illuminating at
time $t$ is defined as \pon. Conversely, the probability of a single subunit
which is active at time $t-1$ going dark at time $t$ can be defined as \poff.
Assuming that each emitter behaves independently, and that all share a common
\pon and \poff, the transition probability between any two states can be
written as:

Equation from oneNote

%Specifically for DNA-PAINT experiments, \pon and \poff can be defined, through the exponential distribution, in terms of the rate constants k_on and k_off often used to describe DNA binding and dissociation.

%P_on = exposure_time * k_on * exp(-exposure_time * k_on)
%P_off = exposure_time * k_off * exp(-exposure_time * k_off)

Altogether, the probability of observing trace \trace can be written as a
function of five parameters: $p(\trace|\n,\parameters)$

\subsection{Inference}

\todo{mention: no closed form solution to estimate \parameters}
\todo{solution: gradient ascent on \parameters}

The model was fit using gradient ascent to find the optimal combination of these 5 parameters to maximize the likelihood of observing intensity trace X. While pon, poff, mu and sigma are continuous, N is discrete and therefore not differentiable. Fortunately, the sample space for possible N values is relatively small ( ~ 50) and it is possible to independently fit a model for each N, then take a maximum over likelihoods to find the optimal N. 
Before fitting for each given N, a rough grid search was used to determine initial guesses for each of the parameters. In the case of multiple local maxima, all were used to initiate gradient ascent rather than just the global maxima to reduce the risk of finding false summits. The likelihood p(X | pon, poff, mu, sigma) was calculated using the forward algorithm and the gradient of this function on all 4 parameters was calculated using JAX. Parameters were then optimized using stochastic gradient ascent.
One challenge is that for long observation sequences the likelihood of any consecutive observation sequence becomes vanishingly small, and numerically unstable. To alleviate this effect, all probabilities were computed in log space which results in summation, rather than multiplication of consecutive probabilities, reducing the small number problem. Additionally, the probabilities for each timepoint were normalized to sum to 1, and later re-converted to further reduce this issue. JAX was used to vectorize almost the entire fitting process allowing the quick calculation of gradients and the fitting of many traces in parallel. 


\begin{figure*}
  \includegraphics[width=\linewidth]{figures/blinx_overview.png}
  \caption{A) Overview of the blinx method (scale bar: 1 $\mu$m) B) blinx is based on a Hidden Markov Model who's parameters are optimized to build the final posterior
  distribution}
  \label{fig:method:overview}
\end{figure*}

\section{Results}

\begin{figure*}
  \includegraphics[width=\linewidth]{figures/placeholders/figure_2_simulated_counting.png}
  \caption{A) Trace simulated from 10 emitters, and the posterior distribution estimated by blinx B) simulated from 20 emitters
  C) The blinx posterior is able to accurately estimate significantly higher molecular counts than the current state of the art, lbFCS
  D) As trace length increases, the variance of the blinx posterior decreases E) As the signal to noise ratio increases the variance of 
  the blinx posterior decreases. }
  \label{fig:method:overview}
\end{figure*}

\begin{figure*}

  \begin{panel}{(a)}{0.58\textwidth}
    \vspace{-2mm}%
    \small%
    \setlength\plotwidth{42mm}%
    \setlength\plotheight{60mm}%
    \def\kineticsheatmapcsv{figures/data/kinetics_grid/heatmap.csv}%
\tikzsetnextfilename{kinetics_sim_sweep}%
\begin{tikzpicture}
  \begin{axis}[
    width=\plotwidth,
    height=\plotheight,
    scale only axis=true,
    name=sweep,
    xlabel=\pon,
    ylabel=\poff,
    enlarge x limits=false,
    enlarge y limits=false,
    scaled ticks=false,
    ticklabel style={/pgf/number format/.cd,fixed,precision=3},
    xtick distance=0.04,
    xtick align=outside,
    xtick pos=lower,
    ytick align=outside,
    ytick pos=lower,
    colorbar,
  ]

    \pgfplotsset{
      colormap={posteriorcolormap}{
        color(0.0)=(funkey_color_2!50!white)
        color(0.1)=(funkey_color_2)
        color(0.9)=(funkey_color_1)
        color(1.0)=(funkey_color_1!50!black)
      }
    }

    \addplot[
      matrix plot*,
      mesh/cols=10,
      point meta=explicit,
    ] table [
      col sep=comma,
      x=p_on,
      y=p_off,
      meta=error,
    ] {\kineticsheatmapcsv};

    % highlight qpaint and lbfcs points
    \node[circle,draw=funkey_color_9,thick] (qpaint) at (axis cs:0.01, 0.2) {};
    \node[circle,draw=funkey_color_9,thick] (lbfcs) at (axis cs:0.02, 0.02) {};

  \end{axis}

  \node[anchor=north] (qpaint_zs) at ($(sweep.north east)+(2.8,0)$) {
    \def\zhistogramcsv{figures/data/kinetics_grid/state_histogram.csv}%
    \def\zhistogramcol{qpaint}%
    \def\noylabels{}%
    \setlength\plotwidth{14mm}%
    \setlength\plotheight{16mm}%
    \tikz{\input{figures/plots/z_histogram.plot}}%
  };
  \node[fit=(qpaint_zs),inner sep=0,rounded corners,draw=funkey_color_9,thick] (qpaint_box) {};

  \node[anchor=south] (lbfcs_zs) at (sweep.south-|qpaint_zs.south) {
    \def\zhistogramcsv{figures/data/kinetics_grid/state_histogram.csv}%
    \def\zhistogramcol{lbfcs}%
    \def\noylabels{}%
    \setlength\plotwidth{14mm}%
    \setlength\plotheight{16mm}%
    \tikz{\input{figures/plots/z_histogram.plot}}%
  };
  \node[fit=(lbfcs_zs),inner sep=0,rounded corners,draw=funkey_color_9,thick] (lbfcs_box) {};

  \draw[funkey_color_3,thick] (qpaint) -- (qpaint-|qpaint_box.west);
  \draw[funkey_color_3,thick] (lbfcs) -- (lbfcs-|lbfcs_box.west);

  \node[anchor=south] at (sweep.north) {\strut Expected Squared Error};

\end{tikzpicture}

  \end{panel}
  \begin{panel}{(b)}{0.42\textwidth}
    \vspace{-2mm}%
    \setlength\plotwidth{56mm}%
    \setlength\plotheight{60mm}%
    \input{figures/panels/kinetics_experiments.tikz}
  \end{panel}

  \begin{panel}{(c)}{\textwidth}
    \small%
    \setlength\plotwidth{\textwidth}%
    \setlength\plotheight{36mm}%
    \def\tracecsv{figures/data/kinetics_grid/traces.csv}%
\def\traceintensitycol{qpaint_trace}%
\def\zcol{qpaint_fit}%
\def\histogramcsv{figures/data/kinetics_grid/intensity_histogram.csv}%
\def\histbincol{qpaint_bins}%
\def\histcountcol{qpaint_measured}%
\def\modelfitcol{qpaint_model}%
\def\posteriorcsv{figures/data/kinetics_grid/posteriors.csv}%
\def\posteriorcol{qpaint}%
\tikzsetnextfilename{sim_trace_qpaint}%
\begin{tikzpicture}
  \def\plotlabel{$\n = 10$}
  \input{figures/plots/fancy_trace.plot}
  \def\noylabels{}
  \def\noxlabel{}
  \setlength\plotwidth{0.1\plotwidth}
  \setlength\plotheight{\plotwidth}
  \pgfplotsset{every axis/.style={
    anchor=below south east,
    at={($(histogram.south east)-(2mm,-4mm)$)},
    xtick distance=10
  }}
  \input{figures/plots/p_n_given_x.plot}
\end{tikzpicture}
%
    \vspace{-4mm}%
  \end{panel}

  \begin{panel}{(d)}{\textwidth}
    \small%
    \setlength\plotwidth{\textwidth}%
    \setlength\plotheight{36mm}%
    \def\tracecsv{figures/data/kinetics_grid/traces.csv}%
\def\traceintensitycol{lbfcs_trace}%
\def\zcol{lbfcs_fit}%
\def\histogramcsv{figures/data/kinetics_grid/intensity_histogram.csv}%
\def\histbincol{lbfcs_bins}%
\def\histcountcol{lbfcs_measured}%
\def\modelfitcol{lbfcs_model}%
\def\posteriorcsv{figures/data/kinetics_grid/posteriors.csv}%
\def\posteriorcol{lbfcs}%
\tikzsetnextfilename{sim_trace_lbfcs}%
\begin{tikzpicture}
  \input{figures/plots/fancy_trace.plot}
  \def\noylabels{}
  \def\noxlabel{}
  \setlength\plotwidth{0.1\plotwidth}
  \setlength\plotheight{\plotwidth}
  \pgfplotsset{every axis/.style={anchor=below south east,at={($(histogram.south east)-(2mm,0)$)}}}
  \input{figures/plots/p_n_given_x.plot}
\end{tikzpicture}

  \end{panel}

  \caption{%
    \panelref{a} The counting ability of \ours is dependant on the blinking kinetics, 
      showing increased model accuracy when $\pon > \poff$. 
      Top inset: when $\pon < \poff$, the distribution of
      observed \z{} states is shifted towards 0. Bottom inset: for
      $\pon = \poff$, the distribution of \z{} states is centered at $1/2$ $n$.
    %
    \panelref{b} Experimental blinking kinetics as a function of imager
      concentration (marker sizes) and temperature (color).
    \captbc
  }
\end{figure*}
\begin{figure*}
  \contcaption{
    \capcont
    %
    \panelref{c} Trace and \ours fit, \pon=0.01, \poff=0.2, and true count of $n$=10, representative of (a) top inset.
    %
    \panelref{d} Trace and \ours fit, \pon=0.02, \poff=0.02, and true count of $n$=10, representative of (a) bottom inset.
  }
  \label{fig:results:campare_kinetics}
\end{figure*}

\begin{figure*}

  \begin{panel}{(a)}{\textwidth}
    \hspace{-2mm}%
    \setlength\plotwidth{\textwidth}%
    \setlength\plotheight{36mm}%
    \input{figures/panels/qpaint_trace_n10.tikz}
    \vspace{-4mm}
  \end{panel}

  \begin{panel}{(b)}{\textwidth}
    \hspace{-2mm}%
    \setlength\plotwidth{\textwidth}%
    \setlength\plotheight{36mm}%
    \def\histogramcsv{figures/data/qpaint_kinetics/intensity_histogram.csv}%
\def\tracecsv{figures/data/qpaint_kinetics/trace_and_fit_N20.csv}%
\def\traceintensitycol{trace}%
\def\zcol{fit}%
\def\histbincol{N20_bins}%
\def\histcountcol{N20_measured}%
\def\modelfitcol{N20_model}%
\def\posteriorcsv{figures/data/qpaint_kinetics/posteriors.csv}%
\def\posteriorcol{posterior_20}%
\tikzsetnextfilename{qpaint_trace_n20}%
\begin{tikzpicture}
  \input{figures/plots/fancy_trace.plot}
  \def\noylabels{}
  \def\noxlabel{}
  \setlength\plotwidth{0.1\plotwidth}
  \setlength\plotheight{\plotwidth}
  \tikzset{every axis/.style={anchor=below south east,at={($(histogram.south east)-(2mm,0)$)}}}
  \input{figures/plots/p_n_given_x.plot}
\end{tikzpicture}

    \vspace{-4mm}
  \end{panel}

  \begin{panel}{(c)}{\textwidth}
    \setlength\plotwidth{72mm}%
    \setlength\plotheight{72mm}%
    \def\posteriormatrixcsv{figures/data/qpaint_kinetics/heatmap.csv}%
\def\qpaintcsv{figures/data/qpaint_kinetics/qpaint_counts.csv}%
\tikzsetnextfilename{qpaint_counting_estimates}%
\begin{tikzpicture}
  \begin{axis}[
    name=trace,
    width=\plotwidth,
    height=\plotheight,
    xlabel=true \n,
    ylabel=estimated \n,
    enlarge x limits=false,
    enlarge y limits=false,
    grid=major,
    grid style={dashed},
    scaled ticks=false,
    ticklabel style={font=\small},
    xtick align=outside,
    xtick pos=lower,
    ytick align=outside,
    ytick pos=lower,
    colorbar,
  ]

    \pgfplotsset{
      colormap={posteriorcolormap}{
        color(0.0)=(white)
        color(0.2)=(funkey_color_2)
        color(1.0)=(funkey_color_2!50!black)
      }
    }

    \addplot[
      matrix plot*,
      mesh/cols=30,
      point meta=explicit,
    ] table [
      col sep=comma,
      x=n_true,
      y=n_fit,
      meta=posterior,
    ] {\posteriormatrixcsv};

  \end{axis}

  \begin{axis}[
    name=trace,
    width=\plotwidth,
    height=\plotheight,
    xmin=0.5,
    xmax=30.5,
    ymin=0.5,
    ymax=35.5,
    grid=major,
    grid style={dashed},
    xticklabel=\empty,
    yticklabel=\empty,
    scaled ticks=false,
    domain=0:30,
    ticklabel style={font=\small},
    legend pos=north west,
  ]

    \addlegendimage{mark=square*,only marks,funkey_color_2}
    \addlegendentry{\ours}
    \addlegendimage{mark=*,only marks,funkey_color_1}
    \addlegendentry{\qpaint}

    \addplot[
      mark=*,
      only marks,
      mark size=1.4pt,
      mark options={draw=white,fill=funkey_color_1,draw opacity=0.6,fill opacity=0.7},
    ] table [
      col sep=comma,
      x=n,
      y=qpaint_count,
    ] {\qpaintcsv};

    \addplot[no marks,thick,gray] {x};

  \end{axis}

\end{tikzpicture}

  \end{panel}

  \caption{
    \panelref{a, b} trace simulated with 10/20 emitters, \pon=0.006, \poff=0.2, 
    and the posterior distribution estimated by \ours
    %
    \panelref{c} At higher molecular counts ($n > 15$) \qpaint begins to underestimate the count, while 
    the posterior of \ours remains accurate.
  }
  \label{fig:results:qpaint_counting}
\end{figure*}


\begin{figure*}[t]

  \begin{panel}{(a)}{\textwidth}
    \hspace{-2mm}%
    \setlength\plotwidth{\textwidth}%
    \setlength\plotheight{36mm}%
    \input{figures/panels/exp_trace_n4.tikz}
    \vspace{-4mm}
  \end{panel}

  \begin{panel}{(b)}{\textwidth}
    \hspace{-2mm}%
    \setlength\plotwidth{\textwidth}%
    \setlength\plotheight{36mm}%
    \def\histogramcsv{figures/data/exp_counting/intensity_histograms.csv}%
\def\tracecsv{figures/data/exp_counting/traces_and_fits.csv}%
\def\posteriorcsv{figures/data/exp_counting/posteriors.csv}%
\def\traceintensitycol{N1_trace_good}%
\def\zcol{N1_fit_good}%
\def\histbincol{N1_good_bins}%
\def\histcountcol{N1_good_measured}%
\def\modelfitcol{N1_good_model}%
\def\posteriorcol{posterior_1}%
\tikzsetnextfilename{exp_trace_n1}%
\begin{tikzpicture}
  \input{figures/plots/fancy_trace.plot}
  \def\noylabels{}
  \def\noxlabel{}
  \setlength\plotwidth{0.1\plotwidth}
  \setlength\plotheight{\plotwidth}
  \pgfplotsset{every axis/.style={anchor=below south east,at={($(histogram.south east)-(2mm,2mm)$)}}}
  \input{figures/plots/p_n_given_x.plot}
\end{tikzpicture}

    \vspace{-4mm}
  \end{panel}

  \begin{panel}{(c)}{0.59\textwidth}
    \vspace{6mm}%
    \hspace{4mm}%
    \setlength\plotwidth{22mm}%
    \tikzsetnextfilename{exp_origami_samples_n4}%
\begin{tikzpicture}

  \matrix[
      matrix of nodes,
      every node/.style={inner sep=0},
      column sep=1mm,
      row sep=1mm,
      ampersand replacement=\&] (samples) {
    \includegraphics[width=\plotwidth]{figures/data/exp_counting/4bs_origami_1.png}
    \&
    \includegraphics[width=\plotwidth]{figures/data/exp_counting/4bs_origami_2.png}
    \&
    \includegraphics[width=\plotwidth]{figures/data/exp_counting/4bs_origami_3.png}
    \&
    \includegraphics[width=\plotwidth]{figures/data/exp_counting/4bs_origami_4.png}
    \\
  };

  % scale bar
  \node[
    rectangle,
    minimum width=0.25\plotwidth, % images are 400px, 1/4 is 100px and that is 20nm
    fill=white,
    anchor=south east,
    inner sep=0pt,
    outer sep=2mm] (scalebar) at (samples-1-4.south east) {};
  \node[anchor=south,white] at (scalebar.center) {\tiny $20 \micrometer$};
\end{tikzpicture}
%
  \end{panel}
  \begin{panel}{(d)}{0.2\textwidth}
    \hspace{2mm}%
    \def\plotwidth{0.8\textwidth}%
    \def\plotheight{20mm}%
    \tikzsetnextfilename{exp_mean_probabilities_n1}%
\begin{tikzpicture}

  \pgfplotsset{every axis/.style={name=n1}}
  \def\posteriorcsv{figures/data/exp_counting/mean_probabilities.csv}
  \def\posteriorcol{n1_prob}
  \input{figures/plots/p_n_given_x.plot}
  \node[anchor=south] at (n1.outer north) {origami with $\n=1$};

\end{tikzpicture}
%
  \end{panel}
  \begin{panel}{(e)}{0.2\textwidth}
    \hspace{2mm}%
    \def\plotwidth{0.8\textwidth}%
    \def\plotheight{20mm}%
    \tikzsetnextfilename{exp_mean_probabilities_n4}%
\begin{tikzpicture}

  \pgfplotsset{every axis/.style={name=n4}}
  \def\posteriorcsv{figures/data/exp_counting/mean_probabilities.csv}
  \def\posteriorcol{n4_prob}
  \input{figures/plots/p_n_given_x.plot}
  \node[anchor=south] at (n4.outer north) {origami with $\n=4$};

\end{tikzpicture}
%
  \end{panel}

  \caption{TODO}
  \label{fig:results:experimental}
\end{figure*}


%-------------
\subsection{Simulated Experiments}
%-------------
% Simulated traces based on experimentally relevant parameters
Running \ours as a forward model to simulate experiment, 
we generated traces for counts $\n=1$ to 30 (\figref{fig:results:sim_counting}a,b).
	%
	We determined experimentally relevant camera parameters from a calibration of the
	sCMOS camera (\parametersc: $\camgain=2.17$, $\camoffset=4791$, $\camvar=774$). 
	%
	We determined kinetic (\parameterst: $\pon=0.036$, $\poff=0.028$), and emission parameters 
	(\parameterse: $\re=2.79$, $\rb=6.77$) from an initial experiment, 
	which closely match those reported in literature \cite{stein_2021}.
	%
	We generated traces for 4000 frames, corresponding to an imaging time of 14 minutes.

% priors were placed on camera parameters, but all others were uniform
For inference, we placed tight empirically determined, Gaussian priors on 
the camera parameters.
	%
	We also placed flat uniform priors on the kinetic parameters, allowing the model 
	to fit blinking rates without external bias,
	%
	and loose priors on the emission parameters (\rb, \re) to prevent 
	over-estimation of count (see \secref{discussion}). 
	% how we got these priors
	We experimentally determined the prior on \rb by averaging the 
	intensity of the pixels immediately outside the region of interest.
	%
	We estimated the prior on \re  by first fitting a count of $\n=1$ to the trace.
	%
	The distribution of all observed $z$-states is unimodal, 
	therefore, whichever \z{}-state is the most common, the second most common 
	will be $\z{} \pm 1$. 
	%
	We observed that fitting $n=1$ captures this specific transition, and provides a good
	estimate of photon emission rate of a single emitter \re.
	%
	
% blinx can count!
\ours accurately fit simulated traces with counts up to $\n=30$ (\figref{fig:results:sim_counting}c).
	%
	For $\n=10$ and below, \ours estimated the true count with a substantially 
	higher likelihood than all other possibilities. 
	%
	Above $\n=10$ the posterior broadened as other possibilities become more likely. 
	%
	The most likely estimate was the true count in 177/300 (59\%) of traces, and 
	was within 1 of the true count in 267/300 (89\%) of all traces.
	%
	Importantly, in many cases \ours is able to fit the true count despite the fact that 
	not every state is occupied \ie the model is not just counting the 
	number of peaks in the intensity histogram (\figref{fig:results:sim_counting}a,b).
	% blinx outperforms lbFCS, counting up to 30 
	This is a tripling in performance compared to \lbfcs, which counted 
	accurately up to $\n=6$ and was within 1 of the true count for 64/300 (21\%) of traces, 
	but failed to estimate higher counts.
	%
	Upon further analysis, the main limitation of lbFCS is the estimation of the 
	intensity of a single emitter, corresponding to \re in \ours. 
	%
	\lbfcs relies on a histogram based method to determine this value, while 
	\ours is able to jointly optimize this parameter with the others.
	%
	Providing this value to \lbfcs led to a substantial recovery of performance.
	% 
	With \re provided, \lbfcs identified 200/300 (67\%) of traces within 1 of the true count.
	%

% Longer traces lead to better fits
A longer trace provides more
information to the model and increases the counting accuracy.
	%
	We simulated and fit traces of lengths ranging from 100 to 10,000 frames, with the same parameters 
	as the previous section (\figref{fig:results:sim_counting}d).
	%
	As expected, accuracy increased and variance of the posterior distribution decreased, 
	as trace length increased.
	% Also notice a underestimation of count with low trace length
	Interestingly for short trace lengths, \ours consistently underestimated the true count. 
	% This is most likely due to the system only occupying a subset of all possible zs in this short time
	This is likely because only a subset of possible states were observed in this short time. % maybe more explanation?

% there is a lower bound to SNR, below a specific SNR, model maximally overcounts
We determined the robustness of our model and the minimum signal-to-noise ratio 
(SNR) needed to achieve an accurate count.
	%
	We artificially increased the variance of the readout noise \camvar, 
	to simulate traces with SNRs ranging from 9 to 1 (where 9 
	corresponds to the parameters previously used)
	%
	Because noise in our model is a function of intensity, quantifying the SNR of a trace 
	is not trivial. 
	%
	For simplicity, here we define SNR as the difference in intensity between 
	the first two states (\z{0} and \z{1}) divided by the standard deviation of the difference. 
	%
	In effect this is the higher bound of SNR for a given trace. 
	%
	\ours shows accurate counting for SNRs $\geq 4$ (\figref{fig:results:sim_counting}e). 
	%
	Interestingly, for  SNR $\leq 4$, \ours estimates the count as 25, 
	the highest \n tested, no matter the true count. 
	%
	This is due to the model adding states to compensate for the wide distribution of intensities  
	(see \secref{discussion}).

%-------------
\subsection{Effect of kinetic parameters}
%-------------
The performance of our model depends on the true kinetic parameters of the system (\pon and \poff).
%
	To find a regime that maximizes our models counting ability, we simulated and fit traces 
	from $\n=1$ to 20, with a range of kinetic parameters, 
	while holding all other parameters constant.
	%
	We ensured that this range covered experimentally relevant kinetics from literature including:
	qPAINT: (\pon, \poff) = (0.006, 0.2) \cite{jungmann_2016} and lbFCS (0.02, 0.02) \cite{stein_2021}. 
	%
	We summarized the resulting posteriors by calculating the expected squared error over all true counts: 
	$\sum w(\truen - \n)^2$, where $w$ is the probability of the estimated count \n 
	(\figref{fig:results:campare_kinetics}a).

The result can be divided into two clear regimes: $\pon < \poff$ and $\pon \geq \poff$.
	%
	In the first regime, where $\pon < \poff$ we see significant limitations to our models counting ability.
	%
	Furthermore, when \pon is substantially lower than \poff, our model loses almost 
	all ability to count and estimates $\truen=1$ or 2, regardless of the true $\truen$.
	%
	In contrast, in the second regime, where $\pon \geq \poff$ we see a heightened ability of our model, 
	and accurate counting up to $n=20$.
	%
	A clear difference between these two regimes can be seen in the distribution 
	of their hidden states \z{} (\figref{fig:results:campare_kinetics}a, insets).
	%
	When $\pon < \poff$, the distribution is shifted towards 0, and a majority 
	of the time is spent in the lowest two \z{}-states (\figref{fig:results:campare_kinetics}c). 
	In this regime, the true count becomes indistinguishable without prior information. 
	%
	When $\pon \geq \poff$, this distribution is centered, or even shifted towards $n$ 
	and a larger portion of the states are visited (\figref{fig:results:campare_kinetics}d). 
	%
	This provides ample information for the model to infer the correct $n$.
	
% Experimental
% Do we need to explain how DNA-PAINT works
In DNA-PAINT, the blinking rate is determined by the kinetics of single-stranded DNA binding,
which in turn are dependent on experimental conditions such as temperature and concentration, 
and the specific DNA sequence~\citep{jungmann_single-molecule_2010}.
	%
	As a result, the kinetic parameters \pon and \poff of our model can be 
	tuned by adjusting the temperature and imager concentration (\figref{fig:results:campare_kinetics}b).
	%
	We imaged DNA-origami with a known count of $\n=1$ at 25\textdegree C and an 
	imager concentration of 10 nM, we measured $\pon=0.028$ and $\poff=0.072$ 
	(\figref{fig:results:campare_kinetics}b),
	%
	which places these conditions firmly in the poor counting 
	accuracy regime of $\pon < \poff$ (\figref{fig:results:campare_kinetics}a). % fix
	%
	In order to move to a more favorable kinetic regime, 
	we increased the imager concentration and decreased the temperature.
	%
	Increasing imager concentration to 30 nM raised \pon to 0.071 and 
	decreasing temperature to 13\textdegree C decreased \poff to 0.037 
	(\figref{fig:results:campare_kinetics}b).
	%
	The effects of temperature and concentration were largely independent 
	of one another, providing precise control over the kinetic parameters. % cite jungmann paper
	%
	We also observed a substantial decrease in SNR with decreasing temperature, 
	and with increasing concentration. 
	%
	This was an expected side effect of increasing concentration, as more imager would increase \rb.
	%
	But the effect of temperature on SNR was surprising. 
	We hypothesize that this is due to the stabilization partial 
	binding between imager and docker strands at low temperatures.
	%
	Accounting for both the increase in counting accuracy and the decrease 
	in SNR, we identified imaging conditions of 20 nM and 13\textdegree C as optimal.

%-------------
\subsection{qPAINT Kinetic Regime}
%-------------
% qPAINT operates in a different kinetic regime that lbFCS
qPAINT relies on the accurate measure of the average dark time between blinking events~\citep{jungmann_2016}. 
	%
	As a result, this method is optimized for an entirely different kinetic regime, where blinking 
	events are short and infrequent (\figref{fig:results:qpaint_counting}a,b).
	% 
	This regime presents a challenge to \ours (\figref{fig:results:campare_kinetics}a top inset). 
	%
	If the only states ever observed are $\z{}=0$ or 1, there is insufficient 
	information to estimate count without prior knowledge.
	% 
	qPAINT faces the same limitation and relies on a calibration 
	of the blinking kinetics of a single binding site.
	% 
	\ours, being a Bayesian model can incorporate a similar calibration 
	as priors of the kinetic parameters.
	%
	With this addition, and the tightening of the priors on \re and \rb, the counting 
	ability of our model is restored (\figref{fig:results:qpaint_counting}c). 
	%
	\ours estimated the true count as the most likely in 133/300 (44\%) of traces, and 
	estimated within 1 of the true count in 269/300 (90\%) of all traces,
	comparable to the favorable kinetic regime (\figref{fig:results:sim_counting}a).
	%
	However, obtaining these priors requires additional experimental steps, and is often not trivial. 
	%
	Thus a tradeoff exists.
	%
	Without any calibration, \ours is able to accurately count in a subset of kinetic conditions ($\pon > \poff$).
	%
	But with an experimental calibration, the counting ability of \ours is expanded to a wider range of kinetic conditions. 

% qPAINT undercounts but blinx does not
Due to the stochastic nature of blinking, multiple emitters can 
be active at any given time-point, which becomes increasingly likely at higher counts.
	%
	This is not compatible with the qPAINT requirement of well separated, 
	single emitter blinking events.
	%
	As a result the measured average dark time is longer than the true average dark time
	and  qPAINT begins to underestimate molecular count 
	(especially noticeable above $\n=20$, see \figref{fig:results:qpaint_counting}c). 
	%
	The activity of multiple emitters at a single timepoint is explicitly modeled in both the 
	intensity and transition models previously derived (\eqref{eq:methods:p_c_given_z} and \eqref{eq:method:transition}). 
	%
	As a result, our model avoids underestimating, and provides an accurate measurement 
	of the molecular count up to $\n=30$.
	
%-------------
\subsection{Experimental Counting}
%-------------
% briefly describe experimental setup
To experimentally validate the counting performance of \ours, we used DNA-Origami, 
which provides nano-scale control over the number and location of emitters~\citep{rothemund_folding_2006}.
	% Why DNA origami
	DNA-Origamis were designed containing 1 and 4 DNA-PAINT docker strands, 
	spaced in a grid 20 \nanometer apart. 
	%
	This distance was specifically chosen, so that the true number of docker strands 
	could be visually confirmed through super-resolution post-processing.
	%
	Incorporation efficiency is roughly 80 percent for each docker site \cite{strauss_2018}, 
	so only a fraction of the origamis were expected to contain all 4 dockers. 
	% 
	Origamis were first imaged at 13\textdegree C, low laser power (1.5\%), and 20 nM imager concentration to 
	collect traces (4,000 frames at 200 ms exposure time) for counting with \ours (\figref{fig:results:experimental}a,b).
	%
	The system was then allowed to warm to 25\textdegree C, a buffer exchange performed, and new imager 
	added at 10 nM.
	%
	The origamis were imaged again at high laser power (40\%),
	and post-processed with Picasso \citep{schnitzbauer_2017} to obtain super-resolution ground truth (\figref{fig:results:experimental}c).
	%
	Only origamis that had a visual count matching the designed count (1 or 4) were selected for analysis with \ours.

Of the 131 traces with a known count of 1, \ours correctly counted 112 (85\%)
and the average the posterior distributions, is shown in \figref{fig:results:experimental}d.
	%
	Possible counts up to $\n=8$ were tested, but the likelihood for a count above $n=3$ was negligible for all traces.
	%
	For the traces with a known count of 4, 71/110 (65\%) were correctly identified as 4, 
	and 103/110 were identified as between 3 and 5. 
	%
	The average posterior distribution is shown in \figref{fig:results:experimental}e.
	Importantly, these traces were not preprocessed or filtered before undergoing \ours counting analysis.


\begin{figure*}
  \includegraphics[width=\linewidth]{figures/simulated_counting.png}
  \caption{A) Trace simulated from 10 emitters, and the posterior distribution estimated by blinx B) simulated from 20 emitters
  C) The blinx posterior is able to accurately estimate significantly higher molecular counts than the current state of the art, lbFCS
  D) As trace length increases, the variance of the blinx posterior decreases E) As the signal to noise ratio increases the variance of 
  the blinx posterior decreases. }
  \label{fig:method:overview}
\end{figure*}

\begin{figure}
  \includegraphics[width=\linewidth]{figures/qpaint_kinetics.png}
  \caption{comapre in bad kinetic regime}
  \label{fig:method:overview}
\end{figure}

\begin{figure*}
  \includegraphics[width=\linewidth]{figures/compare_kinetics.png}
  \caption{A) the counting ability of blinx varies significantly with blinking kinetics of the emitters, showing increased model accuracy when
   $p_{on} > p_{off}$. B) When $p_{on} < p_{off}$ the distribution of observed z states is shifted towards 0 C) When $p_{on} > p_{off}$ the 
   distribution of z states is shifted towards the maximum possible z state (N)}
  \label{fig:method:overview}
\end{figure*}

\begin{figure}
  \includegraphics[width=\linewidth]{figures/experimental_counting.png}
  \caption{experimental counting}
  \label{fig:method:overview}
\end{figure}

\section{Discussion} \label{discussion}
% Overview of highlights
Here, we presented \ours a probabilistic model, to estimate the absolute 
number of subunits in an area to small to be spatially separated. 
    %
    Our model was validated on both synthetic and experimental intensity traces,
    and benchmarked against the current state of the art methods \lbfcs \cite{stein_2021} and \qpaint \cite{jungmann_2016}.
    %
    Although experimentally DNA-PAINT was used, our model is not 
    limited to DNA-PAINT, and is compatible with any system that generates repeated fluorescent blinking.
    %
    We anticipate that \ours will be adaptable to any experimental molecular counting pipeline,
    and provide an increase in accuracy as well as probabilistic results, benefiting downstream analysis.

% Assumptions
There are a few assumptions that are essential to \ours.
    %
    Firstly, it is assumed that all subunits behave independently.
    % 
    In the intensity model, this means that observed intensity will scale linearly with the number of emitters active $z$.
    %
    In the transition model, this means that the blinking kinetics of one spot have no effect on the blinking kinetics of any other.
    %
    In practice, although not common, subunits in extremely close proximity could potentially 
    have steric hinderance effects on neighboring subunits, violating this assumption.
    %
    Other methods like qPAINT make this same assumption and have been shown to work well in experimental 
    systems \cite{fischer_quantitative_2021, jayasinghe_true_2018}. 
    
Secondly, it is assumed that all subunits within a spot have identical properties (\pon, \poff, \re).
    %
    Rather than model each subunit individually, \ours models the sum total behavior of all subunits combined, 
    and as a result each subunit is assumed to be interchangeable with any other within the same spot.
    %
    In applications where non-uniform behavior among subunits is expected, \eg due to 3D conformation, 
    1 subunit might be less accessible to diffusion of imager than the others \cite{civitci_2020}, a decrease in model performance is expected.
    %
    Finally, it is also assumed that all properties remain constant over time. 
    %
    Experimentally, drift in emission properties (\re, \rb) is commonly seen due to an unstable focus, and should be corrected before any processing with \ours.
    % 
    Change in kinetic properties over time is also sometimes observed and could be caused by damage to emitters, or changes in temperature. 
    %
    Unlike the other two, this assumption is not fundamental to the structure of \ours. 
    %
    Future versions of this model could account for these changing parameters.
    % 
    Intentional changes in parameters could even be used to gain more information from the system and further increase model performance. 
    

% Limitations
While, \ours has been designed to be applicable to as many different systems as possible, 
there are a few limitations that should be considered.
    % overestimation of count
    The primary limitation of \ours is the tendency to overestimate the molecular count, 
    especially in traces with low SNR. 
    %
    This is caused by the distribution at the heart of the intensity model. 
    %
    The model expects intensities to be observed in evenly spaced, normally distributed 
    peaks (see histograms in \figref{fig:results:experimental}a,b).
    %
    When an intensity is observed between two of these peaks, it can sometimes be cheaper, 
    in terms of likelihood, to add a new \z{}-state to the model,
    rather than account for that intensity with the given distribution.
    %
    In the extreme limit, an infinite number of states could be used to perfectly explain any intensity trace.
    %
    % mention histogram comparison to identify over-counting?
    %
    This limitation can be avoided by specifying a prior on \re, the photon emission rate of a single fluorophore. 
    %
    A prior on \re effectively increases the cost of adding additional states.

Another limitation is that the intensity model does not capture all the noise in the system.
    %
    As seen in \figref{fig:results:experimental}a, there are a significant amount of frames where the 
    measured intensity is greater than $10^4$ and out of our model's distribution.
    % 
    The frequency of this effect was dependant on the imaging temperature (data not shown).
    %
    As a result, we hypothesize that this effect is associated with the DNA-PAINT imaging system.
    % 
    To compensate for this behavior a baseline probability was incorporated into the intensity model 
    \ie there is a fixed minimum likelihood for observing any intensity.
    %
    Further, to reduce the influence of this effect on counting accuracy, the highest 0.5\% of 
    intensities values were excluded when calculating likelihood (the specific percent is adjustable as a hyperparameter).

% Outlook
Finally, the Bayesian framework of \ours, opens this method to many opportunities of future expansion.
    % 
    For example with the simple modification of setting $\pon=0$, this 
    model could support the counting of photobleaching events. 
    %
    Additionally, by incorporating dynamic priors, our model could 
    also capture changing conditions over time, potentially even increasing the counting ability.
    %extend to include PALM and STORM
    Further, with minor modifications to the transition distribution to account for photobleaching,
    this method could be extended to other stochastically blinking emitters, 
    such as those used in PALM or STORM, opening the door for molecular counting in living samples.


{
  \small
  \bibliographystyle{plain}
  \bibliography{references}
}

\clearpage
% \begin{appendices}
% \renewcommand{\thefigure}{S\arabic{figure}}
% \setcounter{figure}{0}
% \section{Supplementary Methods}
\subsection{DNA-Origami specifics}
\subsection{Extracting traces from images}

\section{Supplementary Figures}

\begin{figure*}
    \includegraphics[width=\linewidth]{../figures/si_preliminary_trace.png}
    \caption{Caption goes here}
    \label{fig:si:preliminary_trace}
\end{figure*}
  

\section{Experimental Notes}

\subsection{DNA-PAINT at cold temperatures}
% DNA sequence design for colder temperatures
% see multiple simultaneous stable binding events with 5xCTC docker and 8nt imager


% \end{appendices}
\end{document}
